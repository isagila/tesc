\subsection{%
  Приложения: касательная плоскость и нормаль к поверхности.%
}

\begin{definition}
  Точка \(M\) поверхности \(F(x, y, z(x, y)) = 0\) (неявное задание поверхности)
  называется обыкновенной, если существуют все первые частные производные и при
  этом они не равны нулю одновременно.
\end{definition}

\begin{definition}
  Точка \(M\) поверхности \(F(x, y, z(x, y)) = 0\) (неявное задание поверхности)
  называется особой, если все первые частные производные равны нулю или одна из
  них не существует.
\end{definition}

\begin{definition}
  Прямая называется касательной к поверхности в точке \(M_0\), если она является
  касательной к какой-либо кривой, лежащей в данной поверхности и проходящей
  через точку \(M_0\).
\end{definition}

\begin{remark}
  Касательных прямых в точке может быть бесконечно много, а может быть несколько
  или вообще не быть.
\end{remark}

\begin{theorem}
  Все касательные к поверхности в обыкновенной точке образуют плоскость.
\end{theorem}

\begin{proof}
  Пусть уравнение поверхности, задано неявно \(F(x, y, z(x, y)) = 0\). Возьмём
  обыкновенную точку на поверхности \(M_0 (x_0, y_0, z_0)\) и сделаем из неё
  \quote{небольшой шаг} по поверхности в точку \(M (x_0 + \Delta x, y_0 + \Delta
  y, z_0 + \Delta z)\). Тогда \(\vec{s} = \vec{M_0M} = (\Delta x, \Delta y,
  \Delta z)\) это вектор по направлению касательной. Проведем через точку
  \(M_0\) кривую \(l\) (лежащую в исследуемой поверхности) и параметризуем её

  \begin{equation*} \label{eq:tgt-is-plane-1} \tag{1}
    l \colon \qquad
    \begin{cases}
      x = \alpha(t) \\
      y = \beta(t) \\
      z = \gamma(t)
    \end{cases}
  \end{equation*}

  Точка \(M_0\) обыкновенная, значит исследуемая функция дифференцируема в этой
  точке. Учитывая, что \(F(x, y, z(x, y)) = 0 \implies \fullder{F}{t} = 0\) по
  формуле производной сложной функции получаем

  \begin{equation*} \label{eq:tgt-is-plane-2} \tag{2}
    \begin{aligned}
      \fullder{F}{t}
      = \partder{F}{x} \cdot \fullder{x}{t}
        + \partder{F}{y} \cdot \fullder{y}{t}
        + \partder{F}{x} \cdot \fullder{z}{t}
    \\
      \prh{\partder{F}{x}, \partder{F}{y}, \partder{F}{z}} \cdot
      \prh{\fullder{x}{t}, \fullder{y}{t}, \fullder{z}{t}}
      = 0
    \end{aligned}
  \end{equation*}

  Рассмотрим каноническое уравнение  касательной в точке \(M_0\) с направляющим
  вектором \(\vec{s} = (\Delta x, \Delta y, \Delta z)\).

  \begin{equation*} \label{eq:tgt-is-plane-3} \tag{3}
    \begin{aligned}
      \frac{x - x_0}{\Delta x}
      = \frac{y - y_0}{\Delta y}
      = \frac{z - z_0}{\Delta z}
    \\
      \frac{x - x_0}{\frac{\Delta x}{\Delta t}} 
      = \frac{y - y_0}{\frac{\Delta y}{\Delta t}} 
      = \frac{z - z_0}{\frac{\Delta z}{\Delta t}}
    \\
      \frac{x - x_0}{\fullder{x}{t}}
      = \frac{y - y_0}{\fullder{y}{t}}
      = \frac{z - z_0}{\fullder{z}{t}}
    \end{aligned}    
  \end{equation*}

  Это значит, что вектор \(\display{\prh{\fullder{x}{t}, \fullder{y}{t},
  \fullder{z}{t}}}\) является направляющим вектор для касательной, т.е. он
  сонаправлен с \(\vec{s}\). Обозначим его \(\frac{\dd \vec{s}}{\dd t}\).

  Вернёмся к \eqref{eq:tgt-is-plane-2}, обозначим \(\display{\vec{n} =
  \prh{\partder{F}{x}, \partder{F}{y}, \partder{F}{z}}}\) и получим \(\vec{n}
  \cdot \frac{\dd \vec{s}}{\dd t} = 0\). Причем т.к. \(M_0\) это обыкновенная
  точка, то оба эти вектора ненулевые, таким  образом они перпендикулярны. В
  итоге имеем вектор \(\vec{n}\), который перпендикулярен вектору по направлению
  любой касательной, т.е. \(\vec{n}\) перпендикулярен любой касательной. Если
  \(\vec{n}\) перпендикулярен любой касательной, то он является вектором нормали
  для плоскости этих касательных.
\end{proof}

\begin{definition}
  Плоскость, содержащая все касательные к поверхности в данной точке, называет
  касательной плоскостью в этой точке.
\end{definition}

\begin{definition}
  Прямая перпендикулярная касательной плоскости в точке касания называется
  нормалью к поверхности в данной точке.
\end{definition}

\begin{remark}
  Существуют два \textbf{вектора} нормали: \quote{положительный} и
  \quote{отрицательный}. У положительного вектора нормали \(\vec{n} = (x, y,
  z)\) коэффициент \(z > 0\), т.е. он образует острый угол о осью \(Oz\).
  Соответственно, у отрицательного вектора нормали \(z < 0\), и он образует
  тупой угол с осью \(Oz\).
\end{remark}

\subheader{Нахождение уравнения касательной плоскости к поверхности}

\begin{enumerate}
\item
  Вычислим \(z_0 = f(x_0, y_0)\). Если \(z_0\) уже дано, т.е. дана точка
  поверхности \(M_0 (x_0, y_0, z_0)\), то этот шаг можно пропустить.

\item
  Запишем данную функцию в неявном виде \(F(x, y, z(x, y)) = 0\). Если уравнение
  поверхности уже дано в неявном виде, то этот шаг можно пропустить.

\item 
  Возьмем частные производные по каждой из трёх переменных в точке \(M_0 (x_0,
  y_0, z_0)\). Здесь \(F(x, y, z)\) рассматривает как функция трех переменных, а
  не как функция двух переменных заданная неявно, поэтому и производные нужно
  вычислять как производные функции трех переменных, а не как производные
  функции двух переменных.

\item
  Искомое уравнение плоскости будет иметь вид

  \begin{equation*}
    F'_x(M_0) \cdot (x - x_0)
    + F'_y(M_0) \cdot (y - y_0)
    + F'_z(M_0) \cdot(z - z_0)
    = 0
  \end{equation*}
\end{enumerate}

\begin{example}
  Пусть требуется найти уравнение касательной плоскости функции к поверхности 
  \(6 x y - 2 x^2 - x y^2 - z^2 + 3 = 0\) в точке \(M_0 (1, 2, 3)\).

  Т.к. \(z_0\) дано и уравнение уже приведено к неявному виду, то вычислим
  частные производные по трём переменных в точке \(M_0\).

  \begin{equation*}
    \begin{aligned}
      F'_x = 6y - 4x - y^2 \implies F'_x(1, 2, 3) = 12 - 4 - 4 = 4
    \\
      F'_y = 6x - 2xy \implies F'_y(1, 2, 3) = 6 - 2 \cdot 2 = 2
    \\
      F'_z = -2z \implies F'_z(1, 2, 3) = -2 \cdot 3 = -6
    \end{aligned}
  \end{equation*}
  
  Запишем уравнение касательной плоскости и упростим его.

  \begin{equation*}
    \begin{aligned}
      4 \cdot (x - 1) + 2 \cdot (y - 2) - 6 \cdot (z - 3) = 0
    \\
      2 x + y - 3 z + 5 = 0
    \end{aligned}
  \end{equation*}
\end{example}

\subheader{Нахождение уравнения нормали к поверхности}

Для того, чтобы составить уравнение нормали, нужна точка (она дана по условию) и
вектор по направлению. Вектор по направлению можно получить из уравнения
касательной плоскости: вектор нормали к касательной плоскости будет вектором по
направлению для искомой прямой. Т.к. коэффициенты \(A\), \(B\), \(C\) в
уравнении плоскости это частные производные (это видно из уравнения касательной
плоскости), то уравнение нормали будет иметь вид

\begin{equation*}
  \frac{x - x_0}{F'_x(M_0)}
  = \frac{y - y_0}{F'_y(M_0)}
  = \frac{z - z_0}{F'_z(M_0)}
\end{equation*}

\begin{example}
  Найдем уравнение нормали в той же точке и к той же поверхности, к которой
  искали уравнение касательной плоскости. Т.к. частные производные уже
  вычислены, то по формуле получаем

  \begin{equation*}
    \frac{x - 1}{4} = \frac{y - 2}{2} = \frac{z - 3}{-6}
  \end{equation*}
\end{example}
