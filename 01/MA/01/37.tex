\subsection{%
  Экстремумы функции двух переменных. Необходимые и достаточные условия.%
}

\begin{definition}
  Точка \(M_0\) называется точкой минимума функции \(z = f(x, y)\), если в
  некоторой проколотой окрестности точки \(M_0\) верно \(z(M) > z(M_0)\).
\end{definition}

\begin{definition}
  Точка \(M_0\) называется точкой максимума функции \(z = f(x, y)\), если в
  некоторой проколотой окрестности точки \(M_0\) верно \(z(M) < z(M_0)\).
\end{definition}

\begin{theorem}[Необходимое условие экстремума функции двух переменных]
  Если функция \(z = f(x, y)\) имеет \textbf{гладкий} экстремум в точке \(M_0\),
  то все её первые частные производные в этой точке равны нулю.
\end{theorem}

\begin{proof}
  Допустим, функция \(z = f(x, y)\) достигает максимума в точке \(M_0 (x_0,
  y_0)\) (для минимума доказательство аналогично), тогда

  \begin{equation*}
    \forall (x, y) \in \nearo{M_0}{\delta} \given  z(x, y) < z(x_0, y_0)
  \end{equation*}

  При этом в плоскости \(y = y_0\) будет выполняться \(z(x, y_0) < z(x_0,
  y_0)\). Т.е. функция одной переменной \(z(x, y_0)\) имеет максимум в точке
  \(x_0\), значит \(z'_x(x_0, y_0) = 0\). Аналогично можно показать, что
  \(z'_y(x_0, y_0) = 0\).
\end{proof}

\begin{theorem}[Достаточное условие экстремума функции двух переменных]
  Пусть функция \(z = f(x, y)\) дифференцируема в точке \(M_0\) дважды, и при
  этом \(M_0\) точка подозрительная на экстремум (в ней выполнено необходимое
  условие экстремума). Введем следующие обозначения

  \begin{equation*}
    \frac{\partial^2 z}{\partial x^2} = A
    \qquad
    \frac{\partial^2 z}{\partial x \partial y} = B
    \qquad
    \frac{\partial^2 z}{\partial y^2} = C
  \end{equation*}

  Вычислим \(\Delta = A C - B^2\), тогда

  \begin{enumerate}
  \item
    \(\Delta < 0 \implies\) нет экстремума. В таком случае функция имеет
    минимакс в данной точке.
  
  \item
    \(\Delta = 0 \implies\) требует дополнительное исследование.
  
  \item
    \(\Delta > 0 \implies\) есть экстремум. Причем при \(A < 0\) это максимум, а
    при \(A > 0\) минимум.
  \end{enumerate}
\end{theorem}

\galleryone{01_37_01}{Достаточное условие экстремума функции двух переменных}

\begin{proof}
  Запишем формулу Тейлора в точке \(M \in \near{M_0}{\delta}\) до \(n = 2\).

  \begin{equation*} \label{thr:suff-ext-2v-1} \tag{1}
    z(M) = z(M_0) + \dd z(M_0) + \frac{\dd^2 z (M_0)}{2} + \smallo(\rho^2)
    \qquad
    (\rho = \sqrt{(\Delta x)^2 + (\Delta y)^2})
  \end{equation*}

  Т.к. \(M_0\) это точка подозрительная на экстремум, то её первые частные
  производные равны нулю, а значит и первый полный дифференциал равен нулю
  
  Пусть \(\Delta x = z(M) - z(M_0)\), а \(\smallo(\rho^2) = k \rho^3 \given k
  \in \RR\). Домножим на два, распишем второй дифференциал и получим
  
  \begin{equation*} \label{thr:suff-ext-2v-2} \tag{2}
    \begin{aligned}
      2 \Delta z = \under{\frac{\partial^2 z}{\partial x^2}}{A} \dd x^2
        + 2 \under{\frac{\partial z^2}{\partial x \partial y}}{B} \dd x \dd y
        + \under{\frac{\partial^2 z}{\partial y^2}}{C} \dd y^2
        + k \rho^3
    \\
      2 \Delta z = A (\Delta x)^2
        + 2 B \cdot \Delta x \cdot \Delta y
        + C (\Delta y)^2
        + k \rho^3
    \\
      \Delta z = \frac{\rho^2}{2} \cdot \prh[\bigg]{
        A \cdot \frac{(\Delta x)^2}{\rho^2}
        + 2 B \cdot \frac{\Delta x}{\rho} \cdot \frac{\Delta y}{\rho}
        + C \frac{(\Delta y)^2}{\rho^2}
        + k \rho
      }
    \end{aligned}
  \end{equation*}
  
  Заменим \(\display{\frac{\Delta x}{\rho} = \cos \alpha}\) и
  \(\display{\frac{\Delta y}{\rho} = \sin \alpha}\) согласно иллюстрации
  (\figref{01_37_01}).

  \begin{equation*} \label{thr:suff-ext-2v-3} \tag{3}
    \Delta z = \frac{\rho^2}{2} \cdot \prh[\Big]{
      A \cdot \cos^2 \alpha
      + 2 B \cos \alpha \sin \alpha
      + C \sin^2 \alpha
      + k \rho
    }
  \end{equation*}
  
  Исследуем знак \(\Delta z\). Т.к. \(\display{\frac{\rho^2}{2} > 0}\), то будем
  работать только с большой скобкой. Приведем первые три слагаемых в скобке к
  общему знаменателю \(A\) и соберем в числителе полный квадрат.

  \begin{equation*} \label{thr:suff-ext-2v-4} \tag{4}
    \begin{aligned}
      \frac{A^2 \cdot \cos^2 \alpha + 2 A B \cos \alpha \sin \alpha
        + A C \sin^2 \alpha}{A} + k \rho
    \\
      \frac{(A \cdot \cos \alpha + B \cdot \sin \alpha)^2 - B^2 \sin^2 \alpha
        + A C \sin^2 \alpha}{A} + k \rho
    \\
      \frac{(A \cdot \cos \alpha + B \cdot \sin \alpha)^2
        + (A C - B^2) \sin^2 \alpha}{A} + k \rho
    \end{aligned}
  \end{equation*}
  
  Рассмотрим случаи.
  
  \subsubheader{Случай I}{\(AC - B^2 > 0\)}
      
  Тогда \(A \neq 0\) (в противном случае \(-B^2 > 0 \implies\) противоречие).
  Рассмотрим числитель дроби (обозначим его \(Q\)).

  \begin{enumerate}
  \item
    Если \(\cos \beta = 0\), то \(\cos \alpha = 1\) и т.к. \(A \neq 0\), то
    \(Q > 0\).

  \item
    Если \(\cos \beta \neq 0\), то т.к. \(A C - B^2 > 0\), то \(Q > 0\).
  \end{enumerate}
      
  Таким образом получили, что знак дроби зависит только от \(A\). При достаточно
  малых \(\rho\) множитель \(k \rho\) не будет вносить вклада в знак \(\Delta
  z\), значит \(\Delta z\) будет иметь тот же знак, что и \(A\).

  \begin{enumerate}
  \item
    \(A < 0 \implies \Delta z < \), значит любой \quote{шаг} из точки \(M_0\)
    уменьшит координату по \(z\), т.е. \(M_0\) это точка максимума.
  
  \item
    \( A > 0 \implies \Delta z > 0\), аналогично получаем, что \(M_0\) это точка
    минимума.
  \end{enumerate}
  
  \subsubheader{Случай II}{\(AC - B^2 = 0\)}
      
  Пусть \(\display{\tg \alpha = -\frac{A}{B}}\), тогда

  \begin{equation*} \label{thr:suff-ext-2v-5} \tag{5}
    A \cdot \cos \alpha + B \cdot \sin \alpha
    = \cos \alpha (A + B \tg \alpha)
    = 0
  \end{equation*}
      
  Таким образом числитель дроби полностью обнуляется, а значит \(\display{\Delta
  z = \frac{k \rho^3}{2}}\) и необходимо дополнительное исследование.
      
  \subsubheader{Случай III}{\(AC - B^2 < 0\)}
      
  Покажем, что в таком случае при подходе к \(M_0\) разными путями \(\Delta z\)
  меняет свой знак, т.е. в точке \(M_0\) нет экстремума. Пусть \(A > 0\), тогда
  рассмотрим направления

  \begin{enumerate}
  \item
    \(\alpha = 0\), тогда вся дробь сокращается до \(A\). Таким образом, т.к.
    \(A > 0\), то и \(\Delta z > 0\).

  \item
    \(\display{\tg \alpha = -\frac{A}{B}}\), в таком случае первая скобка в
    числителе обнуляется (также как это было в случае II). Значит весь числитель
    отрицателен (т.к. \(AC - B^2 < 0\)). Т.к. \(A > 0\), то вся дробь будет
    отрицательна, т.к. \(\Delta z < 0\)
  \end{enumerate}

  При \(A < 0\) случаи рассматриваются аналогично. При \(A = 0\) вернёмся к
  \eqref{thr:suff-ext-2v-3} (т.к. при \(A = 0\) мы не можем привести к общему
  знаменателю \(A\)).
      
  \begin{equation*} \label{thr:suff-ext-2v-6} \tag{6}
    \Delta z = \frac{\rho^2}{2} \cdot \prh[\bigg]{
      A \cdot \cos^2 \alpha
      + 2 B \cos \alpha \sin \alpha
      + C \sin^2 \alpha
      + k \rho
    }
  \end{equation*}
      
  Выберем направление \(\alpha = 0\) и получим, что \(\display{\Delta z =
  \frac{k \rho^3}{2}}\), т.е. знак \(\Delta z\) нельзя точно определить без
  дополнительного исследования. Таким образом мы получили, что при любом \(A\)
  приращение \(\Delta z\) меняет свой знак при переходе через точку \(M_0\),
  значит в этой точке нет экстремума.
\end{proof}
