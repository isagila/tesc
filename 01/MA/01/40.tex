\subsection{%
  Условный экстремум функции двух переменных.%
}

\begin{definition}
  Условным экстремумом функции двух переменных называется экстремум не на всей
  области определения, а только на её части, которая описывается некоторым
  правилом (которое обычно описывается уравнением условия).
\end{definition}

\subheader{Метод множителей Лагранжа для нахождения условного экстремума}

Пусть необходимо найти экстремум функции \(z = f(x, y)\) при условии \(\phi(x,
y) = 0\). Введём вспомогательную функцию \(L(x, y) = f(x, y) + \lambda \cdot
\phi(x, y)\). Тогда для нахождения условного экстремума необходимо решить
следующую систему

\begin{equation*} \label{eq:cond-ext} \tag{JLL}
  \begin{cases}
    L'_x = 0 \\
    L'_y = 0 \\
    \phi(x, y) = 0
  \end{cases}
\end{equation*}

После этого все точки \((x, y)\), являющиеся решением данной системы,
\textbf{необходимо проверить на экстремум} (например с помощью достаточного
условия экстремума функции нескольких переменных).

\begin{remark}
  \(\lambda\) называют множителем Лагранжа, а \(\phi(x, y) = 0\)~--- уравнением
  связи. Если оно задано явно, то его необходимо привести к неявному виду.
\end{remark}

\begin{remark}
  Система \eqref{eq:cond-ext} имеет три уравнения и три неизвестных, так что
  решение (в общем случае) должно найтись, но это не обязательно.
\end{remark}

\begin{proof}
  Покажем, что данный метод действительно работает. Пусть точка \(M_0\) точка
  условного экстремума функции \(z = f(x, y)\) Тогда первые частные производные в
  этой точке равны нулю, а значит и полный первый дифференциал в этой точке равен
  нулю 
  
  \begin{equation*} \label{eq:L-mtd-1} \tag{1}
    \dd z = z'_x(x, y) \dd x + z'_y(x, y) \dd y = 0
  \end{equation*}
  
  Продифференцируем уравнение связи, получим

  \begin{equation*} \label{eq:L-mtd-2} \tag{2}
    \dd \phi = \phi'_x(x, y) \dd x + \phi'_y(x, y) \dd y = 0    
  \end{equation*}
  
  Умножим \eqref{eq:L-mtd-2} на некоторое число \(\lambda\) и прибавим его к
  \eqref{eq:L-mtd-1}.

  \begin{equation*} \label{eq:L-mtd-3} \tag{3}
    \dd z = \prh{z'_x(x, y) + \lambda \phi'_x(x, y)} \dd x
    + \prh{z'_y(x, y) + \lambda \phi'_y(x, y)} \dd y = 0
  \end{equation*}
  
  Т.к. \(\phi(x, y)\) неявно заданная функция, то \(\phi'_x (x, y) \neq 0\).
  Найдём такое \(\lambda\), чтобы первая скобка обнулилась, тогда т.к. сумма
  равна нулю, то и вторая скобка обнулится, значит мы нашли такое
  \(\lambda\), что выполняются два равенства

  \begin{equation*} \label{eq:L-mtd-4} \tag{4}
    \begin{cases}
      z'_x(x, y) + \lambda \phi'_x(x, y) = 0 \\
      z'_y (x, y) + \lambda \phi'_y (x, y) = 0
    \end{cases}  
  \end{equation*}
  
  Рассмотрим вспомогательную функцию \(L(x, y) = z(x, y) + \lambda \phi (x,
  y)\). Равенства \eqref{eq:L-mtd-4} дают нам необходимое (но не достаточное!)
  условие экстремума функции \(L(x, y)\) в точке \((x, y)\). Таким образом имеем
  \(L'_x = 0\) и \(L'_y = 0\). Т.к. у нас есть только необходимое, а не
  достаточное условие, то не все найденные точки будут являться точками
  экстремума, поэтому необходима дополнительная проверка.  
\end{proof}

\begin{remark}
  Этот метод может быть применен и к функциям с б\'oльшим числом аргументов,
  например для функции \(u = f(x, y, z)\) и уравнения связи \(\phi(x, y, z) =
  0\) составим вспомогательную функцию \(L(x, y, z) = f(x, y, z) + \lambda \cdot
  \phi(x, y, z)\). Тогда для нахождения условного экстремума необходимо решить
  следующую систему

  \begin{equation*}
    \begin{cases}
      L'_x = 0 \\
      L'_y = 0 \\
      L'_z = 0 \\
      \phi(x, y, z) = 0
    \end{cases}
  \end{equation*}
\end{remark}
