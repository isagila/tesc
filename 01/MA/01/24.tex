\subsection{%
  Теоремы о дифференцируемых функциях. Теорема Ролля.%
}

\begin{theorem}[Ролля]
 \begin{equation*}
  \begin{rcases}
    f(x) \isdiff{\segment{a}{b}} \\
    f(a) = f(b)
  \end{rcases}
  \implies \exists \xi \in \interval{a}{b} \given f'(\xi) = 0
 \end{equation*}
\end{theorem}

\begin{proof}  
  Т.к. функция дифференцируема на отрезке, то она непрерывна на этом отрезке,
  значит по второй теореме Вейерштрасса она принимает наибольшее и наименьшее
  значения на этом отрезке (\(M\) и \(m\)). Если наибольшее или наименьшее
  значения достигаются на концах отрезка, то \(f(a) = f(b) = M = m\). Значит по
  второй теореме Больцано-Коши функция принимает все значения от \(m\) до \(M\),
  т.е. является константой. Таким образом её производная равна нулю в любой
  точке отрезка.

  Если наибольшее или наименьшее значения достигаются внутри, то это означает,
  что внутри отрезка у функции есть экстремум. Таким образом, по лемме Ферма,
  если функция дифференцируема в точке и при этом в этой точке достигается
  экстремум, то производная в этой точке равна нулю. В обоих случаях нашлась
  некоторая точка, производная в которой равна нулю.
\end{proof}

Геометрический смысл теоремы Ролля заключается в том, что она говорит о смене
монотонности функции (\figref{01_24_01}).

\galleryone{01_24_01}{Геометрический смысл теоремы Ролля}
