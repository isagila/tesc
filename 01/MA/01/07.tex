\subsection{%
  Теоремы о пределах.%
}

\begin{theorem} \label{thr:func-as-lim}
  Если \(\lim_{x \to a} f(x) = L\), то говорят, что функцию \(f(x)\) можно
  представить пределом \(f(x) = L + \infsmall(x)\), где \(\infsmall(x)\)~---
  б.м. в точке \(a\)
\end{theorem}

\begin{proof}
  По условию теоремы \(f(x) = L + \infsmall(x) \implies f(x) - L =
  \infsmall(x)\). Подставим это в определение предела.

  \begin{equation*}
    \lim_{x \to a} f(x) = L \iff
    \forall \epsilon > 0 \exists \delta > 0 \given
    \forall x \in \nearo{a}{\delta} \cap E \implies
    \abs{f(x) - L} < \epsilon
  \end{equation*}

  Мы показали, что \(f(x) - L = \infsmall(x) < \epsilon\), значит определение
  верно.
\end{proof}

\begin{theorem}
  Предел константы равен ей самой.

  \begin{equation*}
    \lim_{x \to a} c = c
    \qquad
    (c = const)
  \end{equation*}
\end{theorem}

\begin{proof}
  Распишем определение предела через неравенства.

  \begin{equation*}
    \forall \epsilon > 0 \exists \delta > 0 \given
    \forall x \colon 0 < \abs{x - a} < \delta \implies
    \abs{c - c} < \epsilon
  \end{equation*}

  Последнее неравенство всегда выполняется, т.к. \(\epsilon > 0\).
\end{proof}

\begin{theorem}
  Предел суммы равен сумме пределов.

  \begin{equation*}
    \lim_{x \to a} (f(x) + g(x)) = \lim_{x \to a} f(x) + \lim_{x \to a} g(x)
  \end{equation*}
\end{theorem}

\begin{proof}
  Воспользуемся представлением функции пределом (\ref{thr:func-as-lim}) и
  получим

  \begin{equation*}
    \begin{aligned}
      \lim_{x \to a} f(x) = A \iff f(x) = A + \infsmall_1(x)
    \\
      \lim_{x \to a} g(x) = B \iff g(x) = B + \infsmall_2(x)
    \end{aligned}
  \end{equation*}

  Сложим полученные равенства и применим свойство суммы бесконечно малых
  функций (\ref{thr:sum-inf-a}).

  \begin{equation*}
    \begin{aligned}
      f(x) + g(x) = A + B + \infsmall_1(x) + \infsmall_2(x)
    \\
      (f(x) + g(x)) = (A + B) + \infsmall(x)
    \end{aligned}
  \end{equation*}

  Далее воспользуемся представлением функции пределом в обратную сторону.

  \begin{equation*}
    \begin{aligned}
      \lim_{x \to a} (f(x) + g(x)) = A + B
    \\
      \lim_{x \to a} (f(x) + g(x)) = \lim_{x \to a} f(x) + \lim_{x \to a} g(x)
    \end{aligned}
  \end{equation*}
\end{proof}

\begin{theorem}
  Предел произведения равен произведению пределов.

  \begin{equation*}
    \lim_{x \to a} (f(x) \cdot  g(x))
    = \lim_{x \to a} f(x) \cdot  \lim_{x \to a} g(x)
  \end{equation*}
\end{theorem}

\begin{proof}
  Воспользуемся представлением функции пределом (\ref{thr:func-as-lim}) и
  получим

  \begin{equation*}
    \begin{aligned}
      \lim_{x \to a} f(x) = A \iff f(x) = A + \infsmall_1(x)
    \\
      \lim_{x \to a} g(x) = B \iff g(x) = B + \infsmall_2(x)
    \end{aligned}
  \end{equation*}

  Перемножим полученные равенства и применим свойства бесконечно малых функций.

  \begin{equation*}
    \begin{aligned}
      f(x) \cdot g(x) = A B + A \infsmall_2(x) + B \infsmall_1(x)
        + \infsmall_1(x) \infsmall_2(x)
    \\
      (f(x) \cdot g(x)) = A B + \infsmall(x)
    \end{aligned}
  \end{equation*}

  Далее воспользуемся представлением функции пределом в обратную сторону.

  \begin{equation*}
    \begin{aligned}
      \lim_{x \to a} (f(x) \cdot g(x)) = A B
    \\
      \lim_{x \to a} (f(x) \cdot g(x))
      = \lim_{x \to a} f(x) \cdot \lim_{x \to a} g(x)
    \end{aligned}
  \end{equation*}
\end{proof}

\begin{theorem}
  Предел частного равен частному пределов (при условии, что знаменатель не равен
  нулю).

  \begin{equation*}
    \lim_{x \to a} \frac{f(x)}{g(x)}
    = \frac{\lim_{x \to a} f(x)}{\lim_{x \to a} g(x)}
  \end{equation*}
\end{theorem}

\begin{proof}
  Воспользуемся представлением функции пределом (\ref{thr:func-as-lim}) и
  получим

  \begin{equation*} \label{eq:lim-div-1} \tag{1}
    \begin{aligned}
      \lim_{x \to a} f(x) = A \iff f(x) = A + \infsmall_1(x)
    \\
      \lim_{x \to a} g(x) = B \iff g(x) = B + \infsmall_2(x)
    \end{aligned}
  \end{equation*}

  Рассмотрим следующий предел

  \begin{equation*} \label{eq:lim-div-2} \tag{2}
    \lim_{x \to a} \prh{\frac{f(x)}{g(x)} - \frac{A}{B}}
  \end{equation*}

  Подставим \eqref{eq:lim-div-1} в \eqref{eq:lim-div-2} и упростим, пользуясь
  свойствами б.м. функций.

  \begin{equation*}
    \lim_{x \to a} \prh{\frac{A + \infsmall_1(x)}{B + \infsmall_2(x)}
      - \frac{A}{B}}
    = \lim_{x \to a} \prh{\frac{A B + \infsmall_1(x) B - A B
      - \infsmall_2(x) A}{(B + \infsmall_2(x)) B}}
    = \lim_{x \to a} \prh{\frac{\infsmall_1(x) - \infsmall_2(x)}{(B +
      \infsmall_2(x)) B}}
    = \frac{0 - 0}{(B + 0) B}
    = 0
  \end{equation*}

  Получили, что предел \eqref{eq:lim-div-2} равен нулю. Это равносильно
  следующему

  \begin{equation*}
    \lim_{x \to a} \frac{f(x)}{g(x)}
    = \frac{A}{B}
    = \frac{\lim_{x \to a} f(x)}{\lim_{x \to a} g(x)}
  \end{equation*}
\end{proof}
