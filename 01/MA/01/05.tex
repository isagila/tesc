\subsection{%
  Предельный переход в неравенствах. Теорема о двух милиционерах.%
}

\begin{theorem} \label{thr:lim-func}
  Если функция \(f(x)\) имеет предел в точке \(a\), то она ограничена в
  некоторой проколотой окрестности этой точки.
\end{theorem}

\begin{proof}
  Раскроем модуль в определении предела.

  \begin{equation*}
    \begin{aligned}
      \lim_{x \to a} f(x) = L \iff
      \forall \epsilon > 0 \exists \delta > 0 \given
      \forall x \in \nearo{a}{\delta} \cap E \implies
      \abs{f(x) - L} < \epsilon
    \\
      L - \epsilon < f(x) < L + \epsilon
    \end{aligned}
  \end{equation*}

  Получается, что функция \(f(x)\) ограничена в проколотой окрестности точки
  \(a\).
\end{proof}

\begin{theorem}[О стабилизации знака] \label{thr:stable-sign}
  Если \(\lim_{x \to a} f(x) = A > 0\), то \(f(x) > 0\) в некоторой окрестности
  \(a\).
\end{theorem}

\begin{proof}
  По теореме \ref{thr:lim-func}, получаем, что \(f(x)\) ограничена в некоторой
  проколотой \(\epsilon\)--окрестности точки \(a\), т.е. \(A - \epsilon < f(x)
  < A + \epsilon\). Пусть \(\epsilon = \frac{A}{2}\), тогда \(f(x) > A -
  \frac{A}{2} > 0\).
\end{proof}

\begin{theorem}[О предельном переходе в неравенствах]
  Если \(f(x) \le g(x)\) в некоторой окрестности точки \(a\), то 
  
  \begin{equation*}
    \begin{rcases}
      \lim_{x \to a} f(x) = A \\
      \lim_{x \to a} g(x) = B
    \end{rcases}
    \implies
    A \le B
  \end{equation*}
\end{theorem}

\begin{proof}
  От противного: пусть \(A > B\), тогда

  \begin{equation*}
    \lim_{x \to a} f(x) - \lim_{x \to a} g(x)
    = \lim_{x \to a} (f(x) - g(x))
    = A - B > 0
  \end{equation*}

  Тогда по \ref{thr:stable-sign} \(f(x) - g(x) > 0\), т.е. \(f(x) > g(x)\) в
  некоторой проколотой \(\delta_1\)--окрестности точки \(a\). По условию \(f(x)
  \le g(x)\) в некоторой \(\delta_2\)-окрестности точки \(a\). Пусть \(\delta =
  \min(\delta_1, \delta_2)\), тогда по свойству окрестностей оба условия должны
  выполняться, т.е. \(f(x) \le g(x)\) и \(f(x) > g(x)\). Противоречие.
\end{proof}

\begin{theorem}[О двух милиционерах] \label{thr:squeezed-func}
  Если \(f(x) \le h(x) \le g(x)\) в некоторой проколотой окрестности точки
  \(a\), то

  \begin{equation*}
    \begin{rcases}
      \lim_{x \to a} f(x) = A \\
      \lim_{x \to a} g(x) = A
    \end{rcases}
    \implies
    \lim_{x \to a}h(x) = A
  \end{equation*}
\end{theorem}

\begin{proof}
  Запишем условие формально, раскрыв два предела по определению

  \begin{equation*}
    \begin{aligned}
      \lim_{x \to a} f(x)  = A \iff
      \forall \epsilon > 0 \exists \delta_1 > 0 \given
      \forall x \in \nearo{a}{\delta_1} \cap X \implies
      \abs{f(x) - A} < \epsilon
    \\
      \lim_{x \to a} g(x) = A \iff
      \forall \epsilon > 0 \exists \delta_2 > 0 \given
      \forall x \in \nearo{a}{\delta_2} \cap X \implies
      \abs{g(x) - A} < \epsilon
    \\
      \exists \delta_3 > 0 \given
      \forall x \in \nearo{a}{\delta_3} \cap X \colon
      f(x) \le h(x) \le g(x)
    \end{aligned}
  \end{equation*}

  Пусть \(\delta = \min(\delta_1, \delta_2, \delta_3)\), тогда по свойству
  окрестностей все условия будут выполняться. Раскроем модули по определению и
  \quote{склеим} полученные неравенства.

  \begin{equation*}
    \begin{rcases}
      A - \epsilon < f(x) < A + \epsilon \\
      A - \epsilon < g(x) < A + \epsilon
    \end{rcases}
    \implies
    A - \epsilon < f(x) \le h(x) \le g(x) < A + \epsilon
    \implies
    A - \epsilon < h(x) < A + \epsilon
  \end{equation*}

  Причем последнее неравенство выполняется в проколотой окрестности точки \(a\)
  для любого \(\epsilon\). Таким образом \(\lim_{x \to a} f(x) = A\) по
  определению предела.
\end{proof}

Теорема \ref{thr:squeezed-func} также называется теоремой о двух жандармах или
теоремой о сжатой функции.
