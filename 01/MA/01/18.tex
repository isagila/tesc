\subsection{%
  Производные элементарных функций: константа, степенная функция.%
}

\begin{theorem}
  \begin{equation*}
    f(x) = c \implies f'(x) = 0
  \end{equation*}
\end{theorem}

\begin{proof}
  По определению производной

  \begin{equation*}
    f'(x) = \lim_{\Delta x \to 0} \frac{\Delta y}{\Delta x}
  \end{equation*}

  Т.к. \(f(x) = c\), то \(\Delta y = 0\), а значит

  \begin{equation*}
    f'(x) = \lim_{\Delta x \to 0} \frac{0}{\Delta x} = 0
  \end{equation*}
\end{proof}

\begin{theorem}
  \begin{equation*}
    f(x) = x^n \implies f'(x) = n x^{n - 1}
  \end{equation*}
\end{theorem}

\begin{proof}
  По определению производной
  
  \begin{equation*}
    f'(x)
    = \lim_{\Delta x \to 0} \frac{\Delta y}{\Delta x}
    = \lim_{\Delta x \to 0} \frac{f(x_0 + \Delta x) - f(x_0)}{\Delta x}    
  \end{equation*}

  Воспользуемся биномом Ньютона и вычислим полученную дробь.

  \begin{equation*}
    \begin{aligned}
      \frac{f(x_0 + \Delta x) - f(x_0)}{\Delta x}
    = \\
      \frac{(x_0 + \Delta x)^n - x_0^n}{\Delta x}
    = \\
      \frac{x_0^n + n x_0^{n - 1} \Delta x + \dotsc + (\Delta x)^n -
        x_0^n}{\Delta x}
    = \\
      \frac{n x_0^{n - 1} \Delta x + \dotsc + (\Delta x)^n}{\Delta x}
    = \\
      n x_0^{n - 1} + \dotsc + (\Delta x)^{n - 1}
    \end{aligned}
  \end{equation*}

  Заметим, что все слагаемые, кроме первого содержат \(\Delta x\), причем мы
  рассматриваем предел при \(\Delta x \to 0\). Это значит, что в пределе все
  слагаемые кроме первого обнулятся и получится

  \begin{equation*}
    f'(x)
    = \lim_{\Delta x \to 0} (n x_0^{n - 1} + \dotsc + (\Delta x)^{n - 1})
    = n x_0^{n - 1}
  \end{equation*}
\end{proof}
