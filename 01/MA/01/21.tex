\subsection{%
  Производные элементарных функций: тангенс и арктангенс.%
}

\begin{theorem}
  \begin{equation*}
    f(x) = \tg x \implies f'(x) = \frac{1}{\cos^2 x}
  \end{equation*}  
\end{theorem}

\begin{proof}
  По определению тангенса

  \begin{equation*}
    \tg x = \frac{\sin x}{\cos x}
    \implies
    f'(x) = \prh{\frac{\sin x}{\cos x}}'
  \end{equation*}

  Далее пользуемся формулами производной частного, синуса и косинуса, а также
  основным тригонометрическим тождеством. Итого

  \begin{equation*}
    f'(x)
    = \frac{(\sin x)' \cos x - (\cos x)' \sin x}{\cos^2 x}
    = \frac{\cos^2 x + \sin^2 x}{\cos^2 x}
    = \frac{1}{\cos^ 2 x}
  \end{equation*}
\end{proof}

\begin{theorem}
  \begin{equation*}
    f(x) = \arctg x \implies f'(x) = \frac{1}{1 + x^2}
  \end{equation*}  
\end{theorem}

\begin{proof}
  Рассмотрим обратную функцию \(x(y) = \tg y\). Её производная будет равна
  \(\display{\frac{1}{\cos^2 y}}\). По формуле производной обратной функции
  получаем, что

  \begin{equation*}
    f'(x) = \frac{1}{1 / \cos^2 y}
  \end{equation*}

  Подставим в это равенство известную тригонометрическую формулу
  \(\display{\frac{1}{\cos^2y} = \tg^2 y + 1}\). Учитывая, что \(y = \arctg x\),
  получаем

  \begin{equation*}
    f'(x)
    = \frac{1}{\tg^2 y + 1}
    = \frac{1}{x^2 + 1}
  \end{equation*}
\end{proof}
