\subsection{%
  Правила дифференцирования: производная и дифференциал суммы и отношения
  функций.%
}

Про производную суммы написано ранее (\ref{sec:01-15}).

\begin{theorem}
  Производная отношения функций равна
  
  \begin{equation*}
    \prh{\frac{f(x)}{g(x)}}'
    = \frac{f'(x) \cdot g(x) - f(x) \cdot g'(x)}{g^2(x)}    
  \end{equation*}
\end{theorem}

\begin{proof}
  Воспользуемся определением производной.

  \begin{equation*}
    \prh{\frac{f(x)}{g(x)}}'
    = \lim_{\Delta x \to 0} \frac{\Delta \prh{\frac{f(x)}{g(x)}}}{\Delta x}
    = \lim_{\Delta x \to 0} \frac{\frac{f(x + \Delta x)}{g(x + \Delta x)} -
      \frac{f(x)}{g(x)}}{\Delta x}
    = \lim_{\Delta x \to 0} \frac{f(x + \Delta x) \cdot g(x) - f(x) \cdot g(x +
      \Delta x)}{g(x + \Delta x) \cdot g(x) \cdot \Delta x}
  \end{equation*}

  Разделим почленно и применим свойства пределов.

  \begin{equation*}
    \prh{\frac{f(x)}{g(x)}}'
    = \lim_{\Delta x \to 0} \frac{f(x + \Delta x)}{\Delta x} \cdot
      \lim_{\Delta x \to 0} \frac{g(x)}{g(x + \Delta x) \cdot g(x)}
      - \lim_{\Delta x \to 0} \frac{g(x + \Delta x)}{\Delta x} \cdot
      \lim_{\Delta x \to 0} \frac{f(x)}{g(x + \Delta x) \cdot g(x)}
  \end{equation*}

  По определению производной заменим два предела производными, а оставшиеся
  пределы просто вычислим и получим искомую формулу.

  \begin{equation*}
    f'(x) \cdot \frac{g(x)}{g^2(x)} - g'(x) \cdot \frac{f(x)}{g^2(x)}
    = \frac{f'(x) \cdot g(x) - f(x) \cdot g'(x)}{g^2(x)}    
  \end{equation*}
\end{proof}

\begin{remark}
  Дифференциал вычисляется аналогично производной.

  \begin{equation*}
    \dd \prh{\frac{f(x)}{g(x)}}
    = \frac{\dd (f(x)) \cdot g(x) - f(x) \cdot \dd (g(x))}{g^2(x)}    
  \end{equation*}
\end{remark}
