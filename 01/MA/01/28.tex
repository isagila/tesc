\subsection{%
  Формула Тейлора.%
}

Т.к. исследовать некоторые функции достаточно трудно, мы можем заменить их
некоторым многочленом \(P_n(x - x_0)\), так чтобы \(f(x) \sim P_n(x - x_0)\) при
\(x \to x_0\).

\begin{theorem}[Формула Тейлора]
  \begin{equation*}
    f(x) \isdiffd[n]{x_0}
    \implies
    f(x) = \sum_{k = 0}^n \frac{f^{(k)}(x_0)}{k!} \cdot (x - x_0)^k + r_n(f, x)
  \end{equation*}

  То, что находится под знаком суммы называется многочленом Тейлора, а \(r_n(f,
  x)\) называется остаточным членом.
\end{theorem}

\begin{proof}
  Предположим, что

  \begin{equation*}
    P_n(x) = c_0 + c_1 (x - x_0)^1 + \dotsc + c_n (x - x_0)^n
  \end{equation*}

  Выразим коэффициенты \(c_i\).

  \begin{equation*}
    \begin{aligned}
      c_0          & = P_n(x_0)       & = f(x_0)
    \\
      c_1          & = P'_n(x_0)      & = f'(x_0)
    \\
      2 \cdot c_2  & = P''_n(x_0)     & = f''(x_0)
    \\
      \dotsc
    \\
      i! \cdot c_i & = P^{(i)}_n(x_0) & = f^{(i)}(x_0)
    \end{aligned}
  \end{equation*}

  Таким образом \(\display{c_i = \frac{f^{(i)}(x_0)}{i!}}\), значит получим
  следующий многочлен

  \begin{equation*}
    f(x_0) + f'(x_0)(x - x_0) + \frac{f''(x_0)(x - x_0)^2}{2} + \dotsc
  \end{equation*}

  Т.к. этот многочлен является лишь приближением функции, то добавим к нему
  остаточный член \(r_n(f, x)\), который отвечает за погрешность приближения.
\end{proof}

\begin{lemma} \label{lem:der-rem-term}
  \begin{equation*}
    f(x) \isdiffd{x_0} \implies r'_n(f, x) = r_{n - 1}(f', x)
  \end{equation*}
\end{lemma}

\begin{proof}
  Выразим остаточный член из формулы Тейлора.

  \begin{equation*}
    r_n(f, x) = f(x) - \sum_{k = 0}^n \frac{f^{(k)}(x_0)}{k!} \cdot (x - x_0)^k
  \end{equation*}

  Теперь возьмём производную (по \(x\)) и упростим.

  \begin{equation*}
    \begin{aligned}
      (r_n(f, x))' = f'(x) - \sum_{k = 1}^n \frac{f^{(k)}(x_0)}{k!} \cdot k
        \cdot (x - x_0)^{k - 1}
    \\
      (r_n(f, x))' = f'(x) - \sum_{k = 1}^n \frac{f^{(k)}(x_0)}{(k - 1)!}
        \cdot (x - x_0)^{k - 1}
    \\
      (r_n(f, x))' = f'(x) - \sum_{k = 0}^{n - 1} \frac{f^{(k + 1)}(x_0)}{k!}
        \cdot (x - x_0)^k
    \end{aligned}
  \end{equation*}

  Далее заметим, что \(f^{(k + 1)} = (f')^{(k)}\), получим

  \begin{equation*}
    (r_n(f, x))' = f'(x) - \sum_{k = 0}^{n - 1}\ \frac{(f')^{(k)}(x_0)}{k!}
      \cdot (x - x_0)^k
  \end{equation*}

  Правая часть является остаточным членом формулы Тейлора для \(f'(x)\) значит
  \(\display{(r_n(f, x))' = r_{n - 1}(f', x)}\).
\end{proof}

\begin{theorem}[Остаточный член формулы Тейлора в форме Пеано]
  \begin{equation*}
    f(x) \isdiffd[n - 1]{x_0}
    \implies
    r_n(f, x) = \smallo((x - x_0)^n)
    \qquad
    (x \to x_0)
  \end{equation*}
\end{theorem}

\begin{proof}
  Мат. индукция. 
  
  \subsubheader{База}{\(n = 1\)}

  При \(n = 1\) формула Тейлора имеет вид

  \begin{equation*} \label{eq:T-Peano-1} \tag{1}
    \begin{aligned}
      f(x) = f(x_0) + f'(x)(x - x_0) + \smallo(x - x_0)
    \\
      \under{f(x) - f(x_0)}{\Delta y}
      = f'(x) \under{(x - x_0)}{\Delta x}
      \smallo \under{(x - x_0)}{\Delta x}
    \end{aligned}
  \end{equation*}
  
  Это верно по критерию дифференцируемости, т.к. \(f(x) \isdiffd{x_0}\).

  \subsubheader{Переход}{\(n - 1 \to n\)}

  Если в формуле Тейлора \(x = x_0\), то \(r_n(f, x_0) = 0\). Значит \(r_n(f, x)
  = r_n(f, x) - 0 = r_n(f, x) - r_n(f, x_0)\). Т.к. \(f(x)\) дифференцируема \(n
  - 1\) раз в окрестности точки \(x_0\), то по теореме Лагранжа получаем

  \begin{equation*} \label{eq:T-Peano-2} \tag{2}
    r_n(f, x) - r_n(f, x_0) = (r_n(f, \xi))' (x - x_0)
  \end{equation*}

  где \(\xi\) лежит между \(x\) и \(x_0\). Мы не можем сказать, что \(\xi \in
  \interval{x}{x_0}\), т.к не знаем, что больше \(x\) или \(x_0\). По лемме о
  производной остаточного члена (\ref{lem:der-rem-term}) получаем

  \begin{equation*} \label{eq:T-Peano-3} \tag{3}
    (r_n(f, \xi))'(x - x_0) = r_{n - 1}(f', \xi)(x - x_0)
  \end{equation*}

  Причем если \(f(x)\) дифференцируема \(n - 1\) раз в некоторой окрестности
  точки \(x_0\), то \(f'(x)\) дифференцируема \(n - 2\) раз в той же
  окрестности, значит мы можем выполнить индукционный переход.

  Т.к. мы предполагали, что \(r_n(f, x) = \smallo((x - x_0)^n)\), то

  \begin{equation*} \label{eq:T-Peano-4} \tag{4}
    r_{n - 1}(f', \xi)(x - x_0) = \smallo((\xi - x_0)^{n - 1})(x - x_0)
  \end{equation*}

  Т.к. \(\xi\) лежит между \(x\) и \(x_0\), то \(\abs{\xi - x_0} < \abs{x -
  x_0}\), значит

  \begin{equation*} \label{eq:T-Peano-5} \tag{5}
    \smallo((\xi - x_0)^{n - 1}) = \smallo((x - x_0)^{n - 1})
  \end{equation*}

  В итоге получаем равенство

  \begin{equation*}\label{eq:T-Peano-6} \tag{6}
    r_{n - 1}(f', \xi)(x - x_0) = \smallo((x - x_0)^n)
  \end{equation*}

  Приравняем \eqref{eq:T-Peano-3} и \eqref{eq:T-Peano-6} и подставим это в
  \eqref{eq:T-Peano-2}, получим искомое равенство \(r_n(f, x) = \smallo((x -
  x_0)^n)\).
\end{proof}

\begin{theorem}[Остаточный член формулы Тейлора в форме Лагранжа]
  \begin{equation*}
    \begin{rcases}
      f^{(n)}(x) \iscont{\segment{x_0}{x}} \\
      \exists f^{(n + 1)} \in \interval{x_0}{x}
    \end{rcases}
    \implies
    r_n(f, x) = \frac{f^{(n + 1)}(\xi)}{(n + 1)!} \cdot (x - x_0)^{n + 1}
    \qquad
    (\xi \in \interval{x_0}{x})
  \end{equation*}
\end{theorem}

\begin{proof}
  Мат. индукция.

  \subsubheader{База}{\(n = 0\)}

  При \(n = 0\) формула Тейлора имеет вид

  \begin{equation*} \label{eq:T-L-1} \tag{1}
    \begin{aligned}
      f(x) = f(x_0) + f'(\xi) \cdot (x - x_0)
      \qquad
      \xi \in \interval{x_0}{x}
    \\
      \frac{f(x) - f(x_0)}{x - x_0} = f'(\xi)
    \end{aligned}
  \end{equation*}
  
  Это верно по теореме Лагранжа.

  \subsubheader{Переход}{\(\frac{f^{(n + 1)}(\xi)}{(n + 1)!}\)}

  Пусть формула 
  
  \begin{equation*} \label{eq:T-L-2} \tag{2}
    r_{n - 1}(f, x) = \frac{f^{(n)}(\xi)}{n!} \cdot (x - x_0)^n
  \end{equation*}

  справедлива для \(n - 1\). Выразим \(\display{\frac{r_{n}(f, x)}{(x - x_0)^{n
  + 1}}}\). Если в формуле Тейлора \(x = x_0\), то \(r_n(f, x_0) = 0\), значит
  
  \begin{equation*} \label{eq:T-L-3} \tag{3}
    \frac{r_{n}(f, x)}{(x - x_0)^{n + 1}}
    = \frac{r_n(f, x) - 0}{(x - x_0)^{n + 1} - 0}
    = \frac{r_n(f, x) - r_n(f, x_0)}{(x - x_0)^{n + 1} - (x_0 - x_0)^{n + 1}}
  \end{equation*}
  
  Применяя вначале формулу Коши, а потом лемму о производной остаточного члена
  (\ref{lem:der-rem-term}) получаем

  \begin{equation*} \label{eq:T-L-5} \tag{5}
    \exists \mu \in \interval{x_0}{x} \given
    \frac{r_n(f, x) - r_n(f, x_0)}{(x - x_0)^{n + 1} - (x_0 - x_0)^{n + 1}}
    = \frac{(r_n(f, \mu))'}{(n + 1) \cdot (\mu - x_0)^n}
    = \frac{r_{n - 1}(f', \mu)}{(n + 1) \cdot (\mu - x_0)^n}
  \end{equation*}

  Т.к. мы предполагали, что формула \eqref{eq:T-L-2} истинна то 
  
  \begin{equation*} \label{eq:T-L-6} \tag{6}
    r_{n - 1}(f', \mu) = \frac{(f')^{(n)}(\xi)}{n!} \cdot (\mu - x_0)^n
  \end{equation*}
  
  Подставим это в \eqref{eq:T-L-5}
  
  \begin{equation*}
    \frac{r_{n - 1}(f', \mu)}{(n + 1) \cdot (\mu - x_0)^n}
    = \frac{f^{(n + 1)}(\xi) \cdot (\mu - x_0)^n}{n! \cdot (n + 1) \cdot (\mu -
      x_0)^n}
    = \frac{f^{(n + 1)}(\xi)}{(n + 1)!}
  \end{equation*}
  
  Значит 

  \begin{equation*}
    \begin{aligned}
      \frac{r_{n}(f, x)}{(x - x_0)^{n + 1}} =  \frac{f^{(n + 1)}(\xi)}{(n + 1)!}
    \\
      r_{n}(f, x) = \frac{f^{(n + 1)}(\xi)}{(n + 1)!} \cdot (x - x_0)^{n + 1}
    \end{aligned}
  \end{equation*}  
\end{proof}

\begin{definition}
  Формулой Маклорена называется формула Тейлора при \(x_0 = 0\) и с остаточным
  членом в форме Пеано.

  \begin{equation*}
    f(x) = \sum_{k = 0}^n \frac{f^{(k)}(0)}{k!} \cdot x^k + \smallo(x^n)
  \end{equation*}
\end{definition}

\begin{remark}
  Разложения некоторых функций по формуле Маклорена.

  \begin{equation*}
    \begin{aligned}
      e^x
        = 1 + \frac{x}{1!}
        + \dots
        + \frac{x^n}{n!}
        + \smallo(x^n)
    \\
      \ln(1 + x)
        = x - \frac{x^2}{2} + \frac{x^3}{3}
        - \dotsc
        + (-1)^{n + 1} \cdot \frac{x^n}{n}
        + \smallo(x^n)
    \\
      \sin x
        = x - \frac{x^3}{3!} + \frac{x^5}{5!}
        - \dotsc
        + (-1)^{n - 1} \cdot \frac{x^{2n - 1}}{(2n - 1)!}
        + \smallo(x^{2n})
    \\
      \cos x
        = 1 - \frac{x^2}{2!} + \frac{x^4}{4!}
        - \dotsc
        + (-1)^n \cdot \frac{x^{2n}}{(2n)!}
        + \smallo(x^{2n + 1})
    \end{aligned}
  \end{equation*}
\end{remark}
