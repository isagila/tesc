\subsection{%
  Правила дифференцирования: производная сложной функции, инвариантность
  дифференциала.%
}

\begin{theorem}[Производная сложной функции]
  \begin{equation*}
    \begin{rcases}
      g(x) \isdiffd{x} \\
      f(g) \isdiffd{g_0 = g(x)}
    \end{rcases}
    \implies
    f'(g(x)) = f'(g) \cdot g'(x)
  \end{equation*}
\end{theorem}

\begin{proof}
  По критерию дифференцируемости имеем

  \begin{equation*} \label{eq:diff-cpx-func-1} \tag{1}
    \begin{aligned}
      u = g(x) \isdiffd{x}
      \implies
      \Delta u = g'(x) \Delta x + \smallo(\Delta x)
    \\
      y = f(g) \isdiffd{g_0}
      \implies
      \Delta y = f'(u) \Delta u + \smallo(\Delta u)
    \\
      \Delta y = f'(u)(g'(x) \Delta x + \smallo(\Delta x)) + \smallo(\Delta u)
    \end{aligned}
  \end{equation*}
  
  Также заметим, что 

  \begin{equation*} \label{eq:diff-cpx-func-2} \tag{2}
    u = g(x) \isdiffd{x} \implies g(x) \iscontd{x}
    \iff
    \Delta x \to 0 \implies \Delta u \to 0
  \end{equation*}

  Напишем определение производной для исходной функции.

  \begin{equation*} \label{eq:diff-cpx-func-3} \tag{3}
    \prh{f(g(x))}'
    = \lim_{\Delta x \to 0} \frac{\Delta y}{\Delta x}
    = \lim_{\Delta x \to 0} f'(u) g'(x)
      + f'(u) \frac{\smallo(\Delta x)}{\Delta x}
      + \frac{\smallo(\Delta u)}{\Delta x}
  \end{equation*}

  Второе и третье слагаемые в дают ноль в пределе, как отношение б.м. функции
  более высокого порядка и её аргумента (см. \eqref{eq:diff-cpx-func-2}), значит
  \(\prh{f(g(x)}' = f'(g) g'(x)\).
\end{proof}

\begin{remark}
  Если необходимо взять производную от композиции нескольких функций, то можно
  пользоваться \quote{расширенной} формулой производной композиции функций
  (правилом цепочки).

  \begin{equation*}
    \prh{f(g(h(\phi(x))))}' = f'(g) \cdot g'(h) \cdot h'(\phi) \cdot \phi'(x)
  \end{equation*}
\end{remark}

\begin{theorem}
  Первый дифференциал сохраняет инвариантность формы.
\end{theorem}

\begin{proof}
  Рассмотрим \(y = f(x)\), пусть \(x = u(t)\). Используя определение
  дифференциала и производную сложной функции получаем

  \begin{equation*}
    \dd y
    = y'_t \dd t
    = y'_x \cdot \under{x'_t \dd t}{\dd x}
    = y'_x \dd x
  \end{equation*}
  
  Таким образом, форма дифференциала не зависит от того, является аргумент
  функции независимой переменной или функцией другого аргумента.
\end{proof}

\begin{theorem}[Производная обратной функции]
  \begin{equation*}
    \begin{rcases}
      f(x) \isdiffd{x_0} \\
      f'(x_0) \neq 0 \\
      \exists g(y) = f^{-1}(y)
    \end{rcases}
    \implies
    g'(y) = \frac{1}{f'(x)}
  \end{equation*}
\end{theorem}

\begin{proof}
  Раскроем производную по определению и преобразуем.

  \begin{equation*} \label{eq:diff-inv-func-1} \tag{1}
    f'(x) = \lim_{\Delta x \to 0} \frac{\Delta y}{\Delta x}
    = \frac{1}{\lim_{\Delta x \to 0} \frac{\Delta x}{\Delta y}}
  \end{equation*}

  Заметим, что

  \begin{equation*} \label{eq:diff-inv-func-2} \tag{2}
    \begin{aligned}
      f(x) \isdiffd{x_0}
      \implies f(x) \iscontd{x_0}
      \iff \Delta x \to 0 \implies \Delta y \to 0
    \\
      \frac{1}{\lim_{\Delta x \to 0} \frac{\Delta x}{\Delta y}}
      = \frac{1}{\lim_{\Delta y \to 0} \frac{\Delta x}{\Delta y}}
    \end{aligned}
  \end{equation*}

  Выражение в знаменателе это и есть производная обратной функции. Таким образом
  из \eqref{eq:diff-inv-func-1} и \eqref{eq:diff-inv-func-2} следует, что

  \begin{equation*}
    f'(x)
    = \frac{1}{\lim_{\Delta y \to 0} \frac{\Delta x}{\Delta y}}
    = \frac{1}{g'(y)}
    \implies \prh{f^{-1}(y)}' = \frac{1}{f'(x)}
  \end{equation*}
\end{proof}

\begin{theorem}[Производная параметрически заданной функции]
  \begin{equation*}
    \begin{rcases}
      x = a(t) \isdiffd{t_0} \\
      y = b(t) \isdiffd{t_0} \\
      \exists a^{-1}(t) \\
      a'(t_0) \neq 0
    \end{rcases}
    \implies
    y'(x) = \frac{b'(t_0)}{a'(t_0)}
  \end{equation*}
\end{theorem}

\begin{proof}
  Т.к. \(a(t)\) обратима, то \(t = A(x)\), тогда по формуле производной сложной
  функции получаем

  \begin{equation*} \label{eq:diff-param-func-1} \tag{1}
    \prh{b(A(x))}'_x = b'_A \cdot A'_x
  \end{equation*}
  
  По формуле производной обратной функции \(\display{A'_x = \frac{1}{x'_A}}\).
  Учитывая, что \(A(x) = t\) и \(b(t) = y\), подставим это в
  \eqref{eq:diff-param-func-1}.

  \begin{equation*} \label{eq:diff-param-func-2} \tag{2}
    b(t)'_x = b'_t \cdot \frac{1}{x'_t} 
    \implies
    y'(x) = \frac{y'_t}{x'_t}
  \end{equation*}
\end{proof}