\subsection{%
  Геометрический вектор в координатном пространстве. Определение,
  характеристики.%
}

\begin{remark}
  Точку (как упорядоченную пару) нельзя полноценно рассматривать как вектор,
  т.к. для точек на плоскости не определено сложение и умножение на число.
\end{remark}

Каждому геометрическому вектору в координатном пространстве поставим в
соответствие упорядоченный набор чисел (которые будут называться его
координатами в выбранном базисе) так, чтобы данный вектор являлся линейной
комбинацией базисных векторов с этим набором коэффициентов.

Выберем \quote{удобный} ортонормированный базис в ДСК: векторы \(\vec{i}\) и
\(\vec{j}\) единичной длины в направлении осей (такие векторы называются
декартовыми ортами). Тогда 

\begin{equation*}
  \vec{a} = x \vec{i} + y \vec{j}
  \qquad
  \vec{a} \Rarr{} (x, y)
\end{equation*}

Скалярное произведение определим как обычно (\ref{rem:dot-pdf}):

\begin{equation*}
  \dotpdtv{a}{b}
  = a_x b_x + a_y b_y
  = \abs{\vec{a}} \cdot \abs{\vec{b}} \cdot \cos \angle (\vec{a}, \vec{b})
\end{equation*}

\begin{definition} \label{def:vec-len}
  Длиной (нормой) вектора будем называть

  \begin{equation*}
    \norm{\vec{a}}
    = \abs{\vec{a}}
    = \sqrt{a^2}
    = \sqrt{a_x^2 + a_y^2}
  \end{equation*}
\end{definition}

\begin{definition}
  Направлением будем называть углы, которые вектор образует с осями
  (\figref{01_22_01}), а косинусы этих углов~--- направляющими.
\end{definition}

\galleryone{01_22_01}{Направление вектора}

\begin{theorem}
  Сумма квадратов направляющих равна единице.
\end{theorem}

\begin{proof}
  Выразим косинусы через проекции вектора \(\vec{a}\) на оси.

  \begin{equation*}
    \begin{cases}
      a_x = \text{пр}_{Ox} \vec{a} = \abs{\vec{a}} \cos \alpha
      \implies
      \cos \alpha = \dfrac{a_x}{\abs{\vec{a}}}
    \\
      a_y = \text{пр}_{Oy} \vec{a} = \abs{\vec{a}} \cos \beta
      \implies
      \cos \beta = \dfrac{a_y}{\abs{\vec{a}}}
    \\
      a_z = \text{пр}_{Oz} \vec{a} = \abs{\vec{a}} \cos \gamma
      \implies
      \cos \gamma = \dfrac{a_z}{\abs{\vec{a}}}
    \end{cases}
  \end{equation*}

  Сложим квадраты полученных косинусов и раскроем длину вектору по определению
  \ref{def:vec-len}.

  \begin{equation*}
    \cos^2 \alpha + \cos^2 \beta + \cos^2 \gamma
    = \frac{a_x^2}{a_x^2 + a_y^2 + a_z^2}
    + \frac{a_y^2}{a_x^2 + a_y^2 + a_z^2}
    + \frac{a_z^2}{a_x^2 + a_y^2 + a_z^2}
    = 1
  \end{equation*}
\end{proof}
 
\begin{remark}
  Рассмотрим вектор \(\vec{e} = (\cos \alpha, \cos \beta)\). Его длина равна
  \(1\) (по основному тригонометрическому тождеству, т.к. углы \(\alpha\) и
  \(\beta\) смежные), при этом

  \begin{equation*}
    \vec{a} = \abs{\vec{a}}(\cos \alpha, \cos \beta)
    \qquad
    \vec{e} = \frac{\vec{a}}{\abs{\vec{a}}}
  \end{equation*}
\end{remark}

\begin{definition}
  Вектор \(\vec{e} = (\cos \alpha, \cos \beta)\) называется единичным в
  направлении вектора \(\vec{a}\), а операция приведения вектора к единичному в
  том же направлении называется нормированием.
\end{definition}
