\subsection{%
  Преобразование базиса и координат.%
}

\subheader{Переход к новому базису}

Пусть есть старый базис \(\set{\basis_i}_{i = 1}^n\) и новый базис
\(\set{\basis’_i}_{i = 1}^n\) и необходимо перейти от старого базиса к новому.
Разложим каждый из векторов старого базиса по новому базису.

\begin{equation*}
  \begin{cases}
    \basis_1 & = a_{1, 1} \basis'_1 + \dotsc + a_{1, n} \basis'_n \\
    \vdots & \\
    \basis_n & = a_{n, 1} \basis'_1 + \dotsc +  a_{n, n} \basis'_n 
  \end{cases}  
\end{equation*}

Теперь рассмотрим разложение произвольного вектора \(x\) в старом базисе.
Заменим в этом разложении каждый из векторов старого базиса на его разложение по
новому базису, перегруппируем и получим

\begin{equation*}
  \begin{aligned}
    x & = b_1 \basis_1 + \dotsc + b_n \basis_n  
  \\
    x & = b_1 (a_{1, 1} \basis'_1 + \dotsc + a_{1, n} \basis'_n)
      + \dotsc
      + b_n (a_{n, 1} \basis'_1 + \dotsc + a_{n, n} \basis'_n)
  \\
    x & = (b_1 a_{1, 1} + \dotsс + b_n a_{n, 1}) \basis'_1
    + \dotsc
    + (b_1 a_{1, n} + \dotsc + b_n a_{n, n}) \basis'_n
  \end{aligned}
\end{equation*}

\begin{remark}
  В матричном виде переход к новому базису можно выразить так

  \begin{equation*}
    B' = A^{-1 }B
  \end{equation*}

  А новые координаты можно вычислить так

  \begin{equation*}
    X_{B'} = A^T X_B
  \end{equation*}
\end{remark}

\subheader{Преобразования системы координат (СК)}

В качестве преобразований СК будет рассматривать только движения, причем будем
рассматривать движения именно СК, а не фигуры.

\begin{definition}
  Движение это отображение пространства в себя, которые сохраняет расстояние
  между точками.

  \begin{equation*}
    \begin{aligned}
      M(x, y) & \rarr{f} M'(x', y')
    \\
      O & \rarr{f} O'
    \\
      \abs{\vec{OM}} & = \abs{\vec{O'M'}}
    \end{aligned}
  \end{equation*}
\end{definition}

Существует несколько типов движений.

\begin{enumerate}
\item
  Осевая симметрия.
\item
  Параллельный перенос (сдвиг).
\item
  Поворот относительно точки.
\end{enumerate}

\subheader{Осевая симметрия}

\begin{definition}
  Осевой симметрией называется симметрия относительно одной (или нескольких) из
  осей.
\end{definition}

Координаты при этом меняются следующим образом.

\begin{equation*}
  \begin{array}{cc}
    \text{Симметрия относительно \(OY\)}
    &
    \text{Симметрия относительно \(OX\)}
  \\
    \begin{cases}
      x' = -x \\
      y' = y
    \end{cases}
    &
    \begin{cases}
      x' = x \\
      y' = -y
    \end{cases}
  \end{array}
\end{equation*}

\subheader{Параллельный перенос}

При параллельном переносе СК на вектор \(\vec{OO'}\) каждая точка переносится на
этот вектор. Перенос обозначается \(P_{\vec{a}}\), где \(\vec{a}\) это вектор
переноса.  

Координаты при этом меняются следующим образом:

\begin{equation*}
  \begin{cases}
    x = x' + x_0 \\
    y = y' + y_0
  \end{cases}
  \qquad
  \begin{cases}
    x' = x - x_0 \\
    y' = y - y_0
  \end{cases} 
\end{equation*}

где \((x_0, y_0)\) координаты точки \(O'\) в старой СК.

\begin{remark}
  При параллельном переносе направления осей сохраняются, т.е.
  \(OX \collinear O'X'\) и \(OY \collinear  O'Y'\).
\end{remark}

\subheader{Поворот СК}

Рассмотрим поворот СК \textbf{вокруг начала координат} на угол \(\alpha\)
\textbf{против часовой стрелки}.

Координаты при этом меняются следующим образом:

\begin{equation*} \label{eq:CS-rotate} \tag{ROT}
  \begin{cases}
    x = x' \cos \alpha - y' \sin \alpha \\
    y = x' \sin \alpha + y' \cos \alpha
  \end{cases}
  \qquad
  \begin{cases}
    x' = x \cos \alpha + y \sin \alpha \\
    y' = -x \sin \alpha + y \cos \alpha
  \end{cases}
\end{equation*}

\begin{remark}
  Матрица \(\display{\mtxp{
    \cos \alpha & -\sin \alpha \\
    \sin \alpha & \cos \alpha
  }}\)  называется матрицей поворота. Если умножить старые координаты на неё, то
  получатся новые координаты. Матрица обратного преобразования является обратной
  матрицей к матрице поворота. 
\end{remark}

\galleryone{01_19_01}{Смена координат при повороте СК}
 
Покажем, как можно получить формулы \eqref{eq:CS-rotate}.

\begin{proof}
  При повороте центр СК остается прежним и угол между осями не изменяется,
  значит \(O' = O\) и \(\angle(OX, OY) = \angle(OX', OY')\). Пусть
  \(\abs{\vec{a}} = \rho\), введём полярную СК (\(O\) полюс, \(OX\) полярная
  ось). Обозначим \((x, y)\) координаты вектора \(\vec{a}\) в старой СК, а
  \((x', y')\)~--- в новой. Из \figref{01_19_01} получаем

  \begin{align*}
    \begin{cases}
      x' = OB = \rho \cos (\omega - \alpha) \\
      y' = OA = \rho \sin(\omega - \alpha)
    \end{cases}
    \label{eq:CS-rotate-1} \tag{1}
  \\
    \begin{cases}
      x = OC = \rho \cos \omega \\
      y = OD = \rho \sin \omega
    \end{cases}
    \label{eq:CS-rotate-2} \tag{2}
  \end{align*}

  В \eqref{eq:CS-rotate-1} применим тригонометрические формулы.

  \begin{equation*} \label{eq:CS-rotate-3} \tag{3}
    \begin{cases}
      x' = \under{\rho \cos \omega}{x} \cos \alpha
        + \under{\rho \sin \omega}{y} \sin \alpha
      \\
      y' = \under{\rho \sin \omega}{y} \cos \alpha
        - \under{\rho \cos \omega}{x} \sin \alpha
    \end{cases}
  \end{equation*}

  Подставим \eqref{eq:CS-rotate-2} в \eqref{eq:CS-rotate-3}.

  \begin{equation*}
    \begin{cases}
      x' = x \cos \alpha + y \sin \alpha \\
      y' = y \cos \alpha - x \sin \alpha
    \end{cases}
    \implies
    \begin{cases}
      x' = x \cos\alpha + y \sin \alpha \\
      y' = -x \sin \alpha + y \cos \alpha
    \end{cases}
  \end{equation*}
\end{proof}
