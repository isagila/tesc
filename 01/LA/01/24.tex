\subsection{%
  Коллинеарность, компланарность, ортогональность векторов. Критерии.%
}

\begin{definition}
  Два вектора называются коллинеарными, если они лежат на одной прямой или на
  параллельных прямых.
\end{definition}

\begin{theorem}[Критерий коллинеарности]
  \begin{equation*}
    \vec{a} \collinear \vec{b}
    \iff \vecpdtv{a}{b} = 0
    \iff \vec{a} = k \vec{b} \; (k \in \RR)
  \end{equation*}
\end{theorem}

\begin{definition}
  Три вектора называются компланарными, если они лежат в одной плоскости.
\end{definition}

\begin{theorem}[Критерий компланарности]
  Три вектора компланарны \(\iff\) их смешанное произведение равно нулю.
\end{theorem}

\begin{proof}
  Если три вектора компланарны, то они лежат в одной плоскости, значит два из
  них можно использовать как базис этой плоскости (случаи, когда это нельзя
  сделать  можно рассмотреть отдельно~--- они тривиальны). В таком случае третий
  вектор будет их линейной комбинацией, т.к. в определителе одна из строчек
  будет линейной комбинацией двух других. В таком случае по свойству
  определитель (а значит и смешанное произведение) будет равен нулю.
\end{proof}

\begin{definition}
  Два \textbf{ненулевых} вектора называются ортогональными, если их скалярное
  произведение равно нулю.
  
  \begin{equation*}
    \vec{a} \perp \vec{b} \iff \dotpdtv{a}{b} = 0
  \end{equation*}
\end{definition}
