\subsection{%
  Пространство решений однородной СЛАУ. Фундаментальная система решений.%
}

\begin{remark}
  Пространство решений СЛАУ это линейное подпространство \(X\) исходного
  линейного пространства \(L\) системы, причем \(\dim X < \dim L\) (в общем
  случае).
\end{remark}

В пространстве решений СЛАУ можно выделить базис, который будет называться
фундаментальной системой решений СЛАУ. Размерность этого базиса будет \(n -
\rank A\), где \(n\) количество переменных, а \(A\)~--- исследуемая СЛАУ.

\begin{definition}
  Система \(\set{X_i}_{i = 1}^n\) называется фундаментальной системой решений
  (\textit{ФСР}) СЛАУ, если
  
  \begin{enumerate}
  \item
    Она линейно независима.
    
  \item
    Добавление любого решения СЛАУ в эту систему делает её линейно зависимой.
    
  \item
    Любое решение СЛАУ представимо в виде линейной комбинации векторов этой
    системы.
  \end{enumerate}
\end{definition}
