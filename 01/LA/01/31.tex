\subsection{%
  Классификация кривых второго порядка.%
}

\begin{definition}
  Задачей классификации кривых второго порядка называется задача сведения
  уравнение к простейшему (каноническому) виду при помощи переноса и поворота
  СК.
\end{definition}

\begin{definition}
  Группой старших членов в общем уравнении кривой второго порядка называется
  \(a_{11} x^2 + 2 a_{12} xy + a_{22} y^2\).
\end{definition}

\begin{definition}
  Группой линейных членов в общем уравнении кривой второго порядка называется
  \(2 a_{13} x + 2 a_{23} y + a_{33}\).
\end{definition}

\subheader{Перенос СК \(Oxy \to O'x'y'\)}

При переносе координаты меняются по следующим правилам

\begin{equation*}
  \begin{cases}
    x = x' + x_0 \\
    y = y' + y_0
  \end{cases}
\end{equation*}

Подставим это в общее уравнение и получим новое уравнение

\begin{equation*}
  \begin{cases}
    a'_{11} = a_{11} \\
    a'_{12} = a_{12} \\
    a'_{22} = a_{22} \\
    a'_{13} = a_{13} + a_{11} x_0 + a_{12} y_0 \\
    a'_{23} = a_{23} + a_{12} x_0 + a_{22} y_0 \\
    a'_{33} = a_{33} + a_{11} x_0^2 + 2 a_{12} x_0 y_0
      + a_{22} y_0^2 + 2 a_{13} x_0 + 2 a_{23} y_0
  \end{cases}
\end{equation*}

\subheader{Поворот СК \(Oxy \to Ox'y'\)}

При повороте координаты меняются по следующим правилам

\begin{equation*}
  \begin{cases}
    x = x' \cos \alpha - y' \sin \alpha \\
    y = x' \sin \alpha + y' \cos \alpha
  \end{cases}  
\end{equation*}

Подставим это в общее уравнение и получим новое уравнение

\begin{equation*}
  \begin{cases}
    a'_{11} = a_{11} \cos^2 \phi + 2 a_{12} \sin \phi \cos\phi
      + a_{22} \sin^2 \phi
  \\
    a'_{12} = \sin \phi \cos \phi (a_{22} - a_{11})
      + a_{12} (\cos^2 \phi - \sin^2 \phi)
  \\
    a'_{22} = a_{11} \sin^2 \phi - 2 a_{12} \sin \phi \cos\phi
      + a_{22} \cos^2 \phi
  \\
    a'_{13} = a_{13} \cos \phi + a_{23} \sin \phi
  \\
    a'_{23} = a_{23} \cos \phi - a_{13} \sin \phi
  \\
    a'_{33} = a_{33}
  \end{cases}
\end{equation*}

\begin{definition}
  Будем называть инвариантом функцию \(I(a_{11}, a_{12}, a_{22}, a_{13}, a_{23},
  a_{33})\), которая не меняет своего значения при переносе и повороте СК.  
\end{definition}

Выделим три инварианта.

\begin{equation*}
  I_1 = a_{11} + a_{22}
  \qquad
  I_2 = \mtxv{
    a_{11} & a_{12} \\
    a_{12} & a_{22}
  }
  \qquad
  I_3 = \mtxv{
    a_{11} & a_{12} & a_{13} \\
    a_{12} & a_{22} & a_{23} \\
    a_{13} & a_{23} & a_{33}
  }
\end{equation*}

Для того, чтобы доказать \(I_1\), \(I_2\) и \(I_3\) действительно инварианты
необходимо выполнить перенос и поворот и показать, что инвариант сохранил свое
значение (это несложное, но крайне рутинное действие, поэтому мы его опустим).

\begin{definition}
  В зависимости от знака \(I_2\) линии делятся на
  
  \begin{enumerate}
  \item
    \(I_2 < 0 \implies\) линия гиперболического типа.
  
  \item
    \(I_2 = 0 \implies\) линия параболического типа.
  
  \item
    \(I_2 > 0 \implies\) линия эллиптического типа.
  \end{enumerate}
\end{definition}

\begin{definition}
  Кривая второго порядка называется центральной (центрально симметричной), если
  при осевой симметрии \((x, y) \to (-x, -y)\) уравнение не меняется. Таким
  образом общее уравнение центральной кривой имеет вид

  \begin{equation*}
    a_{11} x^2 + 2 a_{12} xy + a_{22} y^2 + a_{33} = 0
  \end{equation*}
\end{definition}

\begin{theorem}
  У центральной кривой второй инвариант не равен нулю.
\end{theorem}

\begin{proof}
  Выполним перенос СК, обратим внимание на коэффициенты \(a'_{13}\) и
  \(a'_{23}\). Для того, чтобы линия была центральной, необходимо обнулить их,
  таким образом мы получаем СЛАУ относительно \(x_0\) и \(y_0\):

  \begin{equation*}
    \begin{cases}
      a_{11} x_0 + a_{12} y_0 = - a_{13} \\
      a_{12} x_0 + a_{22} y_0 = -a_{23}
    \end{cases}
  \end{equation*}

  Если определитель основной матрицы не равен нулю, то это СЛАУ крамеровского
  типа, и она имеет единственное решение (что нам и требуется). Заметим, что
  определитель основной матрицы это второй инвариант. Таким образом, если \(I_2
  \ne 0\), то система имеет единственное решение, значит мы можем обнулить
  коэффициенты \(a_{13}\) и \(a_{23}\), а значит мы можем привести уравнение
  кривой второго порядка к общему уравнению центральной кривой.
\end{proof}

\begin{remark}
  Согласно данным ранее определениям кривые эллиптического и гиперболического
  типа всегда центральные, а кривые параболического типа - нет (но они могут
  иметь ось симметрии).
\end{remark}

\subheader{Приведение к главным осям (обнуление \(a_{12}\))}

Выполним поворот СК, т.к. мы хотим обнулить \(a'_{12}\), то получим уравнение

\begin{equation*}
  \begin{aligned}
    \sin \phi \cos \phi (a_{22} - a_{11})
    + a_{12} (\cos^2 \phi - \sin^2 \phi) = 0
  \\
    \sin (2 \phi) (a_{22} - a_{11}) + 2 a_{12} \cos (2 \phi) = 0
  \\
    \tg (2 \phi) = \frac{2 a_{12}}{a_{11} - a_{22}}
  \end{aligned}
\end{equation*}

Получаем формулу для нахождения угла поворота для приведения к главным осям.
Поворачиваем СК на наименьший положительный угол \(\phi\), удовлетворяющий
этому уравнению. Если \(a_{11} = a_{22}\), то берем \(\phi = \dfrac{\pi}{4}\).

\subheader{Классификация прямых}

Вычисляем три инварианта, далее рассматриваем случаи.

\subsubheader{Случай I.}{Центральная кривая (\(I_2 \neq 0\)).}

\begin{enumerate}
\item
  Переносом обнулим \(a_{13}\) и \(a_{23}\).

\item
  Поворотом обнулим \(a_{12}\).

\item
  Получим уравнение \(a''_{11} (x'')^2 + a''_{22} (y'')^2 + a''_{33} = 0\).

\item
  Выразим третий инвариант после переноса (\(a_{13} = a_{23} = 0\)).

  \begin{equation*}
    I_3
    = \mtxv{
      a_{11} & a_{12} & 0 \\
      a_{12} & a_{22} & 0 \\
      0 & 0 & a_{33}
    }
    = a_{33} \cdot \mtxv{
      a_{11} & a_{12} \\
      a_{12} & a_{22}
    }
    = a_{33} \cdot I_2
  \end{equation*}

\item
  Значит \(a_{33} = \dfrac{I_3}{I_2}\), подставим в уравнение из пункта 3.
  Получим уравнение, которое далее будем исследовать по случаям.

  \begin{equation*}
    a''_{11} (x'')^2 + a''_{22} (y'')^2 = -\dfrac{I_3}{I_2}
  \end{equation*}
\end{enumerate}

\subsubheader{Случай I.а}{Гиперболическая кривая \(I_2 < 0\)}
    
Т.к. \(I_2 = a''_{11} a''_{22} < 0\), то \(a''_{11}\) и \(a''_{22}\) разных
знаков.

\begin{twocolumns}
  Если \(a''_{11} > 0\) и \(a''_{22} < 0\), тогда

  \begin{equation*}
    \begin{cases}
      I_3 < 0 \implies \text{сопряженная гипербола}     \\
      I_3 = 0 \implies \text{пара пересекающихся прямых} \\
      I_3 > 0 \implies \text{гипербола}
    \end{cases}
  \end{equation*}
  \columnbreak

  Если \(a''_{11} < 0\) и \(a''_{22} > 0\), тогда

  \begin{equation*}
    \begin{cases}
      I_3 < 0 \implies \text{гипербола}                   \\
      I_3 = 0 \implies \text{пара пересекающихся  прямых} \\
      I_3 > 0 \implies \text{сопряженная гипербола}
    \end{cases}
  \end{equation*}
\end{twocolumns}
  
Пара пересекающихся прямых получается из разложения разности квадратов, например
в первом случае.

\begin{equation*}
  (\sqrt{a''_{11}} x'' + \sqrt{-a''_{22}} y'') \cdot
    (\sqrt{a''_{11}} x'' - \sqrt{-a''_{22}} y'') = 0
\end{equation*}
    
Причем запись \(\sqrt{-a''_{22}}\) корректна, т.к. \(a''_{22} < 0\).

\subsubheader{Случай I.b}{Эллиптическая кривая \(I_2 > 0\)}

Т.к. \(I_2 = a''_{11} a''_{22}  > 0\), то \(a''_{11}$ и $a''_{22}\) одного
знака.

\begin{twocolumns}
  Если \(a''_{11} > 0\) и \(a''_{22} > 0\), тогда

  \begin{equation*}
    \begin{cases}
      I_3 < 0 \implies \text{эллипс}        \\
      I_3 = 0 \implies \text{точка } (0, 0) \\
      I_3 > 0 \implies \varnothing
    \end{cases}
  \end{equation*}
  \columnbreak

  Если \(a''_{11} < 0\) и \(a''_{22} < 0\), тогда

  \begin{equation*}
    \begin{cases}
      I_3 < 0 \implies \varnothing          \\
      I_3 = 0 \implies \text{точка } (0, 0) \\
      I_3 > 0 \implies \text{эллипс}
    \end{cases}
  \end{equation*}
\end{twocolumns}

\subsubheader{Случай II.}{Нецентральная кривая (\(I_2 = 0\))}

\begin{enumerate}
\item
  Поворотом обнулим \(a_{12}\).

\item
  Т.к. \(I_2 = a'_{11} a'_{22} = 0\), то либо \(a'_{11} = 0\) и \(a'_{22} \ne
  0\), либо наоборот. При этом коэффициенты \(a_{11}\) и \(a_{12}\) не могут
  быть равны нулю одновременно, т.к. в таком случае группа старших членов
  обнуляется, и мы получаем не кривую второго порядка (а прямую в общем случае).

\item
  Рассмотрим случай \(a'_{11} = 0\) и \(a'_{22} \ne 0\) (обратный случай
  рассматривается аналогично). Исследуем следующее уравнение

  \begin{equation*}
    a'_{22} (y')^2 + 2 a'_{13} x' + a'_{23} y' + a'_{33} = 0
  \end{equation*}
\end{enumerate}
        
\subsubheader{Случай II.a}{\(I_3 \neq 0\)}

\begin{equation*}
  \mtxv{
    0       & 0       & a'_{13} \\
    0       & a'_{22} & a'_{23} \\
    a'_{13} & a'_{23} & a'_{33}
  }
  = -(a'_{13})^2 \cdot a'_{22} \ne 0
\end{equation*}

Т.к. мы рассматриваем случай \(a'_{22} \neq 0\), то \(a'_{13} \neq 0\). Получаем
уравнение

\begin{equation*}
  a'_{22} (y')^2 + 2 a'_{23} y' + 2 a'_{13}x + a'_{33} = 0
\end{equation*}
        
С помощью выделения полного квадрата и переноса СК можно убедиться, что данное
уравнение задает параболу.
        
\subsubheader{Случай II.b}{\(I_3 = 0\)}

\begin{equation*}
  \mtxv{
    0       & 0       & a'_{13} \\
    0       & a'_{22} & a'_{23} \\
    a'_{13} & a'_{23} & a'_{33}
  }
  = -(a'_{13})^2 \cdot a'_{22} = 0  
\end{equation*}

Т.к. мы рассматриваем случай \(a'_{22} \neq 0\), то \(a'_{13} = 0\). Получаем
уравнение

\begin{equation*}
  \begin{aligned}
    a'_{22} (y')^2 + 2 a'_{23} y' + a'_{33} = 0
  \\  
    a'_{22} \prh{y' + \frac{a'_{23}}{a'_{22}}}^2
      - \frac{(a'_{23})^2}{a_{22}} + a'_{33} = 0
  \\
    \prh{y' + \frac{a'_{23}}{a'_{22}}}^2
    = \prh{\frac{a'_{23}}{a_{22}}}^2 - \frac{a'_{33}}{a'_{22}}
  \end{aligned}
\end{equation*}
          
Таким образом в зависимости от знака выражения в правой части это уравнение
будет задавать

\begin{enumerate}
\item
  Правая часть \(> 0 \implies\) пара параллельных прямых.

\item
  Правая часть \(= 0 \implies\) одна прямая.

\item
  Правая часть \(< 0 \implies \varnothing\).
\end{enumerate}

\begin{remark}
  Общее уравнение кривой второго порядка описывает только одно из множеств

  \begin{enumerate}  
  \item
    Эллипс.
    
  \item
    Гипербола.
    
  \item
    Парабола.
    
  \item
    Пара пересекающихся прямых.
    
  \item
    Пара параллельных прямых.
    
  \item
    Одна прямая.
    
  \item
    Точка.
    
  \item
    Пустое множество.
  \end{enumerate}

  Причем вырожденные случаи (4--8) возникают только при \(I_3 = 0\).
  О случаях 4 и 5 говорят, что кривая \quote{распалась} на прямые.
\end{remark}
