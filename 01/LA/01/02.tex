\subsection{%
  Линейное пространство арифметических векторов. Определение, проверка аксиом.%
}

\begin{definition}
  Линейное (векторное) пространство это коммутативная аддитивная группа, в
  которой (\(\forall \lambda, \mu \in \CC\))
  
  \begin{enumerate}
  \item
    Определено умножение на число \(\lambda \in \CC\).
    
  \item
    Умножение коммутативно \(\lambda a = a \lambda\).
    
  \item
    Относительно умножения выполняется дистрибутивность \(\lambda (a + b) =
    \lambda a + \lambda b, (\lambda + \mu) a = \lambda a + \mu a\).
    
  \item
    Относительно умножения выполняется ассоциативность \((\lambda \mu) a =
    \lambda (\mu a)\).
    
  \item
    Относительно умножение есть нейтральный элемент \(\exists \theta \in \RR
    \given \forall a \in L \given \theta \cdot a = a\).
  \end{enumerate}
\end{definition}

\begin{definition}
  Арифметическим вектором называется упорядоченный набор из \(n\) чисел вида
  \((x_1, x_2, \dotsc, x_n)\).
\end{definition}

\begin{enumerate}
  \item
  Сложение определим как \((x_1, \dotsc, x_n) + (y_1, \dotsc, y_n) = (x_1 + y_1,
  \dotsc, x_n + y_n)\).
  
  \item
  Умножение на число определим как \(\lambda \cdot (x_1, \dotsc, x_n) = (\lambda
  x_1, \dotsc, \lambda x_n)\).
\end{enumerate}

\begin{remark}
  Множество векторов одной размерности образует линейное пространство.
\end{remark}

\begin{remark}
  Проверим аксиомы линейного пространства.
  
  \begin{enumerate}
  \item
    Замкнутость \((x_1, \dotsc, x_n) + (y_1, \dotsc, y_n) = (x_1 + y_1, \dotsc,
    x_n + y_n) \in L\).
    
  \item
    Ассоциативность \(((x_1, \dotsc, x_n) + (y_1, \dotsc, y_n)) + (z_1, \dotsc,
    z_n) = (x_1, \dotsc, x_n) + ((y_1, \dotsc, y_n) + (z_1, \dotsc, z_n))\)
    выполнена, т.к. сложение чисел ассоциативно.
    
  \item
    Наличие нейтрального элемента по сложению \((x_1, \dotsc, x_n) + (0, \dotsc,
    0) = (x_1, \dotsc, x_n)\). Нейтральный элемент по сложению это нулевой
    вектор длины \(n\).
    
  \item
    Наличие обратного элемента по сложению \((x_1, \dotsc, x_n) + (-x_1, \dotsc,
    -x_n) = (0, \dotsc, 0)\).
    
  \item
    Коммутативность выполнена, т.к. сложение и умножение чисел коммутативно.
    \(
      (x_1, \dotsc, x_n) + (y_1, \dotsc, y_n)
      = (y_1, \dotsc, y_n) + (x_1, \dotsc, x_n)
    \) и \(
      \lambda (x_1, \dotsc, x_n)
      = (x_1, \dotsc, x_n) \lambda
    \). 
    
  \item
    Дистрибутивность и ассоциативность для умножения на число выполнены, т.к.
    умножение чисел дистрибутивно относительно сложения.
    
    \begin{enumerate}
    \item
      \(\lambda \cdot ((x_1, \dotsc, x_n) + (y_1, \dotsc, y_n)) = \lambda \cdot
      (x_1, \dotsc, x_n) + \lambda \cdot (y_1, \dotsc, y_n)\)
      
    \item
      \((\lambda + \mu) \cdot (x_1, \dotsc, x_n) = \lambda \cdot (x_1, \dotsc,
      x_n) + \mu \cdot (x_1, \dotsc, x_n)\)
      
    \item
      \((\lambda \cdot \mu) \cdot (x_1, \dotsc, x_n) = \lambda \cdot (\mu \cdot
      (x_1, \dotsc, x_n))\)
    \end{enumerate}
    
  \item
    Наличие нейтрального элемента относительно умножения на число \((x_1,
    \dotsc, x_n) \cdot 1 = (x_1, \dotsc, x_n)\). Нейтральный элемент
    относительно умножения на число это \(1\).
  \end{enumerate}
\end{remark}
