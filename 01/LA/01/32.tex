\subsection{%
  Поверхность в пространстве. Кинетический способ задания поверхности.%
}

Способы конструирования поверхности.

\begin{enumerate}
\item
  Деформация/разрезание/склейка плоскости или её части.

\item
  Движения (кинетический способ).
\end{enumerate}

Свойства поверхности.

\begin{enumerate}
\item
  Ограничена/не ограничена.

\item
  Замкнута/незамкнута.

\item
  Двумерна (имеет два направления).

\item
  Двусторонняя/односторонняя.  
\end{enumerate}

\begin{remark}
  Замкнутая поверхность разбивает пространство на две части. У двусторонней
  поверхности чтобы попасть в ту же точку, но с другой стороны нужно пройти
  через край поверхности. Двусторонняя поверхность имеет две нормали в одной
  точке.

  Таким образом поверхность по сравнению с плоскостью сохраняет только
  двумерность, все остальные свойства не сохраняются.
\end{remark}

\galleryone{01_32_01}{Задание поверхности через образующую и направляющую}

Поверхность можно описать как множество точек кривой \(l\) при движении по
кривой \(d\) (\(l\) может деформироваться по ходу движения).

\begin{definition}
  Кривая \(l\) называется образующей, а кривая \(d\)~--- направляющей.
\end{definition}

\begin{remark}
  Если направляющая и образующая это кривые не выше 2ого порядка, то получится
поверхность не выше второго порядка.
\end{remark}
