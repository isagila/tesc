\subsection{%
  Линейная зависимость арифметических векторов. Линейная зависимость системы
  одного и двух векторов.%
}

\begin{definition}
  Линейной комбинацией системы векторов одной размерности \(\set{\alpha_i}_{i =
  1}^n\) называется \(\Lambda = \lambda_1 \alpha_1 + \dotsc + \lambda_n
  \alpha_n\), где \(\forall \lambda_i \in \RR\).
\end{definition}

\begin{definition}
  Линейная комбинация называется нулевой, если она равна нулю \(\lambda_1
  \alpha_1 + \dotsc + \lambda_n \alpha_n = 0\).
\end{definition}

\begin{definition}
  Нулевая линейная комбинация называется тривиальной, если \(\forall \lambda_i =
  0\).
\end{definition}

\begin{remark}
  Разложить вектор по системе означает представить его в виде линейной
  комбинации этой системы.
\end{remark}

\begin{definition} \label{def:li-dep-1}
  Система векторов одной размерности \(\set{\alpha_i}_{i = 1}^n\) называется
  линейно зависимой, если найдется её нулевая нетривиальная линейная комбинация
  \(\lambda_1 \alpha_1 + \dotsc + \lambda_n \alpha_n = 0\) (\(\exists \lambda_i
  \ne 0\)).
\end{definition}

Есть и другое определение линейно зависимой системы.

\begin{definition} \label{def:li-dep-2}
  Система векторов одной размерности \(\set{\alpha_i}_{i = 1}^n\) называется
  линейно зависимой, если один из векторов этой системы является линейной
  комбинацией других векторов этой системы.
\end{definition}

\begin{theorem}
  Определения \ref{def:li-dep-1} и \ref{def:li-dep-2} равносильны.
\end{theorem}

\begin{proof}
  По \ref{def:li-dep-1} линейной зависимости система векторов называется линейно
  зависимой, если \(\lambda_1 \alpha_1 + \dotsc + \lambda_n \alpha_n = 0\)
  (\(\exists \lambda_i \ne 0\)).
  
  Найдем \(\lambda_i \ne 0\) и разделим на него, затем перенесем все векторы
  (кроме того, у кого был коэффициент \(\lambda_i\)) в правую часть
  
  \begin{equation*}
    \alpha_i
    = -\frac{\lambda_1}{\lambda_i} \alpha_1
      \dotsc
      -\frac{\lambda_n}{\lambda_i} \alpha_n
  \end{equation*}
  
  Получается, что \(\alpha_i\) это линейная комбинация остальных векторов
  системы.
\end{proof}

\begin{definition}
  Система векторов одной размерности \(\set{\alpha_i}_{i = 1}^n\) называется
  линейно независимой, если её линейная комбинация равна нулю только в том
  случае, когда она тривиальна \(\lambda_1 \alpha_1 + \dotsc + \lambda_n
  \alpha_n = 0 \implies \forall \lambda_i = 0\).
\end{definition}

\begin{theorem}
  Система из одного вектора линейно зависима тогда и только тогда, когда этот
  вектор нулевой.
\end{theorem}

\begin{proof}
  Рассмотрим нулевую линейную комбинацию этой системы \(\lambda \vec{a} = 0\).
  
  \(\impliedby\) Если \(\vec{a} = 0\), то найдется бесконечно количество
  \(\lambda\), таких что это равенство верно \(\implies\) система линейно
  зависима.
  
  \(\implies\) Если система линейно зависима, то существует бесконечное
  количество \(\lambda\) таких, что равенство \(\lambda \vec{a} = 0\) верно.
  Значит вектор \(\vec{a}\) может быть равен только нулю.
\end{proof}

\begin{theorem}
  Система из двух векторов линейно зависима тогда и только тогда, когда эти
  векторы коллинеарны.
\end{theorem}

\begin{proof}
  Рассмотрим нулевую линейную комбинацию этой системы \(\lambda \vec{a} + \mu
  \vec{b} = 0\).
  
  \(\impliedby\) Если векторы коллинеарны, то \(\vec{a} = k \vec{b}\), подставим
  это в линейную комбинацию \((\lambda k + \mu) \vec{b} = 0\).
  
  Можно найти сколько угодно много \(\lambda\) и \(\mu\) таких, что \(\lambda k
  + \mu = 0\) Значит существует бесконечное число нетривиальных линейных
  комбинаций \(\implies\) система линейно зависима.
  
  \(\implies\) Если система линейно зависима, то либо \(\lambda \ne 0\), либо
  \(\mu \ne 0\) (либо и то, и другое). Пусть для определенности \(\lambda \ne
  0\) (для \(\mu \ne 0\) доказательство аналогично).
  
  Разделим равенство на \(\lambda\) и перенесём все, кроме вектора \(\vec{a}\),
  направо: \(\vec{a} = -\frac{\mu}{\lambda} \vec{b}\) Таким образом векторы
  коллинеарны.
\end{proof}
