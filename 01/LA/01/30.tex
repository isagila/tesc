\subsection{%
  Кривые второго порядка. Универсальные определения. Полярное уравнение. Общее
  уравнение.%
}

Используя определения эксцентриситета и директрисы (\ref{sec:01-29}),
сформулируем две теоремы.

\begin{theorem} \label{thr:conic-ecc}
  Для эллипса/гиперболы/параболы верно, что отношение расстояния от точки на
  кривой до фокуса к расстоянию от директрисы до той же точки кривой есть
  постоянная величина, равная эксцентриситету.

  \begin{equation*}
    \frac{r_i}{h_i} = \ecc
  \end{equation*}
\end{theorem}

\begin{proof}  
  Рассмотрим эллипс (\figref{01_30_01}). Т.к. \(M (x, y)\) принадлежит
  эллипсу, то

  \begin{equation*} \label{eq:conic-ecc-1} \tag{1}
    \frac{x^2}{a^2} + \frac{y^2}{b^2} = 1
    \implies
    y^2 = b^2 \cdot \prh{1 - \frac{x^2}{a^2}}
  \end{equation*}

  Из \figref{01_30_01} выразим \(r_1\), подставим в него
  \eqref{eq:conic-ecc-1} и упростим полученное выражение.

  \begin{equation*} \label{eq:conic-ecc-2} \tag{2}
    r_1
    = \sqrt{(x - c)^2 + y^2}
    = \sqrt{(x - c)^2 + b^2 \cdot \prh{1 - \frac{x^2}{a^2}}}
    = \sqrt{\frac{a^2 x^2 - 2 a^2 xc + a^2 c^2 + a^2 b^2 - b^2 x^2}{a^2}}
  \end{equation*}

  \galleryone{01_30_01}{Дополнение определения эксцентриситета}
  
  Т.к. мы рассматриваем эллипс, то
  
  \begin{equation*} \label{eq:conic-ecc-3} \tag{3}
    c^2 = a^2 - b^2
    \implies
    \begin{cases}
      a^2 x^2 - b^2 x^2 = c^2 x^2 \\
      a^2 c^2 + a^2 b^2 = a^4 - a^2 b^2 + a^2 b^2 = a^4
    \end{cases}
  \end{equation*}

  Подставим \eqref{eq:conic-ecc-3} в \eqref{eq:conic-ecc-2} и упростим.

  \begin{equation*} \label{eq:conic-ecc-4} \tag{4}
    r_1
    = \sqrt{\frac{a^4 - 2 a^2 xc + c^2 x^2}{a^2}}
    = \sqrt{\frac{(a^2 - x c)^2}{a^2}}
    = \abs{\frac{a^2 - x c}{a}}
    = \abs{a - x \cdot \frac{c}{a}}
  \end{equation*}
  
  Воспользуемся определением эксцентриситета \(\displaystyle{\frac{c}{a} =
  \ecc}\). Мы рассматриваем эллипс, значит \(\ecc \in [0, 1)\), из чего следует,
  что \(a \ge x \ecc\). Таким образом  можно раскрыть модуль с плюсом. Применяя
  все эти соображения к \eqref{eq:conic-ecc-4}, имеем

  \begin{equation*} \label{eq:conic-ecc-5} \tag{5}
    r_1
    = \abs{a - x \cdot \frac{c}{a}}
    = \abs{a - x \cdot \ecc}
    = a - x \cdot \ecc
  \end{equation*}
  
  Из \figref{01_30_01} выразим \(h_1\).

  \begin{equation*} \label{eq:conic-ecc-6} \tag{6}
    h_1
    = \frac{a}{\ecc} - x
    = \frac{a - x \ecc}{\ecc}
  \end{equation*}

  Из \eqref{eq:conic-ecc-5} и \eqref{eq:conic-ecc-6} следует, что
  \(\frac{r_1}{h_1} = \ecc\). Доказательство для \(r_2\) и \(h_2\) аналогично.
  Доказательство для гиперболы и параболы также аналогично.
\end{proof}

\begin{theorem}
  Отношение \(\display{\frac{r_i}{h_i} = \ecc' = const}\) определяет только
  эллипс/гиперболу/параболу.
\end{theorem}

Доказанные теоремы позволяют дать общее определение кривой второго порядка.

\begin{definition}
  Кривой второго порядка называется множество точек плоскости, расстояния от
  которых до данной точки и до данной прямой находятся в постоянном отношении.

  Данная точка называется фокусом, а данная прямая~--- директрисой. Постоянное
  отношение называется эксцентриситетом \(\ecc\), причем

  \begin{enumerate}
    \item
      \(\ecc < 1 \implies\) эллипс.

    \item
      \(\ecc = 1 \implies\) парабола.

    \item
      \(\ecc > 1 \implies\) гипербола.
  \end{enumerate}
\end{definition}

\begin{theorem}
  Уравнение кривой второго порядка в полярных координатах имеет вид
  
  \begin{equation*}
    \rho = \frac{k \ecc}{1 \pm \ecc \cdot \cos \phi}    
  \end{equation*}

  где \(k\) равно расстоянию от фокуса до директрисы, а \(\pm\) отвечает за то,
  какой из фокусов мы берём в качестве центра ПСК.
\end{theorem}

\begin{proof}
  Введём \quote{удобную} ПСК (\figref{01_30_02}). Обозначим \(k = \dist(F, d)\)
  расстояние от фокуса до директрисы, тогда \(r = \rho\) и \(h = k + \rho \cos
  \phi\). Используя определение эксцентриситета, выразим \(\rho\).

  \begin{equation*}
    \begin{aligned}
      \ecc = \frac{r}{h} \implies r = h \cdot \ecc
    \\
      \rho
      = (k + \rho \cos \phi) \cdot \ecc
      = \frac{k \ecc}{1 - \ecc \cdot \cos \phi}
    \end{aligned}
  \end{equation*}
  
  Однако это было проделано только для одного фокуса, для другого фокуса
  получится \(\display{\rho = \frac{k \ecc}{1 + \ecc \cdot \cos \phi}}\).
\end{proof}

\gallerydouble
  {01_30_02}{Кривые второго порядка в ПСК}
  {01_30_03}{Гипербола в ПСК}

\begin{remark}
  При использовании такого уравнения надо помнить, что в начале координат ПСК
  окажется один из фокусов кривой второго порядка, а не её центр как это было в
  ДПСК. Например, на \figref{01_30_03} у красной гиперболы левый фокус находится
  в \((0, 0)\), а у синей~--- правый.
\end{remark}

\begin{remark}
  Общее уравнение \textbf{линии} (т.к. могут быть вырожденные случаи) второго
  порядка имеет вид  

  \begin{equation*}
    a_{11} x^2 + 2 a_{12} xy + a_{22} y^2 + 2 a_{13} x + 2 a_{23} y + a_{33} = 0
  \end{equation*}
\end{remark}

