\subsection{%
  Метод Гаусса (приведение матрицы к ступенчатому виду). Вычисление ранга.%
}

\begin{enumerate}
\item
  Рассмотрим матрицу \(A\), изначально первый столбец будет текущим.
  
\item
  Найдем строчку, в которой в текущем столбце стоит ненулевое число.
  
\item
  Разделим данную строчку на это число и поменяем её со строчкой, номер которой
  равен номеру текущего столбца.
  
\item
  Вычтем эту строчку (умноженную на необходимые коэффициенты) из всех других
  строк так, чтобы обнулить все элементы текущего столбца, которые ниже главной
  диагонали.
  
\item
  Проделаем шаги 2-4 последовательно для всех столбцов слева направо.
  
\item
  Получим матрицу ступенчатого вида.
\end{enumerate}

\begin{example}
  \begin{equation*}
    \mtxp{
      2 & 4 & 6 \\
      1 & 4 & 7 \\
      3 & 2 & 5
    }
      \rarr{I \cdot 0.5}
    \mtxp{
      1 & 2 & 3 \\
      1 & 4 & 7 \\
      3 & 2 & 5
    }
      \rarr{II - I}
    \mtxp{
      1 & 2 & 3 \\
      0 & 2 & 4 \\
      3 & 2 & 5
    }
      \rarr{II \cdot 0.5}
    \mtxp{
      1 & 2 & 3 \\
      0 & 1 & 2 \\
      3 & 2 & 5
    }
      \rarr{III - 3 \cdot I}
    \mtxp{
      1 & 2 & 3 \\
      0 & 1 & 2 \\
      0 & -4 & -4
    }
      \rarr{III + 4 \cdot II}
    \mtxp{
      1 & 2 & 3 \\
      0 & 1 & 3 \\
      0 & 0 & 4
    }
  \end{equation*}
\end{example}

\begin{remark}
  Ранг матрицы равен количеству её ненулевых строк после приведения к
  ступенчатому виду.
\end{remark}

\begin{remark}
  Этот метод можно использовать для решения СЛАУ.
\end{remark}
