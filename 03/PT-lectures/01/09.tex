\subsection{%
  Лекция \texttt{23.10.31}.%
}

\subheader{Сингулярное распределение}

\begin{definition}
  Случайная величина \(\xi\) имеет сингулярное распределение, если существует
  борелевское множество \(B \in \mathcal{B}\) с нулевой мерой Лебега такое, что
  \(\prob{\xi \in B} = 1\), но \(\forall x \in B \given \prob{\xi = x} = 0\).
\end{definition}

Заметим, что данное множество \(B\)~--- несчетное, т.к. в противном случае по
свойству счетной аддитивности получили бы \(\prob{\xi \in B} = 0\). Таким
образом при сингулярном распределении случайная величина \(\xi\) распределена на
несчетном множестве лебеговой меры ноль. Т.к. \(\prob{\xi = x} = 0\), то функция
сингулярного распределения является непрерывной.

\begin{example}
  Функция распределения случайной величины \(\xi\)~--- лестница Кантора.

  \begin{equation*}
    F_{\xi} (x) = \begin{cases}
      0,                            & x \le 0 \\
      \frac{1}{2} F_{\xi} (3 x)     & 0 < x \le \frac{1}{3} \\
      \frac{1}{2}                   & \frac{1}{3} < x \le \frac{2}{3} \\
      \frac{1}{2} + \frac{1}{2} F_{\xi} (3 x - 2) & \frac{2}{3} < x < 1 \\
      1,                            & x \ge 1
    \end{cases}
  \end{equation*}
\end{example}

\begin{theorem}[Лебега]
  Пусть \(F_{\xi} (x)\)~--- функция распределения некоторой случайной величины
  \(\xi\), то ее можно представить в виде линейной комбинации

  \begin{equation*}
    F_{\xi} (x) = p_1 F_1 (x) + p_2 F_2 (x) + p_3 F_3 (x)
  \end{equation*}

  где \(p_1 + p_2 + p_3 = 1\) и \(F_1 (x)\), \(F_2 (x)\), \(F_3 (x)\)~--- это
  функции дискретного, абсолютно непрерывного и сингулярного распределения
  соответственно.
\end{theorem}

Таким образом существуют только дискретные, абсолютно непрерывные, сингулярные
распределения и их смеси.

\subheader{Преобразование случайных величин}

Пусть \(\xi\) это случайная величина на вероятностном пространстве
\(\tuple{\Omega, \euF, \probP}\). Рассмотрим функцию \(g(\xi) \colon \Omega \to
\RR\), где \(g(x) \colon \RR \to \RR\). Вопрос: когда \(g(\xi)\) также является
случайной величиной?

\begin{definition}
  Функция \(g(x)\) называется борелевской, если \(\forall B \in \mathcal{B}
  \given g^{-1} (B) \in \mathcal{B}\), т.е. прообраз любого борелевского
  множества является борелевским множеством.
\end{definition}

\begin{theorem}
  Если функция \(g(x)\)~--- борелевская, \(\xi\)~--- случайная величина на
  вероятностном пространстве \(\tuple{\Omega, \euF, \probP}\), то \(g(\xi)\)
  также будет случайной величиной на том же вероятностном пространстве
  \(\tuple{\Omega, \euF, \probP}\).
\end{theorem}

\begin{proof}
  \(\forall B \in \mathcal{B}\) прообраз \(g^{-1} (B) = \xi^{-1} (B') \in
  \euF \implies g(\xi)\) измеримая функция и является случайной величиной.
\end{proof}

\begin{remark}
  На практике почти все функции борелевские.
\end{remark}

\begin{remark}
  Возможная ситуация, когда \(\xi\)~--- абсолютно непрерывная случайная
  величина, \(g(x)\)~--- непрерывная функция, однако \(g(\xi)\) имеет дискретное
  распределение.
\end{remark}

\begin{theorem} \label{thr:monotonic-density}
  Пусть \(f_{\xi} (x)\)~--- плотность случайной величины \(\xi\) и функция
  \(g(x)\) строго монотонная. Тогда случайная величина \(\eta = g(\xi)\) имеет
  плотность

  \begin{equation*}
    f_{\eta} (x) = \abs{h' (x)} f_{\xi} (h(x))
    \qquad
    h(x) = g^{-1} (x)
  \end{equation*}
\end{theorem}

\begin{proof}
  Пусть \(g(x)\)~--- возрастающая функция, тогда \(h(x)\) также возрастающая и
  случайная величина \(g(\xi)\) распределена на
  \(\interval{g(-\infty)}{g(+\infty)}\). Ее функция распределения

  \begin{equation*}
    \begin{aligned}
      F_{\eta} (x)
      = \prob{\eta < x}
      = \prob{g(\xi) < x}
      = \prob{\xi < h(x)}
      = \int_{-\infty}^{h(x)} f_{\xi} (t) \dd t
    \\
      \mtxb{
        t = h(y) & \dd t = h' (y) \dd y & y = g(t) \\
        y(-\infty) = g(-\infty) & y(h(x)) = g(h(x)) = x
      }
    \\
      F_{\eta} (x)
      = \int_{g(-\infty)}^{x} f_{\xi} (h(y)) h'(y) \dd y
      \implies f_{\eta} (x) = \abs{h'(x)} f_{\xi} (h(x))
    \end{aligned}
  \end{equation*}

  Пусть \(g(x)\)~--- убывающая функция, тогда \(h(x)\) также убывающая и
  случайная величина \(g(\xi)\) распределена на
  \(\interval{g(+\infty)}{g(-\infty)}\). Ее функция распределения

  \begin{equation*}
    \begin{aligned}
      F_{\eta} (x)
      = \prob{\eta < x}
      = \prob{g(\xi) < x}
      = \prob{\xi > h(x)}
      = \int_{h(x)}^{+\infty} f_{\xi} (t) \dd t
    \\
      \mtxb{
        t = h(y) & \dd t = h' (y) \dd y & y = g(t) \\
        y(h(x)) = g(h(x)) = x & y(+\infty) = g(+\infty)
      }
    \\
      F_{\eta} (x)
      = \int_{x}^{g(+\infty)} f_{\xi} (h(y)) h'(y) \dd y
      = \int_{g(+\infty)}^{x} -h'(y) f_{\xi} (h(y)) \dd y
      \implies f_{\eta} (x) = \abs{h'(x)} f_{\xi} (h(x))
    \end{aligned}
  \end{equation*}
\end{proof}

\begin{theorem}[О линейном преобразовании]
  Пусть случайная величина \(\xi\) имеет плотность \(f_{\xi} (x)\). Тогда
  случайная величина \(\eta = a \xi + b\) имеет плотность

  \begin{equation*}
    f_{\eta} (x) = \frac{1}{\abs{a}} f_{\xi} \prh{\frac{x - b}{a}}
  \end{equation*}
\end{theorem}

\begin{proof}
  Линейное преобразование монотонно, имеем

  \begin{equation*}
    g(x) = a x + b
    \qquad
    h(x) = \frac{x - b}{a}
    \qquad
    h'(x) = \frac{1}{a}
  \end{equation*}

  Подставляя все это в \ref{thr:monotonic-density}, получаем

  \begin{equation*}
    f_{\eta} (x) = \frac{1}{\abs{a}} f_{\xi} \prh{\frac{x - b}{a}}
  \end{equation*}
\end{proof}

\begin{lemma}
  \begin{equation*}
    \xi \in N(a; \sigma^2)
    \implies
    \eta = \gamma \xi + b \in N(\gamma a + b; \gamma^2 \sigma^2)
  \end{equation*}
\end{lemma}

\begin{proof}
  \begin{equation*}
    \begin{aligned}
      f_{\xi} (x)
      = \frac{1}{\sigma \sqrt{2 \pi}} e^{\frac{(x - a)^2}{2 \sigma^2}}
      \qquad
      x \in \interval{-\infty}{+\infty}
    \\
      \eta = \gamma \xi + b \in \interval{-\infty}{+\infty}
    \\
      f_{\eta} (x)
      = \frac{1}{\abs{\gamma}} \frac{1}{\sigma \sqrt{2 \pi}} \exp \prh{
        -\frac{\prh{\frac{x - b}{\gamma} - a}^2}{2 \sigma^2}
      }
      = \frac{1}{\sigma \abs{\gamma} \sqrt{2 \pi}} \exp \prh{
        -\frac{(x - b - \gamma a)^2}{2 (\sigma \gamma)^2}
      }
    \end{aligned}
  \end{equation*}

  Получили плотность нормального распределения с параметрами \(\tilde{a} = b +
  \gamma a\) и \(\tilde{\sigma}^2 = (\gamma \sigma)^2\)
\end{proof}

\begin{lemma}
  Если \(\eta \in N(0; 1)\), то \(\xi = \sigma \eta + a \in N(a; \sigma^2)\)
\end{lemma}

\begin{proof}
  Это следствие из предыдущей леммы.
\end{proof}

\begin{lemma}
  Если \(\eta \in \evenly{0}{1}\), то \(\xi = a \eta + b \in \evenly{b}{b +
  a}\).
\end{lemma}

\begin{proof}
  \todo на практике
\end{proof}

\begin{lemma}
  Если \(\xi \in E_{\alpha}\), то \(\eta = \beta \xi \in
  E_{\frac{\alpha}{\beta}}\).
\end{lemma}

\begin{proof}
  \todo на практике
\end{proof}

\subheader{Квантильное преобразование}

\begin{theorem}
  Пусть функция распределения \(F_{\xi} (x)\) случайной величины \(\xi\)
  непрерывна. Тогда случайная величина \(\eta = F(\xi) \in \evenly{0}{1}\).
\end{theorem}

\begin{proof}
  Ясно, что новая случайная величина \(\eta \in \segment{0}{1}\), т.к. \(0 \le
  F(x) \le 1\). Рассмотрим два случая.

  \subsubheader{Случай I.}{\(F(x)\)~--- строго возрастающая функция}

  В этом случае \(F(x)\) имеет обратную функцию \(F^{-1} (x)\) и 

  \begin{equation*}
    F_{\eta} (x)
    = \prob{\eta < x}
    = \prob{F(\xi) < x}
    = \prob{\xi < F^{-1} (x)}
    = \begin{cases}
      0,                 & x < 0 \\
      F(F^{-1} (x)) = x, & 0 \le x < 1 \\
      1,                 & x \ge 1
    \end{cases}
  \end{equation*}

  Получили функцию распределения \(\in \evenly{0}{1}\).

  \subsubheader{Случай II.}{\(F(x)\)~--- не является строго возрастающей
  функцией, т.е. есть интервалы постоянства}

  В этом случае обозначим через \(F^{-1} (x)\) самую левую точку такого
  интервала:

  \begin{equation*}
    F^{-1}(x) = \min_t \set{t \given F(t) = x}
  \end{equation*}

  Тогда при \(0 < x < 1\)

  \begin{equation*}
    F_{\eta} (x)
    = \prob{F(\xi) < x}
    = \prob{\xi < F^{-1} (x)}
    = F(F^{-1} (x)) = x
  \end{equation*}
\end{proof}

Сформулируем обратную теорему. Пусть \(F(x)\)~--- функция распределения
случайной величины \(\xi\), причем не обязательно непрерывная. Обозначим:

\begin{equation*}
  F^{-1} (x) = \inf_t \set{t \given F(t) \ge x}
\end{equation*}

\begin{theorem}
  Пусть случайная величина \(\eta \in \evenly{0}{1}\), а \(F(x)\)~---
  произвольная функция распределения. Тогда случайная величина \(\xi = F^{-1}
  (\eta)\) имеет функцию распределения \(F(x)\).
\end{theorem}

\begin{proof}
  \begin{equation*}
    F_{\xi} (x)
    = \prob{\xi < x}
    = \prob{F^{-1} (\eta) < x}
    = \prob{\eta < F(x)}
    = \frac{F(x) - 0}{1 - 0}
    = F(x)
  \end{equation*}  
\end{proof}

\begin{remark}
  Датчики случайных чисел имеют распределение \(\evenly{0}{1}\). С помощью
  квантильного преобразования мы можем смоделировать любое другое распределение.
\end{remark}

\begin{example}
  Смоделируем показательное распределение \(E_{\alpha}\).

  \begin{equation*}
    \begin{aligned}
      F(x) = \begin{cases}
        0, & x < 0 \\
        1 - e^{-\alpha x}, & x \ge 0
      \end{cases}
    \\
      \eta = 1 - e^{-\alpha x}
      \implies
      x = -\frac{1}{\alpha} \ln (1 - \eta) = F^{-1} (x)
    \end{aligned}
  \end{equation*}

  Таким образом, если \(\eta \in \evenly{0}{1}\), то
  \(\display{-\frac{1}{\alpha} \ln (1 - \eta) \in E_{\alpha}}\).
\end{example}

\begin{example}
  Смоделируем стандартное нормальное распределение \(N(0; 1)\).

  \begin{equation*}
    F_0 (x) = \frac{1}{\sqrt{2 \pi}} \int_{-\infty}^{x} e^{-z^2 / 2} \dd z
  \end{equation*}

  Пусть \(F_0^{-1} (x)\)~--- обратная функция. Таким образом, если \(\eta \in
  \evenly{0}{1}\), то \(F_0^{-1} (\eta) \in N(0; 1)\).
\end{example}

\subheader{Математическое ожидание преобразованной случайной величины}

\begin{theorem}
  Для произвольной борелевской функции \(g(x)\) справедливо:

  \begin{enumerate}
  \item
    Если \(\xi\) это дискретная случайная величина, то

    \begin{equation*}
      \expected{g(\xi)} = \sum_{i = 1}^{\infty} g(x_i) \prob{\xi = x_i}
    \end{equation*}

    при условии, что данный ряд сходится абсолютно.

  \item
    Если \(\xi\) это абсолютно непрерывная случайная величина, то

    \begin{equation*}
      \expected{g(\xi)} = \int_{-\infty}^{\infty} g(x) f_{\xi} (x) \dd x
    \end{equation*}

    при условии, что данный интеграл сходится абсолютно.
  \end{enumerate}
\end{theorem}

\subheader{Свойства моментов}

\begin{lemma}
  \begin{equation*}
    \xi \ge 0
    \implies
    \expected{\xi} \ge 0
  \end{equation*}
\end{lemma}

\begin{lemma} \label{lm:expected-compare}
  \begin{equation*}
    \xi \ge \eta
    \implies
    \expected{\xi} \ge \expected{\eta}
  \end{equation*}
\end{lemma}

\begin{proof}
  \begin{equation*}
    \xi \ge \eta
    \implies \xi - \eta \ge 0
    \implies \expected{\xi - \eta} = \expected{\xi} - \expected{\eta} \ge 0
    \implies \expected{\xi} \ge \expected{\eta}
  \end{equation*}
\end{proof}

\begin{lemma}
  \begin{equation*}
    \abs{\xi} \ge \abs{\eta}
    \implies
    \expected{\abs{\xi}^k} \ge \expected{\abs{\eta}^k}
  \end{equation*}
\end{lemma}

\begin{lemma}
  Если существует момент \(m_t\) случайной величины \(\xi\), то существуют ее
  моменты меньшего порядка \(m_s (s < t)\).
\end{lemma}

\begin{proof}
  Заметим, что

  \begin{equation*}
    \begin{rcases}
      x \ge 1 \colon & \abs{x}^s \le \abs{x}^t \\
      x < 1   \colon & \abs{x}^s \le 1
    \end{rcases}
    \implies
    \abs{x}^s \le \max \prh{\abs{x}^t, 1} \le \abs{x}^t + 1
    \implies
    \expected{\abs{\xi}^s} < \expected{\abs{\eta}^t} + 1
  \end{equation*}

  Если \(\expected{\abs{\eta}^t} < \infty\), то и \(\expected{\abs{\eta}^s} <
  \infty\).
\end{proof}

\begin{theorem}[Неравенство Йенсена] \label{thr:jensen-inequality}
  Пусть функция \(g(x)\) выпукла вниз. Тогда для любой случайной величины
  \(\xi\) верно неравенство

  \begin{equation*}
    \expected{\xi} \ge g \prh{\expected{\xi}}
  \end{equation*}

  Если функция выпукла вверх, то знак неравенства поменяется на противоположный.
\end{theorem}

\begin{proof}
  Если функция \(g(x)\) выпукла вниз, то в любой точке ее графика можно провести
  прямую, лежащую ниже графика, т.е.

  \begin{equation*}
    \forall x_0
    \given \exists k(x_0)
    \given g(x) \ge g(x_0) + k(x_0) (x - x_0)
  \end{equation*}

  Положим \(x_0 = \expected{\xi}\) и \(x = \xi\), тогда

  \begin{equation*}
    g(\xi) \ge g \prh{\expected{\xi}}
      + k \prh{\expected{\xi}} \prh{\xi - \expected{\xi}}
  \end{equation*}

  По \ref{lm:expected-compare} получаем, что

  \begin{equation*}
    \begin{aligned}
      \expected{g(\xi)} \ge \expected{g \prh{\expected{\xi}}}
        + k \prh{\expected{\xi}} \under{\expected{\xi - \expected{\xi}}}{0}
    \\
      \expected{g(\xi)} \ge g \prh{\expected{\xi}}
    \end{aligned}
  \end{equation*}
\end{proof}

\begin{example}
  С помощью \ref{thr:jensen-inequality} можно, например, получить следующие
  неравенства:

  \begin{equation*}
    \begin{aligned}
      \expected{e^{\xi}} \ge e^{\expected{\xi}} &
    \\
      \expected{\xi^n} \ge \prh{\expected{\xi}}^n \qquad (n > 1)
    \\
      \expected{\ln \xi} \le \ln \prh{\expected{\xi}} &
    \end{aligned}
  \end{equation*}
\end{example}
