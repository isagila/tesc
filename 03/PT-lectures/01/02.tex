\subsection{%
  Лекция \texttt{23.09.12}.%
}

Будем строить аксиоматическое определение вероятности. Пусть у нас есть
\(\Omega\)~--- пространство элементарных исходов случайного эксперимента.

\begin{definition} \label{def:sigma-ev-alg}
  Система \(\euF\) подмножеств \(\Omega\) называется \(\sigma\)-алгеброй
  событий, если выполнены следующие свойства

  \begin{enumerate}
  \item
    \(\Omega \in \euF\)
  
  \item
    \(A \in \euF \implies \bar{A} \in \euF\) (замкнутость относительно
    дополнения)
  
  \item
    \(
      A_1, A_2, \dotsc \in \euF \implies
      \bigcup_{i = 1}^{\infty} A_i \in \euF
    \) (замкнутость относительно счетного объединения)
  \end{enumerate}
\end{definition}

\begin{remark}
  В \ref{def:sigma-ev-alg} свойство \(1^{\circ}\) избыточно и может быть
  выведено из \(2^{\circ}\) и \(3^{\circ}\).

  \begin{equation*}
    A \in \euF \implies[\(2^{\circ}\)]
    \bar{A} \in \euF \implies[\(3^{\circ}\)]
    \under{\prh{A \cup \bar{A}}}{\Omega} \in \euF
  \end{equation*}
\end{remark}

\subheader{Свойства \(\sigma\)-алгебры событий}

\begin{lemma}
  \begin{equation*}
    \varnothing \in \euF
  \end{equation*}
\end{lemma}

\begin{proof}
  \begin{equation*}
    \Omega \in \euF \implies[\(2^{\circ}\)]
    \varnothing = \bar{\Omega} \in \euF
  \end{equation*}
\end{proof}

\begin{lemma}
  \(\sigma\)-алгебра событий замкнута относительно счетного объединения.
  
  \begin{equation*}
    A_1, A_2, \dotsc \in \euF \implies
    \bigcap_{i = 1}^{\infty} A_i \in \euF
  \end{equation*}
\end{lemma}

\begin{proof}
  Пусть \(A_1, A_2, \dotsc \in \euF\), тогда

  \begin{equation*}
    A_1, A_2, \dotsc \in \euF \implies[\(2^{\circ}\)]
    \bar{A_1}, \bar{A_2}, \dotsc \in \euF \implies[\(3^{\circ}\)]
    \bigcup_{i = 1}^{\infty} \bar{A_i} \in \euF \implies[\(2^{\circ}\)]
    \bar{\bigcup_{i = 1}^{\infty} \bar{A_i}} \in \euF
  \end{equation*}

  Применяя закон де Моргана (который работает не только в конечном, но и в
  счетном случае), получаем искомое.

  \begin{equation*}
    \bigcap_{i = 1}^{\infty} A_i \in \euF
  \end{equation*}
\end{proof}

\begin{lemma}
  \begin{equation*}
    A, B \in \euF \implies
    A \setminus B \in \euF
  \end{equation*}
\end{lemma}

\begin{proof}
  \begin{equation*}
    A, B \in \euF \implies[\(2^{\circ}\)]
    \bar{B} \in \euF \implies[\(3^{\circ}\)]
    A \setminus B = A \bar{B} \in \euF
  \end{equation*}
\end{proof}

\begin{example}
  \begin{enumerate}
  \item
    \(\euF = \set{\varnothing, \Omega}\)~--- тривиальная.
  
  \item
    \(\euF = \set{\varnothing, \Omega, A, \bar{A}}\)

  \item
    Пусть \(\Omega = \RR\), тогда \(\euF\) это \(\sigma\)-алгебра Борелевских
    множеств на прямой.
  \end{enumerate}
\end{example}

\begin{definition}
  Пусть \(\Omega\)~--- пространство элементарных исходов и \(\euF\) его
  \(\sigma\)-алгебра. Вероятностью на \(\pair{\Omega, \euF}\) называется функция
  \(\probP \colon \euF \to \RR\), обладающая следующими свойствами

  \begin{enumerate}
  \item
    Неотрицательность. \(\forall A \in \euF \given \prob{A} \ge 0\)
  
  \item
    Счетная аддитивность. Если события \(A_1, A_2, \dotsc\) попарно несовместны,
    то \(\display{\prob{\sum_{i = 1}^{\infty} A_i} = \sum_{i = 1}^{\infty}
    \prob{A_i}}\).

  \item
    Нормированность. \(\prob{\Omega} = 1\)
  \end{enumerate}
\end{definition}

\begin{remark}
  Таким образом, вероятность это нормированная мера, а свойства \(1\)--\(3\) 
  называются аксиомами вероятности.
\end{remark}

\begin{definition}
  Тройка \(\tuple{\Omega, \euF, \probP}\) называется вероятностным
  пространством, где \(\Omega\) это пространство элементарных исходов, \(\euF\)
  его \(\sigma\)-алгебра, а \(\probP\)~--- нормированная мера.
\end{definition}

\subheader{Свойства вероятности}

\begin{lemma}
  \begin{equation*}
    \prob{\varnothing} = 0
  \end{equation*}
\end{lemma}

\begin{proof}
  \begin{equation*}
    1
    = \prob{\Omega}
    = \prob{\varnothing + \Omega}
    = \prob{\varnothing} + \prob{\Omega}
    = \prob{\varnothing} + 1
    \implies
    \prob{\varnothing} = 0
  \end{equation*}
\end{proof}

\begin{lemma}[Формула обратной вероятности] \label{lem:inv-prob}
  \begin{equation*}
    \prob{A} = 1 - \prob{\bar{A}}
  \end{equation*}
\end{lemma}

\begin{proof}
  Т.к. \(A\) и \(\bar{A}\) несовместны и \(A + \bar{A} = \Omega\), то

  \begin{equation*}
    1 = \prob{A + \bar{A}} = \prob{A} + \prob{\bar{A}}
    \implies
    \prob{A} = 1 - \prob{\bar{A}}
  \end{equation*}
\end{proof}

\begin{lemma}
  \begin{equation*}
    0 \le \prob{A} \le 1
  \end{equation*}
\end{lemma}

\begin{proof}
  \(\prob{A} \ge 0\) по первой аксиоме вероятности. Далее по \ref{lem:inv-prob}
  имеем, что \(\prob{A} = 1 - \prob{\bar{A}}\). Т.к. \(\prob{\bar{A}} \ge 0\),
  то \(\prob{A} \le 1\).
\end{proof}

\subheader{Аксиома непрерывности}

Пусть имеется убывающая цепочка событий \(A_1 \supset A_2 \supset \dotsc A_n
\supset \dotsc\) и \(\display{\bigcap_{i = 1}^{\infty} A_i = \varnothing}\).
Тогда \(\prob{A_n} \to 0\) при \(n \to \infty\).

\begin{remark}
  Смысл аксиомы непрерывности заключается в том, что при непрерывном изменении
  области \(A \subset \Omega \subset \RR^n\) соответствующая вероятность
  \(\prob{A}\) также должна изменяться непрерывно.
\end{remark}

\galleryone{01_02_01}{Иллюстрация к \ref{thr:axi-cont}}

\begin{theorem} \label{thr:axi-cont}
  Аксиома непрерывности следует из аксиомы счетной аддитивности.
\end{theorem}

\begin{proof}
  Изобразим события \(A_1, A_2, \dotsc\) графически (\figref{01_02_01}). Выразим
  \(A_n\) согласно полученному рисунку.

  \begin{equation*} \label{eq:axi-cont-1} \tag{1}
    A_n = \under{\sum_{i = n}^{\infty} A_i \bar{A_{i + 1}}}{\text{кольца}}
      + \under{\bigcap_{i = n}^{\infty} A_i}{\text{\quote{сердцевина}}}
  \end{equation*}

  Далее заметим, что

  \begin{equation*} \label{eq:axi-cont-2} \tag{2}
    \bigcap_{i = n}^{\infty} A_i
    = A_n \cap \bigcap_{i = n + 1}^{\infty} A_i
    = \bigcap_{i = 1}^{n} A_i \cap \bigcap_{i = n + 1}^{\infty} A_i
    = \bigcap_{i = 1}^{\infty} A_i
    \eqby{усл} \varnothing
  \end{equation*}

  Подставим \eqref{eq:axi-cont-2} в \eqref{eq:axi-cont-1} и получим, что

  \begin{equation*} \label{eq:axi-cont-3} \tag{3}
    \begin{aligned}
      A_n = \sum_{i = n}^{\infty} A_i \bar{A_{i + 1}}
    \\
      A_1
      = \sum_{i = 1}^{\infty} A_i \bar{A_{i + 1}}
      = \sum_{i = 1}^{n} A_i \bar{A_{i + 1}}
        + \under{\sum_{i = n}^{\infty} A_i \bar{A_{i + 1}}}{A_n}
    \end{aligned}
  \end{equation*}

  Далее воспользуемся аксиомой счетной аддитивности.

  \begin{equation*} \label{eq:axi-cont-4} \tag{4}
    \prob{A_1}
      = \sum_{i = 1}^{n} \prob{A_i \bar{A_{i + 1}}} + \prob{A_n}
      = \under{\sum_{i = 1}^{\infty} \prob{A_i \bar{A_{i + 1}}}}{S}
  \end{equation*}

  Получаем числовой ряд \(S\), который будет сходиться, т.е. его сумма равна
  \(\prob{A_1}\), которая принадлежит отрезку \(\segment{0}{1}\). Тогда
  \(\prob{A_n}\) это \quote{хвост} ряда. Т.к. ряд сходится, то его хвост
  стремится к нулю при \(n \to \infty\). Итого \(\prob{A_n} \to 0\) при \(n \to
  \infty\).
\end{proof}

\begin{remark}
  Можно доказать, что счетная аддитивность следует из конечной аддитивности и
  аксиомы непрерывности.
\end{remark}

\subheader{Свойства операций сложения и умножения}

Свойство дистрибутивности \(A (B + C) = A B + A C\). Также
если события \(A\) и \(B\) несовместны, то \(\prob{A + B} = \prob{A} +
\prob{B}\) (это частный случай аксиомы \(2^{\circ}\)).

\galleryone{01_02_02}{Иллюстрация к \ref{lem:prob-of-sum}}

\begin{lemma} \label{lem:prob-of-sum}
  В общем случае (если про совместность событий \(A\) и \(B\) ничего не
  известно) имеем формулу

  \begin{equation*}
    \prob{A + B} = \prob{A} + \prob{B} - \prob{A B}
  \end{equation*}
\end{lemma}

\begin{proof}
  Т.к. \(A + B = A \bar{B} + A B + \bar{A} B\) (\figref{01_02_02}), то по второй
  аксиоме

  \begin{equation*}
    \begin{aligned}
      \prob{A + B}
      & = \prob{A \bar{B}} + \prob{A B} + \prob{\bar{A} B}
    \\
      & = \under{\prh{\prob{A \bar{B}} + \prob{A B}}}{\prob{A}}
        + \under{\prh{\prob{\bar{A} B} + \prob{A B}}}{\prob{B}}
        - \prob{A B}
    \\
      & = \prob{A} + \prob{B} - \prob{A B}
    \end{aligned}
  \end{equation*}
\end{proof}

\begin{example}
  Из колоды в \(36\) карт достали одну карту. Какова вероятность того, что это
  будет дама или пика?

  \solution{} пусть событие \(D\)~--- \quote{выпала дама}, а событие \(S\)~---
  \quote{выпала пика}. Обозначив \(A\) искомое событие, имеем

  \begin{equation*}
    \prob{A}
    = \prob{D} + \prob{S} - \prob{D S}
    = \frac{4}{36} + \frac{9}{36} - \frac{1}{36}
    = \frac{12}{36}
    = \frac{1}{3}
  \end{equation*}
\end{example}

При \(n = 3\) формула усложняется.

\begin{equation*}
  \prob{A_1 + A_2 + A_3}
  = \prob{A_1} + \prob{A_2} + \prob{A_3}
    - \prob{A_1 A_2} - \prob{A_1 A_3} - \prob{A_2 A_3}
    + \prob{A_1 A_2 A_3}
\end{equation*}

В общем же случае получаем следующую формулу

\begin{equation*}
  \prob{A_1 + A_2 + \dotsc + A_n}
  = \sum_{i = 1}^{n} \prob{A_i}
    - \sum_{i < j} \prob{A_i A_j}
    + \sum_{i < j < k} \prob{A_i A_j A_k}
    - \dotsc
    + (-1)^{n - 1} \prob{A_1 \dotsc A_n}
\end{equation*}

\begin{example}
  В \(n\) конвертов раскладываются \(n\) писем. Какова вероятность того, что
  хотя бы одно письмо попадет в свой конверт? Куда стремится эта вероятность при
  \(n \to \infty\).

  \solution{} пусть \(A_i\)~--- \(i\)-тое письмо в своем конверте, а событие
  \(A\)~--- хотя бы одно письмо в своем конверте. Тогда \(A = A_1 + \dotsc +
  A_n\). Применим общую формулу для вероятности суммы. Сначала вычислим
  вероятности пар, троек и т.д.

  \begin{equation*}
    \prob{A_i} = \frac{1}{n}
    \qquad
    \prob{A_i A_j} = \frac{1}{\place{n}{2}}
    \qquad
    \prob{A_i A_j A_k} = \frac{1}{\place{n}{3}}
    \qquad
    \prob{A_1 \dotsc A_n} = \frac{1}{\place{n}{n}} = \frac{1}{n!}
  \end{equation*}

  Далее вычислим число пар, троек и т.д.

  \begin{equation*}
    \prob{A_i} \to n = \comb{n}{1}
    \qquad
    \prob{A_i A_j} \to \comb{n}{2}
    \qquad
    \prob{A_i A_j A_k} \to \comb{n}{3}
  \end{equation*}

  Подставим это в формулу.

  \begin{equation*}
    \prob{A}
    = n \cdot \frac{1}{n}
      - \comb{n}{2} \cdot \frac{1}{\place{n}{2}}
      + \comb{n}{3} \cdot \frac{1}{\place{n}{3}}
      - \dotsc
      + (-1)^{n- 1} \frac{1}{n!}
    = 1 - \frac{1}{2!} + \frac{1}{3!} - \dotsc + (-1)^{n- 1} \frac{1}{n!}
  \end{equation*}

  С помощью разложения в ряд Тейлора функции \(e^{-1}\) можно показать, что
  полученная сумма будет примерно равна \(1 - e^{-1} \approx 0.63\).
\end{example}

\subheader{Независимые события}

Под независимыми событиями логично понимать события, не связанные
причинно-следственной связью, т.е. факт наступления одного события не влияет на
оценку вероятности второго события.

Рассмотрим это на примере классической вероятности. Пусть дано \(\Omega\)~---
пространство элементарных исходов, и два события \(\abs{A} = m_1\) и \(\abs{B} =
m_2\). Проведем пару независимых испытаний, тогда получим пространство
элементарных исходов \(\Omega \times \Omega\), \(\abs{\Omega \times \Omega} =
n^2\) и \(\abs{A B} = m_1 m_2\). Значит

\begin{equation*}
  \prob{A B}
  = \frac{m_1 m_2}{n^2}
  = \frac{m_1}{n} \cdot \frac{m_2}{n}
  = \prob{A} \prob{B}  
\end{equation*}

\begin{definition}
  События \(A\) и \(B\) называются независимыми, если \(\prob{A B} = \prob{A}
  \prob{B}\).
\end{definition}

\begin{lemma}
  Если \(A\) и \(B\) независимы, то \(A\) и \(\bar{B}\), \(\bar{A}\) и \(B\),
  \(\bar{A}\) и \(\bar{B}\) также независимы.
\end{lemma}

\begin{proof}
  Т.к. \(A = A (B + \bar{B}) = A B + A \bar{B}\), то

  \begin{equation*}
    \begin{aligned}
      \prob{A} = \prob{A B} + \prob{A \bar{B}} \implies &
    \\
      & \prob{A \bar{B}}
      = \prob{A} - \prob{A B}
    \\
      & = \prob{A} - \prob{A} \prob{B}
    \\
      & = \prob{A} (1 - \prob{B})
    \\
      & = \prob{A} \prob{\bar{B}}
    \end{aligned}
  \end{equation*}

  Таким образом события \(A\) и \(\bar{B}\) независимы. Остальные свойства
  доказываются аналогично.
\end{proof}

\begin{definition}
  События \(A_1, \dotsc, A_n\) называются независимыми в совокупности, если для
  любого набора номеров вероятность произведения событий с этими номерами
  равняется произведению отдельных вероятностей.
\end{definition}

\begin{remark}
  Из независимости в совокупности при \(k = 2\) следует попарная независимость.
  Обратное в общем случае неверно. Независимость в совокупности это более
  сильное свойство.
\end{remark}

\begin{example}
  Три грани правильного тетраэдра раскрашены в красный, синий, зеленый цвета, а
  четвертая грань~--- во все эти три цвета. Пусть \(A\)~--- \quote{выпала грань
  с красным цветом}, \(B\)~--- \quote{выпала грань с синим цветом} и \(C\)~---
  \quote{выпала грань с зеленым цветом}.

  \begin{equation*}
    \begin{aligned}
      \prob{A} = \prob{B} = \prob{C} = \frac{2}{4} = \frac{1}{2}
    \\
      \prob{A B} = \prob{A C} = \prob{B C} = \frac{1}{4}
    \end{aligned}
  \end{equation*}

  Т.к. \(\prob{A B} = \prob{A} \prob{B}\), то \(A\) и \(B\)
  независимы. Аналогично попарно независимы \(A\) и \(C\), \(B\) и \(C\), т.е.
  все события попарно независимы.

  \begin{equation*}
    \prob{A B C} = \frac{1}{4}
    \neq
    \prob{A} \prob{B} \prob{C} = \frac{1}{8}
  \end{equation*}

  Т.е. события \(A\), \(B\) и \(C\) не являются независимыми в совокупности.
\end{example}

\begin{remark}
  В дальнейшем под \quote{независимыми событиями} будем подразумевать
  независимые в совокупности события.
\end{remark}

\begin{example}
  Какова вероятность того, что при четырех бросках кости выпадет хотя бы одна
  шестерка.

  \solution{} пусть событие \(A_i\)~--- на \(i\)-том броске (\(1 \le i \le 4\))
  выпала шестерка. Обозначим \(B\) событие \quote{выпала хотя бы одна шестерка},
  тогда \(B = A_1 + A_2 + A_3 + A_4\). Получаем, что \(\bar{B}\) это событие
  \quote{не выпала ни одна шестерка}, тогда \(\bar{B} = \bar{A_1} \cdot
  \bar{A_2} \cdot \bar{A_3} \cdot \bar{A_4}\). Т.к. броски кубиков независимые,
  то

  \begin{equation*}
    \prob{\bar{B}}
    = \prob{\bar{A_1}} \prob{\bar{A_2}} \prob{\bar{A_3}} \prob{\bar{A_4}}
    = \prh{\frac{5}{6}}^4 \approx 0.48
  \end{equation*}

  Значит, \(\prob{B} = 1 - \prob{\bar{B}} \approx 0.52\).
\end{example}
