\subsection{%
  \textcolor{red}{!pre} Лекция \texttt{23.09.26}.%
}

\subheader{Последовательности независимых испытаний}

\begin{definition}
  Схемой Бернулли называется серия одинаковых независимых испытаний, каждое из
  которых имеет лишь два исхода: произошло интересующее нас событие (успех) или
  не произошло (неудача). Обозначения:

  \begin{enumerate}
  \item
    \(p\)~--- вероятность успеха при одном испытании.

  \item
    \(q = 1 - p\)~--- вероятность неудачи при одном испытании.

  \item
    \(n\)~--- число независимых испытаний.

  \item
    \(Y_n\)~--- число успехов при \(n\) испытаниях.

  \item
    \(N_n\)~--- число неудач при \(n\) испытаниях.

  \item
    Для краткости обозначим \(p(Y_n = k) = p_n(k)\).
  \end{enumerate}
\end{definition}

\subheader{Формула Бернулли}

\begin{theorem}
  Вероятность того, что при \(n\) испытаниях произойдет ровно \(k\) успехов
  будет равна

  \begin{equation*}
    p_n(k) = \comb{k}{n} p^k q^{n - k}
  \end{equation*}
\end{theorem}

\begin{proof}
  Пусть \(A = \set{Y_n = k}\). Рассмотрим один из элементарных исходов,
  благоприятствующих событию \(A\).

  \begin{equation*}
    A_1 = \under{Y \dotsc Y}{k} \under{N \dotsc N}{n - k}
    \qquad
    \prob{Y} = p, \prob{N} = q
  \end{equation*}

  Т.к. испытания независимы, то

  \begin{equation*}
    \prob{A_1} = p^k q^{n - k}
  \end{equation*}

  Остальные элементарные исходы (благоприятствующие \(A\)) отличаются от данного
  лишь расстановкой \(k\) успехов по \(n\) испытаниям, а вероятности их те же
  самые. Всего таких способов расставить будет \(\display{\comb{k}{n}}\). В
  результате получаем искомую формулу.
\end{proof}

\begin{example}
  Вероятность попасть при одном выстреле равна \(0.8\).
\end{example}
