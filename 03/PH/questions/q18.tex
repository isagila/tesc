\subsection{Линейный интеграл электростатического поля. Циркуляция вектора напряженности ЭС поля. Потенциальность ЭС поля. Электрический потенциал}

\begin{definition}
    Работа поля при перемещении зарядов.

    Линейный интеграл ЭС поля — работа, совершенная этим полем, по перемещению заряда из точки $1$ в точку $2$.

    $$A_{12}=\int\limits_1^2\vec Fd\vec l=q_0\int\limits_1^2\vec Ed\vec l$$

    или

    $$A_{12}=\int\limits_1^2dA=\int\limits_{r_1}^{r_2}k\frac{Qq}{r^2}dr=kQq\bigg(\frac{1}{r_1}-\frac{1}{r_2}\bigg)$$
\end{definition}\

Работа сил электрического поля по перемещению электрического заряда не зависит от траектории заряда. Она зависит только от его начального и конечного положений.

\begin{definition}
    Электрический потенциал (также называемый потенциалом электрического поля, падением потенциала, электростатическим потенциалом) 
    определяется как количество работы энергии, необходимое на единицу электрического заряда для перемещения заряда из контрольной 
    точки в определенную точку электрического поля.

    Численно потенциал равен потенциальной энергии, которой обладал бы в данной точке поля единичный положительный заряд.

    $$\varphi=k\frac{Q}{r}$$
\end{definition}

\begin{definition}
    Циркуляция вектора напряженности электрического поля.

    Если траектория движения заряда замкнута, то $\varphi_1=\varphi_2$ и соответственно работа по его перемещению равна нулю. Тогда:

    $$A=\oint\limits_lqE_ldl=q\oint\limits_kE_ldl=0$$

    Поскольку $q\ne0$, то $E_ldl=0$.
\end{definition}

Работа сил поля по перемещению заряда по замкнутому контуру равна 0,

Циркуляция вектора его напряженности по любому замкнутому контуру равна 0,

$\implies$ это потенциальное (консервативное) поле. Тогда:

$$A_{12}=W_{P1}-W_{P2}$$