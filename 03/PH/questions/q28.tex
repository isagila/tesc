\subsection{Электроемкость. Поля плоского, цилиндрического и сферического конденсаторов. Энергия заряженного конденсатора}

\begin{definition}
    Электроёмкость $[C, Ф_{фарады}]$ — характеристика проводника, мера его способности накапливать электрический заряд.

    $$
    q=C\varphi
    $$
\end{definition}

Ёмкость уединенного проводника зависит от его формы, размеров и диэлектрических свойств среды, в которой находится проводник, 
а также электрических свойств, расположения, форм и размеров окружающих тел.

\begin{definition}
    Конденсатор — система проводников, электростатическое поле которых полностью сосредоточено в объеме, занимаемом этой системой.

    Формула заряда на одной из обкладок конденсатора:

    $$
    q=C(\varphi_1-\varphi_2)=CU
    $$
\end{definition}


\begin{definition}
    Энергия заряженного конденсатора

Процесс возникновения на обкладках конденсатора зарядов $+q\ и -q$ можно представить так, что от одной обкладки последовательно отнимаются 
малые порции заряда $\Delta q$ и перемещаются на другую обкладку.
\end{definition}

Работа переноса очередной порции заряда:

$$
\Delta A=\Delta q(\varphi_1-\varphi_2)=\Delta qU
$$

Тогда энергия:

$$
dW=dA=Udq=\frac{q}{C}dq
$$

Интегрируя, получаем формулу энергии заряженного конденсатора:

$$
W=\frac{q^2}{2C}=\frac{qU}{2}=\frac{CU^2}{2}
$$

