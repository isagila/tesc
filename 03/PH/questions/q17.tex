\subsection{Применение теоремы Гаусса. Сферически симметричное поле. Поле системы точечных зарядов, нити, плоскости}

*Примечание: на самом деле информации сверхмного, но я не знаю, что писать*

Применение - вычисление электрических полей при симметричных распределениях зарядов.
\begin{itemize}
   

    \item  поле бесконечной, равномерно заряженной плоскости в вакууме
    
    $$
    E=\frac{\sigma}{2\varepsilon_0}
    $$
    
    \item поле двух бесконечных параллельных разноименно заряженных плоскостей с одинаковыми поверхностными плотностями зарядов в вакууме
    
    $$
    A^\circ=\frac{\sigma}{\varepsilon_0}
    $$
    
    \item поле равномерно заряженной сферы
    
    $$
    Ф_A^\circ=A^\circ\cdot4\pi r'^2
    $$
    
    \item поле равномерно заряженного шара
    
    $$
    Ф_E=E\cdot4\pi r'^2=\frac{q'}{\varepsilon_0}=\frac{q}{\varepsilon_0}\frac{r'^3}{R^3}
    $$
    
    \item поле равномерно заряженной бесконечной прямой нити
    
    $$
    Ф_A^\circ=E\cdot2\pi rl=\frac{\gamma l}{\varepsilon_0}
    $$
\end{itemize}