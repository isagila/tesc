\subsection{Градиент скалярной функции. Связь между напряженностью и потенциалом. Расчет напряженности по заданному распределению потенциалов}

\begin{definition}
    Градиент — вектор, своим направлением указывающий направление возрастания некоторой скалярной величины

    $$grad\varphi=\big(\frac{\partial\varphi}{\partial x};\frac{\partial\varphi}{\partial y};\frac{\partial\varphi}{\partial z}\big)$$

\end{definition}

\begin{definition}
    Связь между напряжённостью и потенциалом.

    Напряжённость электрического поля равна градиенту его электрического потенциала, взятого с обратным знаком:

    $$E=-grad\varphi$$
\end{definition}

Зависимость электрического потенциала можно записать уравнением $\varphi=f(x,y,z)$, 
тогда проекции вектора напряженности на оси координат имеют вид:

$$E_x=-\frac{\partial\varphi}{\partial x},\ E_y=-\frac{\partial\varphi}{\partial y},\ E_z=-\frac{\partial\varphi}{\partial z},\ $$

Сам  вектор можно представить в виде

$$\vec{E} = E_x \vec{i} + E_y \vec{j} + E_z \vec{k} = -\left( \frac{d\phi}{dx}\vec{i} + \frac{d\phi}{dy}\vec{j} + \frac{d\phi}{dz}\vec{k}\right)$$

, что является градиентом функции $\varphi$:

$$E=-grad\varphi$$
