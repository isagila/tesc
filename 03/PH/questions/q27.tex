\subsection{Диэлектрическая восприимчивость и диэлектрическая проницаемость. Вектор электрического смещения}

\begin{definition}
    Диэлектрическая проницаемость $[\varepsilon]$ — физическая величина, которая показывает, во сколько раз увеличивается 
    емкость конденсатора при заполнении пространства между его обкладками диэлектриком 

    Показывает, во сколько раз напряженность одного и того же поля в вакууме больше, чем в диэлектрике.

    Для всех без исключения веществ, $\varepsilon>1$. Для вакуума $\varepsilon=1$.
\end{definition}

Ёмкость плоского конденсатора в вакууме:

$$
C_0=\frac{\varepsilon_0S}{d}
$$

а при внесении в пространство между его обкладками диэлектрика:

$$
C\varepsilon C_0=\frac{\varepsilon\varepsilon_0S}{d}
$$

при этом заряд на обкладках конденсатора не меняется.

\begin{definition}
    Диэлектрическая восприимчивость

    И для полярных, и для неполярных диэлектриков, которые находятся в не слишком сильных электрических полях, выполняется условие:

    $$
    \vec P=\varepsilon_0\chi\vec E
    $$
\end{definition}

где коэффициент пропорциональности $\chi$ называется электрической восприимчивостью диэлектрика.

($P$ - поляризация диэлектрика, $E$ - внешнее электрическое поле)

\begin{remark}
    Связь между диэлектрической проницаемостью и электрической восприимчивостью диэлектрика:
\end{remark}

$$
\vec D=\varepsilon_0\vec E+\varepsilon_0\chi\vec R=\varepsilon_0(1+\chi)\vec E=\varepsilon\varepsilon_0\vec E \implies \varepsilon = 1+ \chi
$$

В случае вакуума, $\chi=0$.

Электрическая восприимчивость, как и диэлектрическая проницаемость — величина безразмерная.

Главная задача электростатики – расчет электрических полей, то есть $\vec E_0$ в различных электрических аппаратах, кабелях, конденсаторах, и т.д. Эти расчеты сами по себе не просты, да еще наличие разного сорта диэлектриков и проводников еще более усложняют задачу.

Для упрощения расчётов была введена новая векторная величина — вектор электрического смещения (электрическая индукция)

$$
\vec D=\varepsilon\varepsilon_0\vec E
$$

Для точечного заряда в вакууме:

$$
D=\frac{q}{4\pi r^2}
$$