\subsection{%
  Лекция \texttt{23.10.20}.%
}

Множество функций, непрерывных на отрезке \(\segment{a}{b}\), образуют
линейное пространство. Определим скалярное произведение и норму

\begin{equation*}
  \dotpdt{f}{g} = \int_a^b f(x) g(x) \dd x
  \qquad
  \norm{f} = \sqrt{\dotpdt{f}{f}} = \sqrt{\int_a^b f^2 (x) \dd x}
\end{equation*}

и получим Евклидово пространство. Рассмотрим следующую ортонормированную систему

\begin{equation*}
  \set{
    \frac{1}{\sqrt{2 \pi}},
    \frac{1}{\sqrt{\pi}} \sin x,
    \frac{1}{\sqrt{\pi}} \cos x,
    \dotsc
    \frac{1}{\sqrt{\pi}} \sin n x,
    \frac{1}{\sqrt{\pi}} \cos n x
  }
\end{equation*}

Натянем на нее линейную оболочку.

\begin{equation*}
  P_n (x)
  = \frac{a_0}{2} + \sum_{k = 1}^{\infty} \prh{a_k \cos k x + b_k \sin k x}
\end{equation*}

Получим подпространство. Ортогональная проекция \(f_{\perp}\) это минимально
отстоящий от \(f\) многочлен \(P_n (x)\). Его коэффициенты называются
коэффициентами Фурье и имеют вид

\begin{equation*}
  a_n = \frac{1}{\pi} \int_{-\pi}^{\pi} f(x) \cos n x \dd x
  \qquad
  b_n = \frac{1}{\pi} \int_{-\pi}^{\pi} f(x) \sin n x \dd x
\end{equation*}

В таком случае многочлен \(P_n (x)\) называется многочленом Фурье. Определим
расстояние от \(f(x)\) до \(P_n (x)\) как среднеквадратическое отклонение (СКО):

\begin{equation*}
  \delta^2 = \frac{1}{\pi} \int_{-\pi}^{\pi} \prh{
    f(x) - \frac{\alpha_0}{2}
      - \sum_{k = 1}^{n} \prh{\alpha_k \cos k x + \beta_k \sin k x}
  }^2 \dd x
\end{equation*}

Далее будет показано, что именно многочлен Фурье является минимально отстоящим,
т.е. \(\delta^2\)~--- наименьшее при \(\alpha_k = a_k\) и \(\beta_k = b_k\).
Более того, \(\delta \to 0\) при \(n \to \infty\). Сейчас определим ряд Фурье и
изучим его свойства.

\begin{definition}[Тригонометрический ряд Фурье]
  Дана \(f(x)\)~--- периодическая, \(T = 2 \pi\). Тогда

  \begin{equation*}
    \begin{aligned}
      f(x) = \frac{a_0}{2}
        + \sum_{k = 1}^{\infty} \prh{a_k \cos k x + b_k \sin k x}
    \\
      a_k = \frac{1}{\pi} \int_{-\pi}^{\pi} f(x) \cos k x \dd x
      \qquad
      b_k = \frac{1}{\pi} \int_{-\pi}^{\pi} f(x) \sin k x \dd x
    \end{aligned}
  \end{equation*}

  называется представлением функции \(f(x)\) тригонометрическим рядом Фурье.
\end{definition}

\begin{theorem}[Критерий сходимости ряда Фурье к значению функции. Условие
Дирихле]
  \(f(x) \colon \segment{-\pi}{\pi} \to \RR\)~--- кусочно--непрерывна,
  кусочно--монотонна, ограничена. Тогда если \(S(x)\) это сумма ряда Фурье, то
  во всех внутренних точках \(S(x) = f(x)\), а точках конечных разрывов \(x_0\):
  \(S(x) = \frac{1}{2} \prh{f(x_0 - 0) + f(x_0 + 0)}\). При этом \(f(-\pi) =
  f(\pi) = \frac{1}{2} \prh{f(-\pi + 0) + f(\pi - 0)}\).
\end{theorem}

\begin{remark}
  Значения функции в точках разрыва не влияют на ее ряд Фурье, поэтому две
  функции равные везде, кроме точек разрыва, имеют один и тот же ряд.
\end{remark}

\begin{remark}
  За пределами отрезка \(\segment{\-\pi}{\pi}\) функция будет представлена тем
  же рядом, если она периодична с периодом \(2\pi\). Т.к.  \(\cos k x\) и \(\sin
  k x\) периодичны с \(T = 2 \pi\), то

  \begin{equation*}
    S_n (x + T) = S_n (x)
    \qquad
    S (x + T) = \lim_{n \to \infty} S_n (x + T) = S(x)
  \end{equation*}
\end{remark}

\begin{example}
  Пусть \(f(x) = x\) на отрезке \(\segment{-\pi}{\pi}\). Условие Дирихле
  выполнено, найдем коэффициенты Фурье.

  \begin{equation*}
    \begin{aligned}
      a_0
      & = \frac{1}{\pi} \int_{-\pi}^{\pi} f(x) \cos 0 x \dd x
      = \frac{1}{2} \int_{-\pi}^{\pi} x \dd x
      = 0
    \\
      a_k
      & = \frac{1}{\pi} \int_{-\pi}^{\pi} f(x) \cos k x \dd x
      = \frac{1}{\pi k} \int_{-\pi}^{\pi} x \dd \sin x
      = \frac{1}{\pi k} \prh{
        (x \sin k x) \Big\vert_{-\pi}^{\pi}
        - \int_{-\pi}^{\pi} \sin k x \dd x
      }
      = \frac{1}{\pi k^2} \cos k x \Big\vert_{-\pi}^{\pi}
      = 0
    \\
      b_k
      & = \frac{1}{\pi} \int_{-\pi}^{\pi} f(x) \sin k x \dd x
      = -\frac{1}{\pi k} \int_{-\pi}^{\pi} x \dd \cos x
      = -\frac{1}{\pi k} \prh{
        (x \cos k x) \Big\vert_{-\pi}^{\pi}
        - \int_{-\pi}^{\pi} \cos k x \dd x
      }
      = \frac{1}{\pi k} \prh{\pi \cos \pi k + \pi \cos \pi k}
      = \frac{2}{k} (-1)^{k + 1}
    \end{aligned}
  \end{equation*}

  Тогда получаем, что

  \begin{equation*}
    \begin{aligned}
      x = \sum_{k = 1}^{\infty} \frac{2}{k} (-1)^{k + 1} \sin k x
    \\
      S(-\pi) = S(\pi) = \frac{1}{2} \prh{f(-\pi + 0) + f(\pi - 0)} = 0
    \end{aligned}
  \end{equation*}
\end{example}

\begin{remark}
  В примере пользовались свойствами интегралов от четных и нечетных функций.
  Заметим, что если \(f(x)\)~--- четная, то

  \begin{equation*}
    a_k = \frac{2}{\pi} \int_0^{\pi} f(x) \cos k x \dd x
    \qquad
    b_k = 0
  \end{equation*}

  А если \(f(x)\)~--- нечетная, то

  \begin{equation*}
    a_k = 0
    \qquad
    b_k = \frac{2}{\pi} \int_0^{\pi} f(x) \sin k x \dd x
  \end{equation*}
\end{remark}

Как изменится ряд Фурье, если \(f(x) \colon \segment{a}{b} \to \RR\), но
\(\segment{a}{b} \ne \segment{-\pi}{\pi}\)? Рассмотрим сдвиг и растяжение
отрезка.

\subheader{Сдвиг}

Пусть \(f(x) \colon \segment{a}{b} \to \RR\), \(b - a = 2 \pi\). Тогда ряд Фурье
для \(f(x)\) не изменится. Заметим, что если \(\phi(x)\) периодична с периодом
\(T = 2 \pi\), то

\begin{equation*}
  \int_c^{c + 2 \pi} \phi(x) \dd x = \int_{-\pi}^{\pi} \phi(x) \dd x
\end{equation*}

По геометрическому смыслу получаем

\begin{equation*}
  \int_{-\pi}^{c}
  = \int_{\pi}^{c + 2 \pi}
  \implies
  \int_{c}^{c + 2 \pi}
  = \int_{c}^{\pi} + \int_{\pi}^{c + 2 \pi}
  = \int_{c}^{\pi} + \int_{-\pi}^{c}
  = \int_{-\pi}^{\pi}
\end{equation*}

При этом

\begin{equation*}
  \int_a^b \phi(x + T) \dd (x + T)
  = \int_a^b \phi(x) \dd x
\end{equation*}

Тогда разложение в ряд Фурье будет иметь вид

\begin{equation*}
  \begin{aligned}
    f(x) = \frac{a_0}{2}
      + \sum_{k = 1}^{\infty} \prh{a_k \cos k x + b_k \sin k x}
  \\
    a_k
    = \frac{1}{\pi} \int_{a}^{b} f(x) \cos k x \dd x
    = \frac{1}{\pi} \int_{-\pi}^{\pi} \under{f(x)}{f(x - m T)} \cos k x \dd x
    \qquad
    = \frac{1}{\pi} \int_{a}^{b} f(x) \sin k x \dd x
    = \frac{1}{\pi} \int_{-\pi}^{\pi} f(x) \sin k x \dd x
  \end{aligned}
\end{equation*}

\subheader{Растяжение}

Пусть \(f(x) \colon \segment{-l}{l} \to \RR\), \(f(x)\) периодична с периодом
\(T = 2 l\). Введем замену

\begin{equation*}
  \begin{aligned}
    x = \frac{l t}{\pi} & \qquad & t = \frac{\pi x}{l}
  \\
    x \in \segment{-l}{l} & \Rarr{} & t \in \segment{-\pi}{\pi}
  \end{aligned}
\end{equation*}

Тогда получим следующие коэффициенты.

\begin{equation*}
  a_k
  = \frac{1}{\pi} \int_{-\pi}^{\pi} f \prh{\frac{l t}{\pi}} \cos k t \dd t
  = \frac{1}{l} \int_{-\pi}^{\pi} f \prh{\frac{l t}{\pi}} \cos k t
    \dd \frac{l t}{\pi}
  = \frac{1}{l} \int_{-l}^{l} f \under{\prh{\frac{l t}{\pi}}}{x}
    \cos \frac{\pi k}{l} x \dd x
  = \frac{1}{l} \int_{-l}^{l} f (x) \cos \frac{\pi k}{l} x \dd x
\end{equation*}

И аналогично

\begin{equation*}
  b_k = \frac{1}{l} \int_{-l}^{l} f(x) \sin \frac{\pi k}{l} x \dd x
\end{equation*}

\begin{remark}
  Таким образом функции, определенные на произвольном отрезке \(\segment{a}{b}\)
  можно разложить в ряд Фурье используя сдвиг и растяжение.
\end{remark}

\begin{remark}
  Если функция определена на \(\segment{0}{l}\), то получить ее разложение можно
  дополнив четным или нечетным образом до функции на отрезке
  \(\segment{-l}{l}\).
\end{remark}