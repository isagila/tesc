\subsection{%
  Лекция \texttt{23.11.10}.%
}

\begin{theorem}[Необходимое и достаточное условие существование предела]
  Пусть \(z_n = x_n + i y_n\), тогда

  \begin{equation*}
    \exists \lim_{n \to \infty} z_n = L \in \CC
    \iff
    \begin{cases}
      \exists \lim_{n \to \infty} x_n = X \in \RR \\
      \exists \lim_{n \to \infty} y_n = Y \in \RR
    \end{cases}
  \end{equation*}
\end{theorem}

\begin{proof}
  \ness{}
  \begin{equation*}
    \begin{aligned}
      \exists \lim_{n \to \infty} x_n = X
      \iff
      \forall \epsilon_1 > 0 \given
      \exists N(\epsilon_1) \in \NN \given
      \forall n > N \colon
      \abs{x_n - X} < \epsilon_1
      \qquad \epsilon_1 = \frac{\epsilon}{\sqrt{2}}
    \\
      \exists \lim_{n \to \infty} y_n = Y
      \iff
      \forall \epsilon_2 > 0 \given
      \exists N(\epsilon_2) \in \NN \given
      \forall n > N \colon
      \abs{y_n - Y} < \epsilon_2
      \qquad \epsilon_2 = \frac{\epsilon}{\sqrt{2}}
    \\
      \sqrt{(x_n - X)^2 + (y_n - Y)^2}
      < \sqrt{\epsilon_1^2 + \epsilon_2^2}
      = \epsilon
    \\
      \forall \epsilon > 0 \given
      \exists N(\epsilon) \in \NN \given
      \forall n > N \colon
      \abs{z_n - \under{X + i Y }{z}} < \epsilon
      \iff
      z_n \Rarr{n \to \infty} z \in \CC
    \end{aligned}
  \end{equation*}

  \suff{}
  \begin{equation*}
    \begin{aligned}
      \lim_{n \to \infty} L \in \CC
      \bydef
      \forall \epsilon > 0 \given
      \exists N(\epsilon) \in NN \given
      \forall n > N \colon
      \abs{z_n - L} < \epsilon
      \qquad (L = L_x + i L_y)
    \\
      \abs{z_n - L}
      = \sqrt{(x_n - L_x)^2 + (y_n - L_y)^2} < \epsilon
      \implies
      \begin{cases}
        \abs{x_n - L_x} < \epsilon_x \\
        \abs{y_n - L_y} < \epsilon_y
      \end{cases}
      \iff
      \begin{cases}
        \lim_{n \to \infty} x_n = L_x \in \RR \\
        \lim_{n \to \infty} y_n = L_y \in \RR
      \end{cases}
    \end{aligned}
  \end{equation*}
\end{proof}

\subsubheader{4}{Функия комплексного переменного (ФКП)}

\begin{definition}
  Функция комплексного переменного это отображение \(f \colon D \to \CC\) такое,
  что

  \begin{equation*}
    \forall z \in D \subset \CC \given
    \exists \omega \in G \subset \CC \given
    \omega = f(z)
  \end{equation*}
\end{definition}

\begin{remark}
  Заметим, что не накладывается условие единственности значения функции, поэтому
  функции делятся на однозначные и многозначные.
\end{remark}

\begin{definition}
  Пусть \(\omega = f(z)\)~--- функция комплексного переменного. Если \(\forall
  z_1 \neq z_2\) выполняется \(f(z_1) \neq f(z_2)\), то функция \(f(z)\)
  называется однолистной. В противном случае она называется многолистной.
\end{definition}

\begin{remark}
  Функция взаимнооднозначна, если она однозначна и однолистна. В этом случае
  определена обратная функция \(z = f^{-1} (\omega)\).
\end{remark}

\begin{remark}
  Определение суперпозиции (сложной функции) аналогично определению вещественной
  суперпозиции.
\end{remark}

\begin{definition}{Предел функции в точке}
  Пусть \(f \colon D \to \CC\), тогда пределом функции \(f(z)\) в точке \(z_0\)
  называется

  \begin{equation*}
    L \in \CC = \lim_{z \to z_0} f(z)
    \bydef
    \forall \epsilon > 0 \given
    \delta > 0 \given
    \forall z \in \near{z_0}{\delta} \cap D \colon
    \abs{f(x) - L} < \epsilon
  \end{equation*}
\end{definition}

\begin{remark}
  Что означает \(z \to z_0\)? \(z\) стремится к \(z_0\) вдоль какого-либо пути
  \(l\) (\(z = \phi(t) + i \psi(t)\)). В определении предел существует при
  стремлении \(z \to z_0\) по любому пути.
\end{remark}

\begin{example}
  \begin{equation*}
    f(z) = \frac{1}{2 i} \prh{\frac{z}{\bar{z}} - \frac{\bar{z}}{z}}
  \end{equation*}

  Будем приближать \(z\) к \(z_0 = 0\) по путям \(\phi = const\).

  \begin{equation*}
    f(z)
    = \frac{1}{2 i} \prh{
      \frac{\rho e^{i \phi}}{\rho e^{-i \phi}}
      - \frac{\rho e^{-i \phi}}{\rho e^{i \phi}}
    }
    = \frac{1}{2 i} \prh{e^{2 i \phi} - e^{-2 i \phi}}
    = \sin 2 \phi
  \end{equation*}

  Таким образом, при фиксированном \(\phi\) получаем фиксированное число, при
  этом

  \begin{equation*}
    \lim_{z \to 0} f(z)
    = \lim_{z \to 0} \sin 2 \phi
  \end{equation*}

  заполняют весь отрезок \(\segment{-1}{1}\), поэтому нельзя говорить о
  существовании предела в общем смысле.
\end{example}

\begin{remark}
  Предел вдоль выбранного пути аналогичен одностороннему пределу вещественной
  функции. Для существования предела в общем смысле необходимо существование,
  конечность и равенство пределов по любому пути.
\end{remark}

\begin{definition}
  \begin{equation*}
    L \in \CC = \lim_{z \to \infty} f(z)
    \bydef
    \forall \epsilon > 0 \given
    \delta > 0 \given
    \forall z \in D, \abs{z} > 0 \colon
    \abs{f(x) - L} < \epsilon
  \end{equation*}
\end{definition}

\begin{definition}
  \begin{equation*}
    L = \lim_{z \to z_0} f(z) = \infty
    \bydef
    \forall \epsilon > 0 \given
    \delta > 0 \given
    \forall z \in \near{z_0}{\delta} \cap D \colon
    \abs{f(x)} > \epsilon
  \end{equation*}
\end{definition}

\begin{definition}{Непрерывность функции в точке} 
  \begin{equation*}
    f(z) \iscontd{z_0 \in D}
    \bydef
    \exists \lim_{z \to z_0} f(z) = f(z_0)
  \end{equation*}
\end{definition}

\begin{remark}
  Существует равносильное определение непрерывности функции в точке

  \begin{equation*}
    f(z) \iscontd{z_0 \in D}
    \bydef
    \Delta f(z) \Rarr{\Delta z \to 0} 0
    \qquad
    \begin{cases}
      \Delta z = z - z_0 \\
      \abs{\Delta z} = \sqrt{(x - x_0)^2 + (y - y_0)^2} \\
      \Delta f(z) = f(z_0 + \Delta z) - f(z_0)
    \end{cases}
  \end{equation*}
\end{remark}

\begin{remark}
  Непрерывность \(f(z)\) в точке \(z_0 = x_0 + i y_0\) равносильна непрерывности
  \(u(x, y) = \Re f(z)\) и \(v(x, y) = \Im f(z)\) в точках \(x_0\) и \(y_0\).
\end{remark}

Элементарные функции:

\begin{enumerate}
\item
  Линейная \(f(z) = a z + b\).

\item
  Степенная \(f(z) = z^n\) (\(n \in \NN\)).

\item
  Рациональная \(\display{f(z) = \frac{P_n (z)}{Q_m (z)}}\), где \(P_n (z)\),
  \(Q_m (z)\)~--- полиномы.

\item 
  Показательная \(f(z) = e^z = e^{x + i y} = e^{\rho (\cos \phi + i \sin
  \phi)}\).

\item
  Логарифм \(f(z) = \Ln z\). (определение ниже)

\item
  Тригонометрические, гиперболические.

  \begin{equation*}
    \begin{aligned}
      \sin z = \frac{e^{i z} + e^{-i z}}{2 i}
      & \qquad &
      \cos z = \frac{e^{i z} + e^{-i z}}{2}
    \\
      \sinh z = \frac{e^z - e^{-z}}{2} = -i \sin (i z)
      & \qquad &
      \cosh z = \frac{e^z - e^{-z}}{2} = \cos (i z)
    \end{aligned}
  \end{equation*}
\end{enumerate}

\begin{remark}
  \begin{equation*}
    e^{z + 2 i \pi k}
    = e^z e^{2 i \pi k}
    = e^z (\cos 2 \pi k + i \sin 2 \pi k)
    = e^z
  \end{equation*}

  Таким образом это многолистная, однозначная функция.
\end{remark}

\begin{remark}
  Логарифм \(\Ln z\) определим как обратную операцию для \(e^{\omega} = z\).

  \begin{equation*}
    \begin{aligned}
      \omega = u + i v
      \implies e^{u + i v} = z = \rho e^{i (\phi + 2 \pi k)}
      \qquad \phi = \arg z, \Arg z = \phi + 2 \pi k
    \\
      e^{u + i v} = \rho e^{i (\phi + 2 \pi k)}
      \implies \begin{cases}
        e^u = \rho \\
        v = \phi + 2 \pi k
      \end{cases}
      \implies \begin{cases}
        u = \ln \rho \\
        v = \arg z + 2 \pi k
      \end{cases}
    \\
      \Ln z = \omega = \ln q + i (\arg z + 2 \pi k)
    \end{aligned}
  \end{equation*}

  Получили однолистную, но многозначную функцию. Причем \(\ln z = \ln \abs{z} +
  i \arg z\) это главное значение логарифма.
\end{remark}

\begin{remark}
  Геометрический смыслы функций \(f(z) = a z + b\), \(f(z) = \frac{1}{z}\) и
  \(f(z) = z^2\) рассмотрим позже.
\end{remark}

\subsubheader{5}{Дифференцирование функции комплесконого переменного}

\begin{definition}
  Пусть \(f \colon D \to \CC\). Производной \(f(z)\) в точке \(z_0 \in D\)
  называется

  \begin{equation*}
    f'(z_0) 
    \bydef
    \lim_{z \to z_0} \frac{\Delta f}{\Delta z}
    = \lim_{\Delta z \to 0}  \frac{f(z_0 + \Delta z) - f(z_0)}{z - z_0} \in \CC
  \end{equation*}
\end{definition}

\begin{definition}
  Дифференцируемость в точке \(z_0\) равносильна существованию конечной
  производной \(f'(z_0)\). По определению дифференцируемость это представление

  \begin{equation*}
    \Delta f = f'(z_0) \Delta z + \smallo (\Delta z)
  \end{equation*}
\end{definition}

\begin{definition}
  Если функция в точке имеет непрерывную производную, то она называется
  аналитической.
\end{definition}

\begin{theorem}[Условия Коши-Римана (условия аналитичности функции)]
\label{thr:C-R-cond}
  \(f(z) \colon D \to \CC\) аналитическая в \(z_0 \in D\) тогда и только тогда,
  когда функции \(u(x, y) = \Re f(z)\) и \(v(x, y) = \Im f(z)\) дифференцируемы,
  имеют непрерывные производные в точках \(x_0 = \Re z_0\) и \(y_0 = \Im z_0\) и
  выполнены условия:

  \begin{equation*}
    \partder{u}{x} = \partder{v}{y}
    \qquad
    \partder{u}{y} = -\partder{v}{x}
  \end{equation*}
\end{theorem}

\begin{proof}
  \suff{}
  \begin{equation*}
    \begin{aligned}
      \exists f'(z_0) \in \CC
    \\
      \lim_{\Delta z \to 0} \frac{\Delta f}{\Delta z}
      = \lim_{\Delta z \to 0} \frac{f(z_0 + \Delta z) - f(z_0)}{\Delta z}
    \\
      = \lim_{\Delta z \to 0} \frac{
          \prh[\Big]{u(x_0 + \Delta x, y_0 + \Delta y)
            + i v(x_0 + \Delta x, y_0 + \Delta y)}
          -
          \prh[\Big]{u(x_0, y_0) + i v(x_0, y_0)}
        }{(x + i y) - (x_0 + i y_0)}
      = \dotsc
    \end{aligned}
  \end{equation*}

  \todo продолжим доказательство на следующей паре
\end{proof}

\begin{remark}
  В условиях теоремы \ref{thr:C-R-cond} \(f'(z_0)\) представима в виде

  \begin{equation*}
    f'(z_0) = \partder{u}{x} (x_0, y_0) + i \partder{v}{x} (x_0, y_0)
  \end{equation*}

  \todo перепроверить
\end{remark}