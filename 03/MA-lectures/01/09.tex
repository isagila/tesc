\subsection{%
  Лекция \texttt{23.10.27}.%
}

\subheader{Свойства коэффициентов}

Дана \(f(x)\) на \(\segment{-\pi}{\pi}\). Многочлены вида

\begin{equation*}
  \phi(x) = \frac{\alpha_0}{2} + \sum_{k = 1}^{\infty} \prh{
    \alpha_k \cos k x + \beta_k \sin k x
  }
\end{equation*}

приближают \(f(x)\). Среднеквадратичное отклонение примем за

\begin{equation*}
  \sigma^2 = \frac{1}{2 \pi} \int_{-\pi}^{\pi} \prh{f(x) - \phi(x)}^2 \dd x
\end{equation*}

Найдем коэффициенты \(\alpha_k\), \(\beta_k\) для \quote{хорошего} приближения
(\(\sigma^2\) минимально).

\begin{equation*}
  \sigma^2
  = \frac{1}{2 \pi} \int_{-\pi}^{\pi}
    \prh[\Big]{f^2 (x) - 2 f(x) \phi(x) + \phi^2 (x)}^2 \dd x
\end{equation*}

Это можно привести к виду

\begin{equation*}
  \sigma^2
  = \frac{1}{2 \pi} \int_{-\pi}^{\pi} f^2 (x) \dd x
    - \frac{a_0}{4} - \frac{1}{2} \sum_{k = 1}^n \prh{a_k^2 + b_k^2}
    + \frac{(a_0 - \alpha_0)^2}{4} + \sum_{k = 1}^{\infty}
      \prh{(a_k - \alpha_k)^2 + (b_k - \beta_k)^2}
\end{equation*}

Где \(a_k\), \(b_k\) это коэффициенты ряда Фурье для функции \(f(x)\). Таким
образом

\begin{equation*}
  \sigma^2_{\min}
  = \frac{1}{2 \pi} \int_{-\pi}^{\pi} f^2 (x) \dd x - \frac{1}{2} \prh{
    \frac{a_0}{2} + \sum_{k = 1}^n \prh{a_k^2 + b_k^2}
  }
\end{equation*}

при \(\alpha_k = a_k\) и \(\beta_k = b_k\). Отсюда, т.к. \(\sigma^2_{\min} \ge
0\) получаем неравенство, которое называется неравенством Бесселя

\begin{equation} \label{eq:bessel-inequality}
  \frac{1}{2 \pi} \int_{-\pi}^{\pi} f^2 (x) \dd x
  \ge
  \frac{a_0}{2} + \sum_{k = 1}^{\infty} \prh{a_k^2 + b_k^2}
\end{equation}

Мы может заменить \(n\) на \(\infty\), т.к. рассуждения выше справедливы для
любого \(n\). Таким образом, если \(f^2 (x)\) интегрируема и интеграл в левой
части неравенства существует и конечен, то ряд в правой части сходится (он
меньше конечного числа).

Если \(f(x)\) удовлетворяет условиям Дирихле и раскладывается по системе
\(\display{\set{\frac{1}{\sqrt{2 \pi}}, \frac{\cos x}{\sqrt{\pi}},
\frac{\sin x}{\sqrt{\pi}}, \dotsc}}\), то неравенство превращается в равенство,
которое называется равенством Парсенваля и при этом \(\sigma^2 \to 0\) при \(n
\to \infty\).

\begin{equation} \label{eq:parseval-identity}
  \frac{1}{2 \pi} \int_{-\pi}^{\pi} f^2 (x) \dd x
  =
  \frac{a_0}{2} + \sum_{k = 1}^{\infty} \prh{a_k^2 + b_k^2}
\end{equation}

\subheader{Комплексная форма ряда Фурье}

Известно, что

\begin{equation*}
  e^{i \phi} = \cos \phi + i \sin \phi
  \qquad
  e^{-i \phi} = \cos \phi - i \sin \phi
\end{equation*}

Используя эти равенства получаем

\begin{equation*}
  \cos n x = \frac{e^{i n x} + e^{-i n x}}{2}
  \qquad
  \sin n x = \frac{e^{i n x} - e^{-i n x}}{2}
\end{equation*}

Подставим это в общий вид ряда Фурье.

\begin{equation*}
  \begin{aligned}
    f(x) = \frac{a_0}{2} + \sum_{k = 1}^{\infty} \prh{
      a_n \cos n x + b_n \sin n x
    }
  \\
    f(x) = \frac{a_0}{2} + \sum_{k = 1}^{\infty} \prh{
      a_n \cdot \frac{e^{i n x} + e^{-i n x}}{2}
      + b_n \cdot \frac{e^{i n x} - e^{-i n x}}{2 i}
    }
  \\
    f(x) = \frac{a_0}{2} + \sum_{k = 1}^{\infty} \prh{
      e^{i n x} \prh{\frac{a_n}{2} + \frac{b_n}{2 i}}
      + e^{-i n x} \prh{\frac{a_n}{2} - \frac{b_n}{2 i}}
    }
  \\
    f(x) = \frac{a_0}{2} + \sum_{k = 1}^{\infty} \prh{
      e^{i n x} \cdot \frac{a_n - i b_n}{2}
      + e^{-i n x} \cdot \frac{a_n + i b_n}{2}
    }
  \end{aligned}
\end{equation*}

Введем новые обозначения

\begin{equation*}
  c_0 = \frac{a_0}{2}
  \qquad
  c_n = \frac{a_n - i b_n}{2}
  \qquad
  c_{-n} = \frac{a_n + i b_n}{2}
  \qquad
  (n > 0)
\end{equation*}

Поработаем с \(c_n\) и \(c_{-n}\).

\begin{equation*}
  \begin{aligned}
    c_n
    = \frac{1}{2} \prh{
      \frac{1}{\pi} \int_{-\pi}^{\pi} f(x) \cos n x \dd x
      - \frac{i}{\pi} \int_{-\pi}^{\pi} f(x) \cos n x \dd x
    }
    = \frac{1}{2 \pi} \int_{-\pi}^{\pi} f(x)
      \prh[\Big]{\cos n x - i \sin n x} \dd x
    = \frac{1}{2 \pi} \int_{-\pi}^{\pi} f(x) e^{-i n x} \dd x
  \\
    c_{-n}
    = \dotsc
    = \frac{1}{2 \pi} \int_{-\pi}^{\pi} f(x) e^{i n x} \dd x
  \\
    \implies
    c_n = \frac{1}{2 \pi} \int_{-\pi}^{\pi} f(x) e^{-i n x} \dd x
    \qquad
    (n \in \ZZ)
  \end{aligned}
\end{equation*}

Итого получаем следующий общий вид ряда Фурье в комплексной форме.

\begin{equation*}
  f(x) = \sum_{n \in \ZZ} c_n e^{-i n x}
\end{equation*}

\begin{remark}
  Если \(f(x)\) имеет период \(T = 2 l\), тогда

  \begin{equation*}
    f(x) = \sum_{n \in \ZZ} c_n e^{-i \frac{\pi n}{l} x}
    \qquad
    c_n = \frac{1}{2 l} \int_{-l}^l f(x) e^{-i \frac{\pi n}{l} x} \dd x
  \end{equation*}
\end{remark}

\begin{remark}
  \(\alpha_n = \frac{\pi n}{l}\) это волновые числа, а последовательность
  \(\seq{\alpha_n}\) это дискретный спектр. Получаем гармонику:

  \begin{equation*}
    e^{- i \frac{\pi n}{l} x}
    = \cos \frac{\pi n}{l} x - i \sin \frac{\pi n}{l} x
  \end{equation*}
\end{remark}

\subheader{Преобразование Фурье}

Рассмотрим функцию \(f \colon \RR \to \RR\), для которой
\(\display{\int_{-\infty}^{\infty} \abs{f(x)} \dd x \in \RR}\). Имеем

\begin{equation*}
  \begin{aligned}
    f(x)
    = \frac{1}{2 l} \int_{-l}^l f(t) \dd t
      + \frac{1}{l} \sum_{n = 1}^{\infty} \prh[\Bigg]{
        \cos n x \cdot \prh{\int_{-l}^l f(t) \cos \frac{\pi n}{l} t \dd t}
        + \sin  n x \cdot \prh{\int_{-l}^l f(t) \sin \frac{\pi n}{l} t \dd t}
      }
  \\
    f(x)
    = \frac{1}{2 l} \int_{-l}^l f(t) \dd t + \frac{1}{l} \sum_{n = 1}^{\infty}
      \int_{-l}^l f(t) \cos \prh{\frac{\pi n}{l} (t - x)} \dd t
  \\
    \begin{rcases}
      \dd \alpha = \frac{\pi}{l} \\
      \alpha_k = \frac{\pi k}{l} \\ 
      \dd \alpha
        = \Delta_k \alpha
        = \alpha_{k + 1} - \alpha_k
        = \frac{\pi}{l}
    \end{rcases}
    \Rarr{l \to \infty}
    \Delta_k \alpha = \dd a \to 0
  \\
    f(x)
    = \frac{1}{2 l} \int_{-l}^l f(t) \dd t + \frac{1}{\pi} \int_1^{\infty} \prh{
      \int_{-\infty}^{\infty} f(t) \cos (\alpha (t - x)) \dd t
    } \dd \alpha
  \\
    f(x)
    = \frac{1}{\pi} \int_0^{\infty} \int_{-\infty}^{\infty}
      f(t) \cos (\alpha (t - x)) \dd t \dd \alpha
  \end{aligned}
\end{equation*}

Последний полученный интеграл называется интегралом Фурье.

\begin{remark}
  Для интеграла Фурье выполнены свойства ряда Фурье (значения в точках разрыва
  функции, свойство четности и нечетности).
\end{remark}

Полученный интеграл можно разбить на части, тогда

\begin{equation*}
  \begin{aligned}
    f(x)
    = \frac{1}{\pi} \int_0^{\infty} \under{\prh{
      \int_{-\infty}^{\infty} f(t) \cos \alpha t \dd t
    }}{A(\alpha)} \cos \alpha x \dd x
    +
    \frac{1}{\pi} \int_0^{\infty} \under{\prh{
      \int_{-\infty}^{\infty} f(t) \sin \alpha t \dd t
    }}{B(\alpha)} \sin \alpha x \dd x
  \\
    f(x) = \int_0^{\infty} \prh[\Big]{
      A (\alpha) \cos \alpha x + B(\alpha) \sin \alpha x
    } \dd x
  \\
    f(x) = \frac{1}{2 \pi}
      \int_{-\infty}^{\infty} \dd \alpha
      \int_{-\infty}^{\infty} f(t) e^{-i \alpha (t - x)} \dd t
  \end{aligned}
\end{equation*}

Последний переход выполнен по формуле Эйлера. Выражение, получившееся в итоге,
называется преобразованием Фурье.
