\question{Поверхностный интеграл 1-го рода: определение, свойства, вычисление, геометрический и физический смысл.}

\begin{twocolumns}
  \input{figures/MI/13/surf_int.tex}
  \columnbreak

  Пусть \(f(x, y, z)\) это плотность распределения некоторой скалярной величины.
  Введена ДПСК, поверхность простая \(z = z(x, y)\).

  Элемент поверхности \(\dd \sigma\) вырезается координатными плоскостями
  \(x = const, y = const\). Выделим элементарную массу \(\dd  m\). Умножая
  среднюю плотность на размер элементарного участка получаем
  \(\dd m = f(x, y, z) \dd \sigma\). Полную массу получим 'суммированием':

  \begin{align*}
    m = \iint_{S} \dd m = \iint_{S} f(x, y, z) \dd \sigma
  \end{align*}

  Получили поверхностный интеграл 1-ого рода (по участку поверхности).
\end{twocolumns}

\begin{remark}
  Физический смысл поверхностного интеграла 1-ого рода вытекает из его
  построения: он равен массе участка неоднородной поверхности.
\end{remark}

\begin{remark}
  О математическом определении

  Поверхностный интеграл 1-ого рода можно определить математически аналогично
  уже рассмотренным интегралам:

  \begin{enumerate}
    \item Дробим \(S\) плоскостями \(x = const, y = const\) на элементарные
    участки \(\Delta \sigma_{i}\).

    \item В каждом участке выбираем среднюю точку
    \(M_{i}(\xi_{i}, \eta_{i}, \zeta_{i})\) и вычисляем \(f(M_{i})\).

    \item Составляем предел интегральных сумм и переходим к интегралу:

    \begin{align*}
      \lim_{\substack{n \to \infty \\ \tau \to 0}}
        \sum_{i = 1}^{n} f(\xi_{i}, \eta_{i}, \zeta_{i}) \Delta \sigma_{i}
    \end{align*}
  \end{enumerate}
\end{remark}

\begin{remark}
  О вычислении

  Введем параметризацию \(z = z(x, y)\) и спроецируем поверхность на плоскость
  \(Oxy\), т.е. \(D_{xy} = S_{\text{пр. }xy}\). Получим

  \begin{align*}
    \iint_{S} f(x, y, z) \dd \sigma
    = \iint_{D_{xy}} f(x, y, z(x, y))
      \sqrt{1 + (z'_{x})^2 + (z'_{y})^2} \dd x \dd y
  \end{align*}

  При необходимости можно вводить другую параметризацию и проектировать
  поверхность на другую координатную плоскость.
\end{remark}

\begin{remark}
  С помощью поверхностного интеграла 1-ого рода можно найти площадь поверхности
  следующим образом:

  \begin{align*}
    S_{\text{пов.}} = \iint_{S} \dd \sigma
  \end{align*}
\end{remark}