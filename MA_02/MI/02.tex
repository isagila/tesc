\question{Вычисление двойного интеграла. Кратный интеграл.}

\begin{theorem}\label{iint-to-rep}
  Сведение двойного интеграла к повторным

  \begin{align*}
    \iint_{D} f(x, y) \dd x \dd y =
      \int_{x_{1}}^{x_{2}} \dd x \int_{y_{1}(x)}^{y_{2}(x)} f(x, y) \dd y
  \end{align*}
\end{theorem}
\begin{proof}
  Пусть область \(D\) правильная в направлении \(Oy\).
  Найдем \(x_{1}\) и \(x_{2}\)~--- границы области для переменной \(x\).
  Далее будем 'идти' по оси \(x\) от \(x_{1}\) к \(x_{2}\).

  Рассмотрим момент, в котором \(x = const\). В этот момент \(y\) может
  меняться в диапазоне от \(y_{1}(x)\) до \(y_{2}(x)\), где \(y_{1}(x)\),
  \(y_{2}(x)\) это функции от \(x\), задающие 'верхнюю' и 'нижнюю' границы
  текущего отрезка в области \(D\) (для этого и требовалась правильность в
  направлении \(Oy\)). Значит мы можем вычислить площадь сечения как
  
  \begin{align*}
    \int_{y_{1}(x)}^{y_{2}(x)} f(x = const, y) \dd y
    = F(x = const, y) \bigg\vert_{y_{1}(x)}^{y_{2}(x)}
    = \breve{F}(x)
  \end{align*}

  Далее применим формулу для вычисления объема тела с известными площадями
  сечений \eqref{eq:rotate-int-V}:

  \begin{align*}
    V
    = \int_{x_{1}}^{x_{2}} \breve{F} \dd x
    = \int_{x_{1}}^{x_{2}} \left(
      \int_{y_{1}(x)}^{y_{2}(x)} f(x, y) \dd y
    \right) \dd x
    = \int_{x_{1}}^{x_{2}} \dd x \int_{y_{1}(x)}^{y_{2}(x)} f(x, y) \dd y
  \end{align*}
\end{proof}

\begin{remark}
  Полученный интеграл называется кратным (повторным).
\end{remark}

\begin{remark}
  Порядок интегрирования можно изменить, если область правильная в обоих
  направлениях.
  
  Если область правильная только в одном из направлений, то
  внутренний интеграл должен браться по переменной, соответствующей этому
  направлению.

  Если область неправильная ни в одном из направлений, то её необходимо разбить
  на части (пользуясь аддитивностью интегралов), каждая из которых должна быть
  правильной хотя бы в одном из направлений.
\end{remark}