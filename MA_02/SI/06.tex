\question{Интегрирование тригонометрических функций вида \(R(\sin^m x, \cos^n x)\), \(R(\sin mx, \cos nx)\).}

Рассмотрим интегралы вида \(\int \sin^m x \cos^m x \dd x\)

\begin{enumerate}
\item \(n\) или \(m\) нечетное

Пусть \(m\) нечетное, тогда \(m = 2k + 1\). Подставим это в исходный интеграл:

\begin{align*}
  \int \sin^m x \cos^m \dd x =
  \int \sin^m x \cos^{2k} \cos x \dd x =
  \int \sin^m x (1 - \sin^2 x)^k \dd (\sin x)
  \transition{t = \sin x}
  \int t^m x (1 - t^2)^k \dd t
\end{align*}

Получили интеграл от полинома \(\implies\) умеем его решать.

\item \(n\) и \(m\) четные

Обозначим \(n = 2p\), \(m = 2q\), тогда:

\begin{align*}
  \int \sin^m x \cos^m \dd x =
  \int (\sin^2 x)^p (\cos^2 x)^q \dd x =
  \int \left(\frac{1 - \cos 2x}{2}\right)^p
    \left(\frac{1 + \cos 2x}{2}\right)^q \dd x
\end{align*}

Далее раскрываем скобки и упрощаем. Получится либо первый случай
(с нечетной степенью), либо второй, но с меньшей степенью.
\end{enumerate}

Интегралы видов
\begin{itemize}
  \item \(\int \sin mx \sin nx \dd x\)
  \item \(\int \sin mx \cos nx \dd x\)
  \item \(\int \cos mx \cos nx \dd x\)
\end{itemize}
решаются при помощи использования тригонометрических формул, которые сводят
произведение к сумме/разности:

\begin{align*}
  \sin mx \sin nx = \frac{1}{2}\Big(\cos((m - n) x) - \cos((m + n) x)\Big) \\
  \sin mx \cos nx = \frac{1}{2}\Big(\sin((m - n) x) + \sin((m + n) x)\Big) \\
  \cos mx \cos nx = \frac{1}{2}\Big(\cos((m - n) x) + \cos((m + n) x)\Big) \\
\end{align*}

\todo На лекции были интегралы вида \(\int \sin^m x \cos^m \dd x\), а не 
\(R(\sin^m x, \cos^n x)\).