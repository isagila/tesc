\question{Приложения определенного интеграла: вычисление объемов тел с известными площадями сечений и тел вращения.}

Пусть дано некоторое тело и известны площади его сечений в плоскости
\(\perp Ox\), т.е. известна функция \(S(x)\), определяющая площадь сечения в
зависимости от \(x\). Построим интеграл:

\begin{enumerate}
  \item Дробление: отрезок \([a; b]\), где \(a\) и \(b\) это крайние точки тела,
    делится на подотрезки \([x_{i - 1}, x_{i}]\). Через \(x_{i}\) проводится
    плоскость \(\perp Ox\) и выделяется элементарный слой.

  \item Приближаем объем этого слоя объемом цилиндра с основанием
  \(S(\xi_{i})\), где \(\xi_{i}\) это некоторая средняя точка из отрезка
  \([x_{i - 1}, x_{i}]\).

  \item Составляем предел интегральных сумм и переходим к интегралу.
  
  \begin{align*}\label{eq:rotate-int-V}\tag{RV}
    V = \lim_{\substack{n \to \infty \\ \tau \to 0}}
      \sum_{i = 1}^{n} S(\xi_{i}) \Delta x_{i}
    \implies
    V = \int_{a}^{b} S(x) \dd x
  \end{align*}
\end{enumerate}

\begin{remark}
  Сечения обязательно должны быть \(\perp Ox\), в противном случае получится
  объем, умноженный на коэффициент наклона сечения по отношению к оси \(Ox\).
\end{remark}

Рассмотрим нахождение объема тел вращения.

Подставим в полученную выше формулу \(S_{sec} = S_{\circ} = \pi f(x)^2\).
Получим, что объем тела вращения равен:

\begin{align*}
  V = \pi \int_{a}^{b} f(x)^2 \dd x
\end{align*}

