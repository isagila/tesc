\question{Свойства решений ЛОДУ\(_2\): линейная независимость решений, определитель Вронского. Теоремы 1,2.}

Рассмотрим множество \(\Omega\) непрерывных функций с непрерывными производными
2ого порядка. Определим линейный дифференциальный оператор
\(\Linear{y} = y'' + py' + qy \to f(x)\).

\begin{definition}
  Будем называть функции \(y_{1}, \dotsc, y_{n}\) линейно-независимыми на
  отрезке \([a; b]\), если

  \begin{align*}
    \sum_{i = 1}^{n} c_{i} y_{i} = 0 \implies \forall c_{i} = 0
  \end{align*}
\end{definition}

\begin{definition}
  Определитель Вронского (вронскиан) \(\Wrn\) это определитель, составленный из
  \(n\) функций и всех их производных вплоть до \((n - 1)\)-ого порядка. Он
  имеет вид:

  \begin{align*}
    \Wrn = \begin{vmatrix}
      y_{1}  & \dotsc  & y_{n} \\
      \vdots & \ddots & \vdots \\
      y^{(n - 1)}_{1}  & \dotsc  & y^{(n - 1)}_{n}
    \end{vmatrix}
  \end{align*}
\end{definition}

\begin{lemma}\label{wrn-prop-1}
  Если два решения ЛОДУ\(_2\) линейно-зависимы на \([a; b]\), то их
  вронскиан на \([a; b]\) равен нулю.

  \begin{align*}
    \begin{rcases}
      \Linear{y_{1}} = 0 \\
      \Linear{y_{2}} = 0 \\
      y_{1} = \lambda y_{2}
    \end{rcases} \implies \Wrn = 0
  \end{align*}
\end{lemma}
\begin{proof}
  \begin{align*}
    \Wrn = \begin{vmatrix}
      y_{1} & y_{2} \\
      y'_{1} & y'_{2} \\
    \end{vmatrix} = \begin{vmatrix}
      \lambda y_{2} & y_{2} \\
      \lambda y'_{2} & y'_{2} \\
    \end{vmatrix} = \lambda \begin{vmatrix}
      y_{2} & y_{2} \\
      y'_{2} & y'_{2} \\
    \end{vmatrix} = 0
  \end{align*}
\end{proof}

\begin{lemma}\label{wrn-prop-2}
  Если два решения ЛОДУ\(_2\) линейно-независимы на \([a; b]\), то их
  вронскиан на \([a; b]\) не равен нулю.

  \begin{align*}
    \begin{rcases}
      \Linear{y_{1}} = 0 \\
      \Linear{y_{2}} = 0 \\
      y_{1} \neq \lambda y_{2}
    \end{rcases} \implies \Wrn \neq 0
  \end{align*}
\end{lemma}
\begin{proof}
  От противного
  \begin{align*}
    \lets \Wrn = 0 = \begin{vmatrix}
      y_{1} & y_{2} \\
      y'_{1} & y'_{2} \\
    \end{vmatrix} = y_{1} y'_{2} - y'_{1} y_{2} \mid \colon y_{1}^{2} \neq 0 \\
    \frac{y_{1} y'_{2} - y'_{1} y_{2}}{y_{1}^{2}} = 0 \\
    \left(\frac{y_{2}}{y_{1}}\right)' = 0 \\
    \frac{y_{2}}{y_{1}} = const \\
    y_{1} = \lambda y_{2}
  \end{align*}
  Получили противоречие.
\end{proof}

\begin{theorem}
  Линейная зависимость/независимость функций определяется равенством их
  вронскиана нулю.
\end{theorem}
\begin{proof}
  Следствие из \ref{wrn-prop-1} и \ref{wrn-prop-2}.
\end{proof}

\begin{remark}
  Для проверки набора функций на линейную зависимость/независимость лучше
  использовать именно вронскиан, а не непосредственное определение линейной
  зависимости функций на отрезке.
\end{remark}

\begin{theorem}\label{wrn-prop-3}
  Рассмотрим функции на отрезке \([a; b]\). Если на этом отрезке найдется точка,
  в которой вронскиан равен нулю, вронскиан будет равен нулю на всем отрезке.
  Дуально, если найдется точка, в которой вронскиан не равен нулю, то он будет
  не равен нулю на всем отрезке.

  \begin{align*}
    \exists x_{0} \in [a; b] \mid W(x_{0}) = W_{0} \neq 0
      \implies \forall x \in [a, b] \colon W(x) \neq 0 \\
    \exists x_{0} \in [a; b] \mid W(x_{0}) = W_{0} = 0
      \implies \forall x \in [a, b] \colon W(x) = 0 \\
  \end{align*}
\end{theorem}
\begin{proof}
  Пусть \(y_{1}\) и \(y_{2}\) это решения ДУ, тогда

  \begin{align*}
    \begin{cases}
      y_{2}'' + p y_{2}' + q y_{2} = 0 \mid \cdot \; y_{1} \\
      y_{1}'' + p y_{1}' + q y_{1} = 0 \mid \cdot \; y_{2}
    \end{cases} - \\
    (y_{1} y_{2}'' - y_{2} y_{1}'') + p (y_{1} y_{2}' - y_{1}' y_{2}) = 0
  \end{align*}

  Заметим, что выражение в левой скобке это \(\Wrn'\), а во правой~--- \(\Wrn\):

  \begin{align*}
    \Wrn = y_{1} y_{2}' - y_{1}' y_{2} \\
    \Wrn'
    = (y_{1} y_{2}' - y_{1}' y_{2})'
    = y_{1}' y_{2}' + y_{1} y_{2}'' - y_{1}'' y_{2} - y_{1}' y_{2}'
    = y_{1} y_{2}'' - y_{1}'' y_{2}
  \end{align*}

  Подставим это в полученное ранее уравнение:

  \begin{align*}
    (y_{1} y_{2}'' - y_{2} y_{1}'') + p (y_{1} y_{2}' - y_{1}' y_{2}) = 0 \\
    \Wrn' + p \Wrn = 0 \\
    \Wrn = c_{1} e^{-\int p \dd x} \\
    \Wrn(x_{0})
    = c_{1} e^{-\int_{x_{0}}^{x_{0}} p \dd x}
    = c_{1}
    = \Wrn_{0}
    \\
    \Wrn(x)
    = c_{1} e^{-\int_{x_{0}}^{x} p \dd x}
    = \Wrn_{0} e^{-\int_{x_{0}}^{x} p \dd x}
  \end{align*}

  Таким образом, если \(\Wrn_{0} = 0\), то \(\Wrn(x) = 0\) на всем отрезке
  \([a; b]\). Дуально, если \(\Wrn_{0} \neq 0\), то т.к. второй множитель всегда
  больше нуля, то \(\Wrn(x) \neq 0\).
\end{proof}

\begin{remark}
  Таким образом, чтобы узнать равен ли вронскиан нулю на отрезке, достаточно
  узнать его значение в одной произвольной точке этого отрезка.
\end{remark}
