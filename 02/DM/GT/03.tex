\question{Пути и расстояния.}

\begin{definition}
  Маршрут (\textit{walk}) это чередующаяся последовательность из вершин и ребер.
\end{definition}

\begin{definition}
  Путь (\textit{trail}) это маршрут без повторяющихся ребер.
\end{definition}

\begin{definition}
  Цепь (\textit{path}) это маршрут без повторяющихся вершин.
\end{definition}

\begin{remark}
  Для приведенных выше терминов существуют замкнутые версии (т.е. начальная и
  конечная вершины совпадают):

  \begin{itemize}
    \item Замкнутый маршрут (\textit{closed walk})
    \item Замкнутый путь (\textit{closed trail}, \textit{circuit})
    \item Замкнутая цепь/цикл (\textit{closed path}, \textit{cycle})
  \end{itemize}
\end{remark}

\begin{definition}
  Для маршрута (пути, цепи) это количество ребер в нем.
\end{definition}

\begin{definition}
  Обхват (\textit{girth}) это длина кратчайшего цикла в графе.
\end{definition}

\begin{definition}
  Расстояние \(\dist (u, v)\) между двумя вершинами \(u\) и \(v\) это длина
  кратчайшего пути \(u \leadsto v\).
\end{definition}

\begin{definition}
  Длина наибольшего из кратчайших расстояний от данной вершины \(v\) до
  остальных вершин называется эксцентриситетом
  \(\eccentricity{v} = \maxL_{u \in V} \dist{u, v}\)
\end{definition}

\begin{definition}
  Радиусом графа называется минимальный из эксцентриситетов его вершин
  \(\graphRadius{G} = \minL_{v \in V} \eccentricity{v}\).
\end{definition}

\begin{definition}
  Диаметром графа называется максимальный из эксцентриситетов его вершин
  \(\graphDiameter{G} = \maxL_{v \in V} \eccentricity{v}\).
\end{definition}

\begin{definition}
  Множество вершин, эксцентриситет которых равен радиусу называются центром
  графа \(\graphCenter{G} = \{ v \mid \eccentricity{v} = \graphRadius{G}\}\).
\end{definition}
