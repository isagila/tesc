\question{Алгоритм Флойда-Уоршелла.}

\underline{Цель}: ищем кратчайшие пути от всех вершин до всех.

\underline{Оценка по времени}: \(O(V^{3})\)

\underline{Алгоритм}:
\begin{enumerate}
  \item Рассматриваем каждую из вершин \(k \in V\).
  
  \item На каждой итерации проходим по всем возможным парам вершин
  \(\Pair{i, j}\) и пытаемся прорелаксировать путь \(i \leadsto j\) через
  вершину \(k\).

  \item Релаксация пути через вершину \(k\) выглядит так: пусть дана пара
  вершин \(\Pair{i, j}\). Тогда обновим расстояние от \(i\) до \(j\) следующим
  образом: \(d_{ij} = \min(d_{ij}, d_{ik} + d_{kj})\).
\end{enumerate}


\underline{О корректности}:

Построим следующую динамику: пусть \(d[k, i, j]\) это длина кратчайшего пути из
вершины \(i\) в вершину \(j\), содержащего только вершины с индексами меньше
\(k\). Тогда получаем:
\begin{itemize}
  \item Изначально:
  \begin{itemize}[label = \textbullet]
    \item \(\forall (i, j) \in E \colon d[0, i, j] = w_{ij}\)
    \item \(\forall i \in V \colon d[0, i, i] = 0\)
    \item Во всех остальных случаях \(d[k, i, j] = +\infty\)
  \end{itemize}

  \item Переход \(d[k + 1, i, j] = \min \Big(
    d[k, i, j], d[k, i, k]  + d[k, k, j]
  \Big)\)

  Он вытекает из следующих соображений: 

  \begin{itemize}[label = \textbullet]
    \item Путь \(i \leadsto j\) нельзя улучшить через вершину \(k\), тогда
    \(d[k + 1, i, j] = d[k, i, j]\).

    \item Путь \(i \leadsto j\)  можно улучить через вершину \(k\), тогда
    \(d[k + 1, i, j] = d[k, i, k] + d[k, k, j]\).
  \end{itemize}

  Из этих двух случаев мы выбираем наименьший.
\end{itemize}

Динамика корректна по построению, а алгоритм Флойда-Уоршелла это следствие из
этой динамики: мы убираем индекс \(k\), чтобы оптимизировать затрачиваемую
память.

