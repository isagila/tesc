\question{Теорема Татта.}
 
\begin{definition}
  \(1\)-фактор графа \(G\) это \(1\)-регулярный подграф графа \(G\).
\end{definition}

\begin{theorem} 
  Теорема Татта.
  
  Граф \(G\) содержит \(1\)-фактор тогда и только тогда, когда
  
  \begin{align*}
    \forall S \subseteq V \colon k_{0}(G - S) \le \abs{S}
  \end{align*}

  где \(k_{0}(G - S)\) это число нечетных (т.е. содержащих нечетное количество
  вершин) компонент графа \(G - S\).
\end{theorem}
\begin{proof}
  \(\implies\) Пусть \(S \subseteq V\). Если граф \(G - S\) не содержит нечетных
  компонент, то тогда \(k_{0}(G - S) = 0\) и очевидно, что
  \(k_{0}(G - S) \le \abs{S}\).

  Предположим, что \(k_{0}(G - S) = k \ge 1\). Обозначим
  \(G_{1}, \dotsc, G_{k}\) нечетные компоненты графа \(G - s\). Т.к. граф \(G\)
  содержит \(1\)-фактор (обозначим его \(F\)), то некоторое ребро из \(F\)
  должно быть инцидентно вершине из \(G_{i}\) и вершине из \(S\). Таким образом
  \(k_{0}(G - S) \le \abs{S}\).

  \(\impliedby\)
  Пусть \(\forall S \subseteq V \colon k_{0}(G - S) \le \abs{S}\). Для
  \(S = \varnothing\) получаем, что \(k_{0}(G - S) = k_{0}(G) = 0\). Таким
  образом каждая компонента графа \(G\)~--- четная, а значит и сам граф \(G\)
  содержит четное число вершин. Далее по индукции по числу вершин покажем, что
  любой граф четного размера, обладающий этим свойством, содержит \(1\)-фактор.

  \textbf{База}: \(K_{2}\) имеет \(1\)-фактор.

  \textbf{Переход}: пусть теорема выполнена для всех четных графов,
  удовлетворяющих условиям, и содержащих менее \(n\) вершин. Рассмотрим граф
  \(G\), удовлетворяющий условиям, и имеющий четный размер \(n\).

  Заметим, что любая нетривиальная компонента графа \(G\) содержит вершину,
  которая не является точкой сочленения, поэтому существуют такие
  \(R_{j} \subset V\) для которой \(k_{0}(G - R_{j}) = \abs{R_{j}}\). Пусть
  \(S\) это максимальное по мощности множество \(R_{j}\). Обозначим
  \(G_{1}, \dotsc, G_{k}\) нечетные компоненты \(G - S\), тогда
  \(\abs{S} = k \ge 1\).

  Покажем, что \(G - S\) содержит только нечетные компоненты, которые мы
  обозначали \(G_{1}, \dotsc, G_{k}\). Пусть есть четная компонента \(G_{p}\),
  содержащая вершину \(u\), которая не является точкой сочленения.
  Тогда рассмотрим множество \(S_{p} = \cup \{ u \} \), мощность которого равна
  \(k + 1\). Получаем, что \(k_{0}(G - S_{p}) = \abs{S_{p}} = k + 1\), а это
  невозможно.

  \todo доказательство не дописано :(
\end{proof}
