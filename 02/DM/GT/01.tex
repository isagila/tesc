\question{Основные определения.}

\begin{definition}
  Граф это упорядоченная пара вида \(G = \Pair{V, E} \), где
  \(V = \{ v_{1}, \dotsc, v_{n} \}\) это множество вершин, а
  \(E = \{ e_{1}, \dotsc, e_{n} \}\) это множество ребер.
\end{definition}

\begin{definition}
  Порядком графа называется число его вершин \(\abs{V}\).
\end{definition}

\begin{definition}
  Размером графа называется количество его ребер \(\abs{E}\).
\end{definition}

У неориентированного графа \(E \subseteq V^{(2)}\), где \(V^{(2)}\) это
множество всех непустых подмножеств размера не более двух. Таким образом каждое
ребро обозначается как \(\{ u, v \} \in V^{(2)}\) (если \(u \neq v\)) или
\(\{ v \} \in V^{(2)}\) (если данное ребро это петля).

У ориентированного графа \(E \subseteq V^{2}\), т.е. каждое ребро это
упорядоченная пара вида \(\Pair{u, v}\). Ребра ориентированного графа также
называют дугами.

\begin{definition}
  Петля это ребро, которое соединяет вершину саму с собой.
\end{definition}

\begin{definition}
  Мультиребра это ребра, у которых общее начало и общий конец.
\end{definition}

\begin{definition}
  Граф называется простым, если в нем нет мультиребер и петель.
\end{definition}

\begin{definition}
  Мультиграф это граф с мультиребрами.
\end{definition}

\begin{definition}
  Псевдограф это мультиграф с петлями.
\end{definition}

\begin{definition}
  Гиперграф это граф, в котором ребро может соединять несколько вершин
  одновременно.
\end{definition}

\begin{definition}
  Нулевой граф это граф, который не содержит вершин.
\end{definition}

\begin{definition}
  Пустой граф это граф, которые не содержит ребер.
\end{definition}

\begin{definition}
  Граф-синглтон (тривиальный граф) это граф состоящий только из одной вершины.
\end{definition}

\begin{definition}
  Полный граф \(K_{n}\) это простой граф, в котором каждая пара различных вершин
  соединена ребром.
\end{definition}

\begin{definition}
  Взвешенный граф \(G = \Triple{V, E, w}\) это граф, в котором каждому ребру
  сопоставляется некоторое числовое значение (вес), которое определяется весовой
  функцией \(w \colon E \to \text{Num}\).
\end{definition}

\begin{definition}
  Подграфом графа \(G = \Pair{V, E}\) называется граф \(G' = \Pair{V', E'}\)
  такой, что \(V' \subseteq V\) и \(E' \subseteq E\).
\end{definition}

\begin{definition}
  Подграф называется остовным, если он содержит все вершины исходного графа.
\end{definition}

\begin{definition}
  Индуцированный подграф это подграф, который получается из некоторого
  подмножества вершин исходного графа \(V' \subseteq V\) и \textbf{всех} ребер
  исходного графа, соединяющих эти вершины.
\end{definition}

\begin{definition}
  Две вершины называются смежными (соседними), если между ними есть ребро.
\end{definition}

Для иллюстрации отношения смежности обычно используют матрицу смежности. Это
квадратная матрица размера \(V \times V\), ячейки которой содержат:

\begin{itemize}
  \item \(0\) или \(1\) ~--- для простых графов
  \item \(-1, 0, 1\) ~--- для ориентированных графов
  \item \(\NN\) ~--- для взвешенных графов
\end{itemize}

Однако описанные выше правила не строгие: в разных задачах числа в матрице
смежности могу обозначать разные вещи.

\begin{definition}
  Вершина и ребро называются инцидентными, если вершина является одним из концов
  ребра.
\end{definition}

Для иллюстрации отношения инцидентности обычно используют матрицу инцидентности.
Это прямоугольная матрица размера \(V \times E\), строки которой соответствуют
вершинам, а столбцы~--- ребра. Ячейки этой матрицы могут содержать:

\begin{itemize}
  \item \(1\) если вершина и ребро инциденты и \(0\) в противном случае~--- для
    неориентированных графов.
  
  \item \(-1\), если ребро выходит из данной вершин, и \(1\) если входит,
    \(0\) если ребро и вершина неинцидентны~--- для ориентированных графов.
\end{itemize}

Петля в матрице инцидентности обычно обозначается двойкой. Но, как и в случае
с матрицей смежности, описанные правила не являются строгими.

\begin{definition}
  Степенью вершины \(\deg u\) называется количество инцидентных ей ребер
  (петли учитываются дважды).
\end{definition}

Со степенью вершины также связны такие понятия как:

\begin{itemize}
  \item Минимальная степень вершины в графе
    \(\minDegree{G} = \minL_{v \in V} \deg v\).

  \item Максимальная степень вершины в графе
    \(\maxDegree{G} = \maxL_{v \in V} \deg v\).
\end{itemize}

\begin{lemma}
  Лемма о рукопожатиях.

  \begin{align*}
    \sum_{v \in V} \deg v = 2 \abs{E}
  \end{align*}
\end{lemma}
\begin{proof}
  Т.к. каждое ребро инцидентно ровно двум ребрам, то сложив все степени вершин
  мы учтем каждое ребро дважды (по одному разу для каждой из его концевых
  вершин).
\end{proof}

\begin{definition}
  Граф называется \(r\)-регулярным, если степень каждой из его вершин равна
  \(r\).
\end{definition}

\begin{definition}
  Граф называется планарным, если его можно изобразить на плоскости без
  пересечения ребер.
\end{definition}

\begin{definition}
  Граф называется плоским, если он \textit{уже} изображен на плоскости без
  пересечения ребер.
\end{definition}

\begin{theorem}
  Теорема Эйлера для графов

  Для планарных графов выполняется соотношение

  \begin{align*}
    \abs{V} - \abs{E} + \abs{F} = 2
  \end{align*}

  где \(\abs{V}\)~--- количество вершин, \(\abs{E}\)~--- количество ребер, а
  \(\abs{F}\)~--- количество граней.
\end{theorem}
\begin{proof}
  Индукция по количеству граней.

  \textbf{База}: \(F = 2\). Тогда граф представляет собой многоугольник с
  \(\abs{V}\) вершинами и при этом \(\abs{V} = \abs{E} \implies\) равенство
  выполняется.

  \textbf{Переход}: пусть теорема верна для графа в котором \(F\) граней.
  Добавим новую грань, для этого проведем по внешней границе некоторую простую
  цепь, которая будет содержать \(r - 1\) вершин и \(r\) ребер. Подставим эти
  изменения в формулу и упростим:

  \begin{align*}
    (\abs{V} + r - 1) - (\abs{E} + r) + (\abs{F} + 1) \stackrel{?}{=} 2 \\
    \abs{V} - \abs{E} + \abs{F} \stackrel{?}{=} 2
  \end{align*}

  Это верно по предположению индукции.

  \todo figure
\end{proof}