\question{Формальные языки. Операции над формальными языками.}

\begin{definition}
  Формальный язык \(\Lang\) это некоторое подмножества множества всех слов из
  заданного алфавита \(\Sigma\).

  \begin{align*}
    \Lang \subseteq \Sigma^{*}
    \qquad
    \Sigma^{*} = \sum\limits_{k = 0}^{\infty} \Sigma^{k}
  \end{align*}
\end{definition}

Формальный язык может задан перечислением (как множество слов) либо с помощью
грамматики (\(\approx\) формулой).

Т.к. формальный язык по сути является множеством слов, то с ним можно
производить операции характерные для множеств:

\begin{itemize}
  \item Объединение \(\Lang_{1} \cup \Lang_{2}\)
  \item Пересечение \(\Lang_{1} \cap \Lang_{2}\)
  \item Дополнение \(\neg \Lang_{1}\)
\end{itemize}

Однако помимо этих операций с языками можно производить и другие:

\begin{itemize}
  \item Конкатенация \(
    \Lang_{1} \cdot \Lang_{2}
    = \{ x + y \mid x \in \Lang_{1}, y \in \Lang_{2}\}
  \)

  Эта операция порождает новый язык, в котором каждое слово является
  конкатенацией произвольного слова из первого языка с произвольным словом из
  второго языка.
  
  В общем случае она некоммутативна и выполняется равенство
  \(\abs{\Lang_{1} \cdot \Lang_{2}} \le \abs{\Lang_{1}} \cdot \abs{\Lang_{2}}\)

  \item Возведение в степень \(\Lang^{k}\)
  
  Это сокращенная запись для конкатенации языка самим с собой \(k\) раз.

  Стоит отметить, что \(L^{0} = \{ \epsilon \}\), где \(\epsilon\) это
  специальное 'пустое' слово нулевой длины.

  \item Звезда Клини \(\Lang^{*}\)
  
  Это объединение всевозможных степеней языка
  \(\Lang^{*} = \bigcup\limits_{k = 0}^{\infty} L^{k}\)
\end{itemize}
