\question{Теорема Клини.}

\begin{theorem}
  Множество автоматных языков равно множеству регулярных языков:

  \begin{align*}
    AUT = REG
  \end{align*}
\end{theorem}
\begin{proof}
  \(\implies\) Покажем, что \(AUT \subseteq REG\).
  
  Пусть дан НКА. Будем считать, что его 'ребра' являются регулярными
  выражениями. Таким образом задача состоит в том, чтобы сжать автомат до
  состояния, в котором останется только начальное и конечные состояния.
  Будем выполнять две операции:

  \begin{itemize}
    \item Сжатие ребер: заменим переходы \(q_{1} \xrightarrow{c_{1}} q_{2}\) и
    \(q_{1} \xrightarrow{c_{2}} q_{2}\) на переход
    \(q_{1} \xrightarrow{c_{1} | c_{2}} q_{2}\).

    \item Сжатие вершин: если есть пара переходов вида
    \(q_{1} \xrightarrow{c_{1}} q_{2} \xrightarrow{c_{3}} q_{3}\),
    то удалим их и добавим новый переход
    \(q_{1} \xrightarrow{c_{1} c_{2}^{*} c_{3}} q_{3}\),
    где \(c_{2}\) это символ на петле \(q_{2}\) (если такой петли нет, то эта
    часть просто опускается).
  \end{itemize}

  Выполняем эти две операции пока не останутся только стартовое и финишные
  состояния, т.е.
  \(q_{0} \xrightarrow{\alpha_{i}} f_{i} \; \forall f_{i} \in F\).
  Тогда искомым регулярным выражением будет
  \((\alpha_{1} | \dots | \alpha_{m})\).

  \(\impliedby\) Покажем, что \(REG \subseteq AUT\).
  
  Для любого регулярного выражения можно построить \(\epsilon\)-НКА по алгоритму
  Томпсона (см. \ref{thompson-construction}).
\end{proof}
