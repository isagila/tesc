\question{Производящие функции.}

\begin{definition}
  Производящая функция это способ задания бесконечной последовательности с
  помощью коэффициентов степенного ряда.
\end{definition}

Таким образом последовательность \(a_{0}, a_{1}, \dotsc, a_{n}\) будет
представлена в виде порождающей функции следующим образом:

\begin{align*}
  G(x)
  = a_{0} x^{0} + a_{1} x^{1} + \dotsc + a_{n} x^{n}
  = \sum_{n = 0}^{\infty} a_{n} x^{n}
\end{align*}

\begin{remark}
  Переменная \(x\) в порождающих функциях не является переменной в обычно
  понимании этого слова, т.е. не имеет смысла подставлять вместо неё какое-либо
  число и вычислять 'значение' производящей функции.
\end{remark}

\begin{remark}
  С производящими функциями можно проводить арифметические операции, как и с
  обычными функциями: складывать, вычитать, умножать, дифференцировать,
  интегрировать и т.п.
\end{remark}

\underline{Пример \#01}:

С помощью производящих функций можно решать рекуррентные соотношения.

Пусть требуется найти замкнутую формулу для рекуррентного соотношения
\(a_{n} = 4 a_{n - 1} - 3 a_{n - 2}, a_{0} = -4, a_{1} = -2\)

\begin{enumerate}
  \item Записываем условие в виде системы
  
  \begin{align*}
    \begin{cases}
      a_{0} = -4 \\
      a_{1} = -2 \\
      a_{n} = 4 a_{n - 1} - 3 a_{n - 2}
    \end{cases}
  \end{align*}

  \item Умножаем каждой из уравнений на \(z\) в степени, соответствующий
  порядковому номеру члена последовательности

  \begin{align*}
    \begin{cases}
      a_{0} z^{0} = -4 z^{0} \\
      a_{1} z^{1} = -2 z^{1} \\
      a_{n} z^{n} = (4 a_{n - 1} - 3 a_{n - 2}) z^{n}
    \end{cases}
  \end{align*}

  \item Складываем полученные равенства
  
  \begin{align*}
    a_{0} z^{0} + a_{1} z^{1} + \sum_{n = 2}^{\infty} a_{n} z^{n} =
    -4 - 2z + \sum_{n = 2}^{\infty} (4 a_{n - 1} - 3 a_{n - 2}) z^{n}
  \end{align*}

  \item То, что получилось в левой части обозначаем \(G(z)\). Это порождающая
  функция. То, что получилось справа разбиваем на отдельные суммы

  \begin{align*}
    G(z) = -4 - 2z + \sum_{n = 2}^{\infty} 4 a_{n - 1} z^{n} - 
      \sum_{n = 2}^{\infty} 3 a_{n - 2} z^{n}
  \end{align*}

  \item Далее задача заключается в том, чтобы выразить полученные суммы через 
  производящую функцию. ДЛя этого из каждой суммы вынесем число, а также \(z\) в
  такой степени, чтобы оставшаяся степень совпадала с порядковым номером члена
  последовательности

  \begin{align*}
    G(z) = -4 - 2z + 4z \sum_{n = 2}^{\infty} a_{n - 1} z^{n - 1} - 
      3 z^{2} \sum_{n = 2}^{\infty} a_{n - 2} z^{n - 2}
  \end{align*}

  \item Далее поменяем пределы в полученных суммах
  
  \begin{align*}
    G(z) = -4 - 2z + 4z \sum_{n = 1}^{\infty} a_{n} z^{n} - 
      3 z^{2} \sum_{n = 0}^{\infty} a_{n} z^{n}
  \end{align*}

  \item Заметим, что левая сумма это \(G(z)\) без первого слагаемого, а правая
  сумма в точности равна \(G(z)\).

  \begin{align*}
    G(z) = -4 - 2z + 4z (G(z) + 4) - 3z^2 G(z)
  \end{align*}

  \item Выразим \(G(z)\) в явном виде
  
  \begin{align*}
    G(z) (1 - 4z + 3z^2) = -4 + 14z \\
    G(z) = \frac{14z - 4}{3z^2 - 4z + 1}
  \end{align*}

  \item Далее нужно разложить эту дробь на простейшие. Это чисто математическая
  задача, которую можно решить (например) методом неопределенных коэффициентов
  либо методом сокрытия Хэвисайда.

  \begin{align*}
    G(z) = \frac{-5}{1 - z} + \frac{1}{1 - 3z}
  \end{align*}

  \item После этого нужно 'свернуть' полученные дроби используя формулу
  
  \begin{align*}
    \boxed{
      \frac{1}{(1 - \alpha z)^{k + 1}}
      = \sum_{n = 0}^{\infty} \alpha^{n} \binom{n + k}{k} z^{n}
    }
  \end{align*}

  Числители дробей можно вынести за скобку, поэтому получаем

  \begin{align*}
    G(z) = -5 \cdot \sum_{n = 0}^{\infty} z^{n}
      + \sum_{n = 0}^{\infty} 3^{n} z^{n}
  \end{align*}

  \item По определению производящей функции \(n\)-ый член последовательности
   это коэффициент перед \(z^{n}\). Смотрим на получившиеся и собираем
   коэффициенты

   \begin{align*}
     a_{n} = -5 \cdot 1 + 3^{n}
   \end{align*}
\end{enumerate}

Ответ: \(a_{n} = 3^{n} - 5\)

\underline{Пример \#02}:

С помощью производящих функций можно находить количество ограниченных
композиций, т.е. число целочисленных решений уравнений вида

\begin{align*}
  x_{1} + \dotsc + x_{n} = n \\
  l_{i} \le x_{i} \le u_{i}
\end{align*}

Для этого нужно для каждого \(x_{i}\) составить производящую функцию вида

\begin{align*}
  G_{i}(z) = z^{l_{i}} + \dotsc + z^{h_{i}}
\end{align*}

Полученные функции надо перемножить. Ответом будет являются коэффициент перед
\(z^{n}\). Рассмотрим на примере следующей задачи:

Сколько существует способов разделить \(25\) одинаковых печенек на четверых, при
условии, что каждый должен получить как минимум \(3\), но не более \(7\)ми
печенек?

По описанным выше шагам получим четыре одинаковых порождающих функции:

\begin{align*}
  G_{i}(z) = z^{3} + z^{4} + z^{5} + z^{6} + z^{7}
\end{align*}

Значит ответом будет являться

\begin{align*}
  [z^{25}] \Big( z^{3} + z^{4} + z^{5} + z^{6} + z^{7} \Big)^{4}
\end{align*}

Ответ: \(20\)


