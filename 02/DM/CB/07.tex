\question{Принцип включения-исключения.}

\begin{definition}
  Принцип включений-исключений это комбинаторный принцип, который позволяет
  вычислить мощность объединения конечного числа множеств, которые могут
  пересекаться друг с другом.
\end{definition}

Введем обозначения:
\begin{itemize}
  \item \(X\)~--- данное множество элементов
  \item \(\abs{X} = n\)
  \item \(P_{1}, \dotsc, P_{m}\)~--- 'плохие' свойства
  \item \(X_{i} = \{ x \in X \mid x \text{ cодержит свойство } P_{i} \}\)
  \item \(S \subseteq [m]\)~--- некоторое подмножество свойств
  \item \(
    N(S)
    = \bigcap\limits_{i \in S} X_{i}
    = \{ x \in X \mid x \text{ cодержит все свойства из } S \}
  \)
\end{itemize}

\begin{theorem}
  \begin{align*}
    \abs{X \setminus (X_{1} \cup \dotsc \cup X_{m})}
    = \sum_{S \subseteq [m]} (-1)^{\abs{S}} \cdot \abs{N(S)}
  \end{align*}
\end{theorem}
\begin{proof}
  Возьмем произвольный \(x \in X\). Возможны два случая:
  \begin{enumerate}
    \item \(x\) не содержит ни одного свойства \(P_{1}, \dotsc, P_{m}\), тогда
    \(x \in N(\varnothing)\) и \(x \notin N(S) \; \forall S \neq \varnothing\).
    Значит \(x\) 'добавляет' \(1\) к общей сумме.

    \item \(x\) содержит некоторый набор свойств из \(k\) свойств
    \(T \subseteq [m]\).
    Значит \(x \in S\) только тогда, когда \(S \subseteq T\). Получаем

    \begin{align*}\label{PIE-1}\tag{1}
      \sum_{S \subseteq T} (-1)^{\abs{S}} \cdot 1
      = \sum_{i = 1}^{k} \binom{k}{i} (-1)^{i}
    \end{align*}

    Заметим, что по биномиальной теореме

    \begin{align*}
      (1 + (-1))^{k}
      = \sum_{i = 1}^{k} \binom{k}{i} (-1)^{i} \cdot 1^{k - i}
      = \sum_{i = 1}^{k} \binom{k}{i} (-1)^{i}
      = 0
    \end{align*}

    Подставляя это в \ref{PIE-1} получаем, что \(x\) ничего не вносит в итоговую
    сумму.
  \end{enumerate}
\end{proof}
