\question{Сочетания.}

\begin{definition}
  Неупорядоченное размещение \(k\) элементов множества \(\Sigma\) это
  мультимножество \(S\) размера \(k\).
\end{definition}

\begin{definition}
  \(k\)-сочетание (\(k\)-подмножество) это неупорядоченное размещение \(k\)
  различных элементов из \(\Sigma\).
\end{definition}

Множество всех \(k\)-подмножеств обозначается \(\binom{\Sigma}{k}\), если
\(\abs{\Sigma} = n\), то получаем \(\binom{n}{k} = C_{n}^{k}\).

Для подсчета количества \(k\)-подмножеств воспользуемся тем, что каждое
\(k\)-подмножество имеет \(k!\) перестановок:

\begin{align*}
  \abs{P(n, k)} = \binom{n}{k} \cdot k!
  \implies \binom{n}{k} = \frac{n!}{k! \cdot (n - k)!}
\end{align*}
