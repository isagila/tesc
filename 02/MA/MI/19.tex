\question{Скалярное и векторное поля: определения, геометрические характеристики. Дифференциальные и интегральные характеристики полей (определения).}

\begin{definition}
  Скалярная функция \(u = u(x, y, z) \colon \RR^3 \to R\) называется скалярным
  полем.
\end{definition}

\begin{definition}
  Тройка скалярных функций \(P(x, y, z)\), \(Q(x, y, z)\), \(R(x, y, z)\),
  действующих из \(\RR^3\) в \(\RR\) определяют векторное поле, т.е. векторную
  величину \(\vec{F} = (P, Q, R)\), действующую в каждой точке пространства.
\end{definition}

\bgroup
\def\arraystretch{1.5}
\begin{tabularx}{\linewidth}{
  >{\hsize = 0.18\hsize}X |
  >{\hsize = 0.41\hsize}X |
  >{\hsize = 0.41\hsize}X
}
  & Скалярное поле
  & Векторное поле
  \\ \hline
  Геометрические характеристики
  &
    Линии (поверхности) уровня \(u(x, y) = const\).
  &
    Векторные линии и векторные трубки
  \\ \hline
  Дифференциальные характеристики
  & 
    Производная по направлению и градиент

    \begin{align*}
      \frac{\partial u}{\partial s} = \grad u \cdot \vec{s}_{0}\\
      \vec{\text{grad}} u = \grad u = \left(
        \frac{\partial u}{\partial x};
        \frac{\partial u}{\partial y};
        \frac{\partial u}{\partial z}
      \right)
    \end{align*}
  &
    Дивергенция и ротор(вихрь)

    %% если поставить здесь align*, то все сломается
    \[
      \div \vec{F} \bydef 
        \frac{\partial P}{\partial x} + 
        \frac{\partial Q}{\partial y} +
        \frac{\partial R}{\partial z}
      \bydef \grad \cdot \vec{F}
    \]

    \[
      \rot \vec{F} \bydef 
        \begin{vmatrix}
          \vec{i} & \vec{j} & \vec{k} \\
          \grad_{x} & \grad_{y} & \grad_{z} \\
          \vec{F}_{x} & \vec{F}_{y} & \vec{F}_{z}
        \end{vmatrix}
      \bydef \grad \times \vec{F}
    \]
  \\ \hline
  Интегральные характеристики
  &
   \todo Кажется их нет
  &
    Поток и циркуляция

    \begin{align*}
      \flow 
      \bydef \iint_{S} \vec{F} \vec{n}_{0} \dd \sigma
      \bydef \iint_{S} \vec{F}_{n} \dd \sigma
      \\
      \circulation = \oint_{L} \vec{F} \vec{\dd l}
    \end{align*}
\end{tabularx}
\egroup

\begin{definition}
  Векторная линия векторного поля это кривая, в каждой точке которой вектор поля
  \(\vec{F}\) является касательным к ней.
\end{definition}

\begin{definition}
  Объединении непересекающихся векторных линий называется векторной трубкой.
\end{definition}

\begin{remark}
  Отыскание векторных линий сводится к нахождению интегральных кривых из условия

  \begin{align*}
    \frac{\dd x}{P} = \frac{\dd y}{Q} = \frac{\dd z}{R}
  \end{align*}
\end{remark}

\begin{example}
  Дано векторное поле \(\vec{F} = y \vec{i} - x \vec{j}\). Требуется найти
  векторную линию, проходящую через \(M_{0}(1, 0)\).
  
  В данном примере \(P(x, y) = y\), а \(Q(x, y) = -x\). Составим ДУ и решим его:

  \begin{align*}
    \frac{\dd x}{y} = \frac{\dd y}{-x}
    \implies x \dd x + y \dd y = 0 
    \implies \frac{x^2}{2} + \frac{y^2}{2} = C
  \end{align*}

  Подставим начальные условия \(y(1) = 0\):

  \begin{align*}
    \begin{cases}
      x^2 + y^2 = 2C \\
      1 + 0 = 2C
    \end{cases}
    \implies x^2 + y^2 = 1
  \end{align*}
\end{example}

\begin{definition}
  Оператор Гамильтона/Набла

  \begin{align*}
    \grad = \left(
      \frac{\partial}{\partial x};
      \frac{\partial}{\partial y};
      \frac{\partial}{\partial z}
    \right)
  \end{align*}
\end{definition}

\begin{definition}
  Оператор Лапласа (лапласиан)

  \begin{align*}
    \laplacian
    = \grad^2
    = \left(
      \frac{\partial^2}{\partial x^2};
      \frac{\partial^2}{\partial y^2};
      \frac{\partial^2}{\partial z^2}
    \right)
  \end{align*}
\end{definition}


