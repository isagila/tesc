\question{Обыкновенное дифференциальное уравнение (ДУ): задача о радиоактивном распаде и задача о падении тела. Определение ДУ, решения ДУ и их геометрический смысл. Задача Коши.}

\begin{definition}
  Обыкновенным ДУ\(_{n}\) называется

  \begin{align*}
    F(x, y, \dotsc, y^{(n)}) = 0
  \end{align*}
\end{definition}

\begin{remark}
  'Обыкновенное' означает не в частных дифференциалах, т.е. \(y(x)\) это
  функция одно переменной.
\end{remark}

\begin{definition}
  Порядком ДУ называется наивысший порядок входящей в него производной.
\end{definition}

\begin{definition}
  Решением ДУ\(_{n}\) является функция, которая обращает его в верное равенство.
\end{definition}

\begin{definition}
  Кривые, соответствующие решениям ДУ называются интегральными кривыми.
\end{definition}

\begin{definition}
  Если решение ДУ задано неявно \(\phi(x, y(x)) = 0\), то 
  \(\phi(x, y(x)) = 0\) называется интегралом ДУ.
\end{definition}

\begin{definition}
  Решение ДУ с неопределенными константами \(c_{i}\) называется общим решением
  (общим интегралом) ДУ.  
\end{definition}

\begin{definition}
  Решение ДУ с определенными константами \(c_{i}\) называется частным решением
  (частным интегралом) ДУ.
\end{definition} 

\begin{definition}
  Система из ДУ\(_{n}\) и \(n\) начальных условий вида

  \begin{align*}
    \begin{cases}
      \text{ДУ}_n \\
      y_0 = y(x_0) \\
      y'_0 = y'(x_0) \\
      \dots \\
      y^{(n - 1)}_0 = y^{(n - 1)}(x_0)
    \end{cases}
  \end{align*}

  называется задачей Коши. \(n\) начальных условий необходимы для определения
  \(n\) констант \(c_{i}\).
\end{definition}

\begin{remark}
  Подробнее про задачу Коши и геометрический смысл решений рассказано в
  \ref{ex-un-Cp}.
\end{remark}

\underline{Задача о радиоактивном распаде}: 

Пусть есть \(Q\) грамма урана, скорость распада которого зависит от его массы с
некоторым коэффициентом \(k\). Требуется вывести формулу для подсчета массы
урана в момент времени \(t\).

Составим ДУ и решим его:

\begin{align*}
  \frac{\dd Q}{\dd t} = -k Q \mid \colon Q \neq 0
  \\
  \frac{\dd Q}{Q \dd t} = -k
  \iff
  \frac{\dd \ln Q}{\dd t} + k = 0
  \\
  \frac{\dd \ln Q + k \dd t}{\dd t} = 0
  \iff
  \frac{\dd (\ln Q + k t)}{\dd t} = 0
  \\
  \ln Q + kt = c_{1} \\
  Q = \widehat{c_{1}} e^{-k t}
\end{align*}

Из полученных интегральных кривых выбираем одну, которая соответствует
заданным начальным условиям.

\underline{Задача о падении тела}: 

Тело свободно падает вниз с заданной начальной скоростью. Требуется вывести
закон движения (закон изменения координат с течением времени).

Составим ДУ и решим его:

\begin{align*}
  \vec{F} = m \vec{a} = m \vec{g} \implies a = y''(t) = g \\
  \frac{\dd^{2} y(t)}{\dd t^{2}} = g
  \implies \frac{\dd v(t)}{\dd t} = g 
  \implies v(t) = g t + c_{1} \\
  \frac{\dd y(t)}{\dd t}  = v(t) = g t + c_{1}
  \implies y(t) = \frac{g t^{2}}{2} + c_{1} t + c_{2}
\end{align*}

Как и в первой задаче здесь получено общее решение. Константы можно найти
подстановкой начальных условий.
