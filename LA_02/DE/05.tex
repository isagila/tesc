\question{Линейное уравнение первого порядка. Метод Лагранжа.}

\begin{definition}
  Линейным однородным уравнением первого порядка (ЛОДУ\(_1\)) называется
  уравнение вида
  \begin{lequation}{lode-def}
    y' + p(x) y = 0
  \end{lequation}
\end{definition}

ЛОДУ\(_1\) является уравнением с разделяющими переменными, поэтому оно решается
следующим образом:

\begin{lequation}{lode-sln}
  y' + p(x) y = 0 \\
  \frac{\dd y}{\dd x} = -p(x) y \\
  \frac{\dd y}{y} = -p(x) \dd x \\
  \overline{y} = C \cdot \underbrace{e^{-\int p(x) \dd x}}_{y_1}
\end{lequation}

\begin{remark}
  При решении данного уравнения мы поделили на \(y \neq 0\). Заметим, что
  \(y = 0\) также является решением ЛОДУ\(_1\), однако оно получаемо из общего
  решения при \(C = 0\).
\end{remark}
  
\begin{definition}
  Линейным неоднородным уравнением первого порядка (ЛНДУ\(_1\)) называется
  уравнение вида
  \begin{lequation}{lhde-def}
    y' + p(x) y = q(x), \hspace{10pt} q(x) \neq 0
  \end{lequation}
\end{definition}

\textbf{Метод Лагранжа} (метод вариации произвольной постоянной) для решения
ЛНДУ\(_1\):

\begin{enumerate}
  \item Найдем частное решение \(y_{1}\) соответствующего однородного уравнения.
  \item Будем искать решение ЛНДУ\(_1\) в виде \(y(x) = y_{1}(x) \cdot C(x)\).
  Преобразуем ДУ в соответствии с этой заменой
  \begin{lequation}{lhde-sln}
    y' + p(x) y = q(x) \\
    y_{1}'(x) C(x) + y_{1}(x) C'(x) + p(x) y_{1}(x)  C(x) = q(x) \\
    y_{1}(x) C'(x) + C(x) 
    \underbrace{\Big( y_{1}'(x) + p(x) y_{1}(x) \Big)}_{= 0}
    = q(x) \\
    y_{1}(x) C'(x) = q(x) \\
    C(x) = \int \frac{q(x)}{y_{1}(x)} \dd x + C
  \end{lequation}
  \item Подставим найденную функцию \(C(x)\) в \(y(x) = y_{1}(x) \cdot C(x)\).
\end{enumerate}

\todo Уравнение Бернулли, Клеро, Риккати и пр.

