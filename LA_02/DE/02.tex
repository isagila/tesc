\question{Уравнение с разделяющимися переменными.}

\begin{definition}
  Уравнение вида
  \begin{lequation}{sep-var}
    m(x) N(y) \dd x + M(x) n(y) \dd y = 0
  \end{lequation}
  называется уравнением с разделяющимися переменными.
\end{definition}

Для решения таких уравнений необходимо разделить обе части на \(M(x) N(y)\),
перенести одно из слагаемых в правую часть, после чего проинтегрировать обе
части.

\begin{lequation}{sep-var-sln}
  m(x) N(y) \dd x + M(x) n(y) \dd y = 0 \\
  \frac{m(x)}{M(x)} \dd x + \frac{n(y)}{N(y)} \dd y = 0 \\
  \int \frac{m(x)}{M(x)} \dd x = - \int \frac{n(y)}{N(y)} \dd y 
\end{lequation}

\begin{remark}
  В случае, если \(M(x) = 0\) или \(N(y) = 0\), то уравнение решается
  непосредственным интегрированием.
\end{remark}

\begin{remark}
  Решения вида \(x = const, y = const\) не всегда получаемы из общего решения.
\end{remark}

