\question{Собственные числа и собственные векторы оператора. Теоремы о диагональной матрице оператора.}

\begin{definition}
  Пусть дан оператор \(\opA \colon V^{n} \to V^{n}\) с матрицей \(A\) в
  некотором базисе \(\Basis = \{ \basis_i \}_{i = 1}^{n}\). Тогда многочлен

  \begin{lequation}{lo-ch-pol}
    \det (A - \lambda E) 
  \end{lequation}

  относительно \(\lambda \in \RR\) называется характеристическим многочленом.
\end{definition}

\begin{definition}
  Пусть дан оператор \(\opA \colon V^{n} \to V^{n}\). Подпространство
  \(U \subseteq V^{n}\) называется \textit{инвариантным}, если

  \begin{lequation}{lo-inv-subspace}
    \forall x \in U \colon \opA x \in U
  \end{lequation}
\end{definition}

\begin{definition}
  Пусть дан оператор \(\opA \colon V^{n} \to V^{n}\) с матрицей \(A\) в
  некотором базисе \(\Basis = \{ \basis_i \}_{i = 1}^{n}\).
  
  \(x \neq 0 \in V^{n}\) называется собственным вектором для оператора \(\opA\),
  если

  \begin{lequation}{lo-eival-def}
    \exists \lambda \in \CC \colon \opA x = \lambda x
  \end{lequation}

  Тогда \(\lambda\) называется собственным числом (собственным значением)
  оператора \(\opA\).
\end{definition}

\begin{theorem}\label{lo-ei-roots}
  Собственные числа оператора являются корнями характеристического многочлена
  \(\det(A - \lambda E)\).
\end{theorem}
\begin{proof}
  \(\implies\) Пусть \(\lambda\) -- собственное число, тогда
  
  \begin{lequation}{lo-ei-roots-proof-1}
    \exists x \neq 0 \in V^{n} \mid \opA x = \lambda x \\
    \opA x = \lambda x
    \implies A x = (\lambda E) x
    \implies (A - \lambda E) x = 0
  \end{lequation}

  Теперь рассмотрим оператор
  \(\opB \colon V^{n} \to V^{n}, \opB = \opA - \lambda I\):

  \begin{lequation}{lo-ei-roots-proof-2}
    (A - \lambda E) x = 0 \implies \opB x = 0 \\
    x \neq 0 \in \Ker \opB \implies \dim \Ker \opB > 0 \\
    \begin{cases}
      \dim \Ker \opB + \dim \Im \opB = n \; (\ref{lo-sum-dim}) \\
      \dim \Ker \opB > 0
    \end{cases}
    \implies \dim \Im \opB < n
    \\
    \dim \Im \opB < n
    \implies \Rang \opB < n
    \implies \Rang B < n
    \\
    \Rang B < n
    \implies \Rang (A - \lambda E) < n
    \implies \det (A - \lambda E) = 0
  \end{lequation}

  \(\impliedby\) Аналогичные рассуждения, но в обратную сторону
  \begin{lequation}{lo-ei-roots-proof-3}
    \det (A - \lambda E) = 0 
    \implies \Rang (A - \lambda E) < n
    \implies \Rang B < n
    \\
    \Rang B < n
    \implies \Rang \opB < n
    \implies \dim \Im \opB < n
    \\
    \begin{cases}
      \dim \Ker \opB + \dim \Im \opB = n \; (\ref{lo-sum-dim}) \\
      \dim \Im \opB < n
    \end{cases}
    \implies \dim \Ker \opB > 0
    \\
    \dim \Ker \opB > 0
    \implies \exists x \neq 0 \colon \opB x = 0
    \\
    \opB x = 0
    \implies (A - \lambda E) x = 0
    \implies \opA x = \lambda x
  \end{lequation}
\end{proof}

\begin{definition}
  Полученное в процессе доказательства \ref{lo-ei-roots} уравнение

  \begin{lequation}{lo-ch-eq}
    (A - \lambda E) x = 0
  \end{lequation}

  называют характеристическим (вековым) уравнением.
\end{definition}

\begin{definition}
  Базис, составленный из собственных векторов, называют собственным базисом.
\end{definition}

\begin{theorem}
  Матрица оператора в собственном базисе диагональна.
\end{theorem}
\begin{proof}
  Матрица оператора в некотором базисе это коэффициенты разложения образов
  базисных векторов по этому же базису. Рассмотрим первый базисный вектор:

  \begin{lequation}{op-eibs-proof-1}
    \begin{cases}
      \opA \basis_1 = a_{1,1} \basis_{1} + \dots + a_{n,1} \basis_{n} \\
      \opA \basis_1 = \lambda_{1} \basis_1 \\
    \end{cases}
    \implies
    \begin{cases}
      a_{1,1} = \lambda_{1}, \\
      a_{i,1} = 0 \; \forall i \neq 1
    \end{cases}
  \end{lequation}

  Аналогично можно рассмотреть все оставшиеся базисные векторы. Таким образом
  матрица оператора в базисе из собственных векторов будет иметь вид

  \begin{lequation}{op-eibs-proof-1}
    A = \begin{pmatrix}
      \lambda_{1} & 0           & \dots  & 0           \\
      0           & \lambda_{2} & \dots  & 0           \\
      \vdots      & \vdots      & \ddots & \vdots      \\
      0           & 0           & \dots  & \lambda_{n} \\
    \end{pmatrix}
  \end{lequation}
\end{proof}