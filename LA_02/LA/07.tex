\question{Матрица линейного оператора. Преобразование матрицы при переходе к новому базису.}

Пусть дан оператор \(\opA \colon V^{n} \to V^{n}\) и \(x, y \in V^{n}\),
\(\opA x = y\).

Выделим в \(V^{n}\) базис, разложим \(x\) по этому базису. После чего применим
к нему оператор \(\opA\):

\begin{lequation}{lo-mtx-1}
  \Basis = \{ \basis_{1}, \dots, \basis_{n} \} \\
  x = x_{1} \basis_{1} + \dots + x_{n} \basis_{n} \\
  y = x_{1} \opA \basis_{1} + \dots + x_{n} \opA \basis_{n} \\
\end{lequation}

Далее применим оператор к каждому из базисных векторов:

\begin{lequation}{lo-mtx-2}
  \opA \basis_{i} = a_{1,i} \basis_{1} + \dotsc + a_{n, i} \basis_{n} \\
  y
  = x_{1} \Big( a_{1,1} \basis_{1} + \dotsc + a_{n,1} \basis_{n} \Big)
  + \dots
  + x_{n} \Big( a_{1,n} \basis_{1} + \dotsc + a_{n,n} \basis_{n} \Big) \\
  y
  = \basis_{1} \Big( x_{1} a_{1,1} + \dotsc + x_{n} a_{1,n} \Big)
  + \dots
  + \basis_{n} \Big( x_{1} a_{n,1} + \dotsc + x_{n} a_{n,n} \Big)
\end{lequation}

Заметим, что \(y\) также можно разложить по базису. Составим СЛАУ и запишем её
в матричном виде:

\begin{lequation}{lo-mtx-3}
  \begin{Bmatrix}
    x_{1} a_{1,1} & + \dotsc + & x_{n} a_{1,n} = & y_{1} \\
    \vdots        & \ddots    & \vdots          & \vdots \\
    x_{n} a_{n,1} & + \dotsc + & x_{n} a_{n,n} = & y_{n} 
  \end{Bmatrix} \iff AX = Y
\end{lequation}

\begin{definition}
  Матрицей оператора \(\opA\) \textbf{в данном базисе} называется матрица
  составленная из столбцов-коэффициентов разложения образов базисных векторов
  по этому же базису.
\end{definition}

\begin{remark}
  Если \(A^{-1} = A^{T}\), то матрица оператора называется ортогональной.
\end{remark}
